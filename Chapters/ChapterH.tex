% Chapter Template

\chapter{Conclusion} % Main chapter title

\label{chap:Chapter7} % Change X to a consecutive number; for referencing this chapter elsewhere, use \ref{ChapterX}

We have shown, thoroughly and comprehensively, that the proposed methods are superior to existing methods. In some cases there is a greater than 12\% increase in the accuracy of the results over a greater variation of fluorescence images. We have suceeded in moving a few steps closer to fully automated analysis. However, that is not to say that there isn't any limitations with the proposed methods.

%----------------------------------------------------------------------------------------
%	SECTION 1
%----------------------------------------------------------------------------------------
\section{Limitations}

\begin{enumerate}
	\item The remapping function presented in \Cref{sec:contrastcorrection} has 5 tuning parameters. Although there is greater freedom in tuning the curve, the added complexity in parameter finding is not welcomed.
	
	\item The preprocessing scheme presented in \Cref{chap:Chapter4} has a total of 14 tuning parameters. This makes it difficult to devise parameter estimation methods.
	
	\item In the propsosed parameter settings for the proposed parameter estimation method for ACWE graph cut segmentation in \Cref{sec:cvgc_weightingandparameterestimation}, we maintained a proportional relation between $\lambda_0$ and $\lambda_1$. Similarly, the $\alpha$ parameter remains constant and is not subject to appropriate variation with the image. 
\end{enumerate}

%----------------------------------------------------------------------------------------
%	SECTION 2
%----------------------------------------------------------------------------------------
\section{Future Work and Extensions}

\begin{enumerate}
	\item The parameters for the remapping function can be estimated from the image. This will allow the preprocessing scheme to be more adaptable over a greater variety of images and reduce manual parameter tuning.
	
	\item For the proposed parameter estimation scheme, we used simple mathematical relations. More sophisticated, comprehensive and accurate mathematical relations might allow for better segmentation results and applicability to a large class of images.
	
	\item Devise a method for automatic seeding and use those seeds as input for the propsed method in \Cref{sec:interactiveproposedweighting}. This will allow an interactive segmentation technique to become automatic.
	
	\item The parameter estimation technique in \Cref{sec:cvgc_weightingandparameterestimation} can be extended to colour images.
	
	\item The energy function proposed \Cref{sec:interactiveproposedweighting} can be extended to colour images.
\end{enumerate}

