% Chapter 1

\chapter{Introduction} % Main chapter title

\label{chap:Chapter1} % For referencing the chapter elsewhere, use \ref{chap:Chapter1} 

%----------------------------------------------------------------------------------------

% Define some commands to keep the formatting separated from the content 
\newcommand{\keyword}[1]{\textbf{#1}}
\newcommand{\tabhead}[1]{\textbf{#1}}
\newcommand{\code}[1]{\texttt{#1}}
\newcommand{\file}[1]{\texttt{\bfseries#1}}
\newcommand{\option}[1]{\texttt{\itshape#1}}

%----------------------------------------------------------------------------------------

\section{Literature Review}
The Literature Review is here.

%----------------------------------------------------------------------------------------

\section{Outline and Contributions}
The introduction is here.

%----------------------------------------------------------------------------------------

\section{Thesis Overview}

The remainder of the thesis outline.\\
\\
\textbf{\hyperref[chap:Chapter2]{Chapter~\ref*{chap:Chapter2}}} is where we cover the mathematical foundation to Graph Cut image segmenation.\\
\\
\textbf{\hyperref[chap:Chapter3]{Chapter~\ref*{chap:Chapter3}}} is where we cover the mathematical foundation to Graph Cut image segmenation.\\
\\
\textbf{\hyperref[chap:Chapter4]{Chapter~\ref*{chap:Chapter4}}} is where we cover the mathematical foundation to Graph Cut image segmenation.\\
\\
\textbf{\hyperref[chap:Chapter5]{Chapter~\ref*{chap:Chapter5}}} is where we cover the mathematical foundation to Graph Cut image segmenation.\\
\\
\textbf{\hyperref[chap:Chapter6]{Chapter~\ref*{chap:Chapter6}}} is where we cover the mathematical foundation to Graph Cut image segmenation.\\
\\
\textbf{\hyperref[chap:Chapter7]{Chapter~\ref*{chap:Chapter7}}} is where we cover the mathematical foundation to Graph Cut image segmenation.\\
\\
\textbf{\hyperref[chap:Chapter8]{Chapter~\ref*{chap:Chapter8}}} concludes the thesis with suggestions for further work.
