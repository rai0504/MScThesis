% Chapter 1

\chapter{Introduction} % Main chapter title

\label{chap:Chapter1} % For referencing the chapter elsewhere, use \ref{Chapter1} 

%----------------------------------------------------------------------------------------

% Define some commands to keep the formatting separated from the content 
\newcommand{\keyword}[1]{\textbf{#1}}
\newcommand{\tabhead}[1]{\textbf{#1}}
\newcommand{\code}[1]{\texttt{#1}}
\newcommand{\file}[1]{\texttt{\bfseries#1}}
\newcommand{\option}[1]{\texttt{\itshape#1}}

%----------------------------------------------------------------------------------------

\section{What is Image Segmentation}

definition\\
types: region, edge\\
human segmentation -> Gestalt Groupings\\
Machine segmentation\\
Good segmentation vs Bad Segmentation\\
Goal of Image segmentation\\
Differences between bottom-up and top-down image segmentation\\
number of labels (2 labels -> binarization or binary segmentation, etc)

%----------------------------------------------------------------------------------------

\section{Gestalt Theory of Visual Perception}

A psychological view of visual perception. The aim here is to give a brief realisation of the current understandings of human visual perception. Since image segmentation is predominantly guided by subjective human perception, it is wholesome to understand, at least briefly, what human perception is all about; at least to our current understanding.

[It would be good to get a few experts to manually segment the same images and do a similarity comparison. This will prove that even experts in the same field are subject to their own interpretation of an image. Also compare the manual segmentation to people that are not experts in the field. Discuss, how trustworthy are manual segmentations?]

Gestalt - movement in experimental psychology. Developed in Germany We percieve objects as well-origanised patterns rather than seperate components. Gestalt is a theory that the brain operates wholistically, with self-organising tendencies. "The whole is greater than the sum of its parts." Illusory Contours - The Kanisza triangle as figure ground illusory contours. Three main principles: Grouping(proximity, similarity, continuity, closure), Goodness of Figures, Figure/Ground Relationships.

Gestalt - The German word mean Organised Whole

Goodness of Figure, or the Law of Pragnaz. Pragnaz is the German word for Pregnant, but in the sense of pregnant with meaning, not with child.

Figure/Ground Relationships: Figure-Foreground, Ground-Background, Contours-"belong" to the Figure. Reversible Figure/Ground Relationship.

Problems with Gestalt Theory: 
- It is a phenomenological approach.
- Some terms are vague. E.g What is the simplest organisation?

\begin{definition}[Contrast]
	When perception is influenced by comparison. Three types:\\
	Brightness Contrast\\
	Colour Contrast\\
	Size Contrast
\end{definition}

\begin{definition}[Context]
	When a stimulus can be interpreted in more than one way, the context resolves the ambiguity.
\end{definition}

\begin{definition}[Figure Ground]
	Separation of an image into figure and ground.
\end{definition}

\begin{definition}[Closure]
	We tend to see figures as whole even though lines enclosing them are incomplete.
\end{definition}

\begin{definition}[Good Continuation, or Good Figure]
	Where lines intersect, we tend to see them as continuing along their previous course, rather than suddenly changing direction. As a result we tend to decompose figures into their simplest components.
\end{definition}

\begin{definition}[Perceptual Consistancies]
	4 types:\\
	Shape Consitency: We tend to see object as holding its essential shape even though the shape of its image changes with our view of it.\\
	Size Consistency: An object appears to retain its essential size event though its image size changes with distance.\\
	Brightness/Lightness Consistency: Object seem to retain about the same brightness or lightness under widely differing levels of illumination.\\
	Colour Consistency: Object appear to retain their essential colour even though illuminated by somewhat differently coloured lights.
\end{definition}

\begin{definition}[Principles of Grouping]
	The principles by which  you recognise object as belonging to the same group. They include:\\
	Similarity: Object are recognised as belonging to the same group when they have a similar appearance.\\
	Proximity: We percieve object as belonging to the same group based on their relative distances from one another. \\
	Common Fate: We percieve object as belonging to the same group when the same things are happening to them.
\end{definition}

%----------------------------------------------------------------------------------------

\section{Energy Minimisation}

Brief energy minimisation and what it is. Huge number of problems in vision are inference problems which can be found from energy minimisation. The relation of the concept of energy in physics is not important but useful.

Vision problems are much more complex involving hundreds or even millions of interdependant variables. Some energies precisely model the desired inference problem while some are coarse approximations. Some energies are easy to optimise while others are known to be NP-hard. Once an accurate energy and satisfying algorithm are available, the associated inference problem is essentially solved.

Many of the most important developments in computer vision began with a proposal for a better energy, better algorithm or a combination of both.

%----------------------------------------------------------------------------------------

\section{Labelling Problems}

What is a labelling problem? Type of labelling problems. What is needed for a labelling problem. What is a discrete labelling problem in vision? What are the type of labels (semantic or related to geometry).

Data driven criteria - Data influences the outcome. These preferences are dereived from machine learning.

Regularisation criteria - When we explicitely prefer some kinds  of labelling over others, these criteria are called regularisers. The most prominent in computer vision are spatially coherent labellings. The idea of smoothness in preference to noisy. We know from experience that object and medical data correpsonds to coherent labels more often than not - varies from application to application but has now become a rule of thumb. Because data in computer vision tends to be highly correlated in space.

%----------------------------------------------------------------------------------------

\section{Labelling Problems as Energy Minimisation}

General case of expressing data-driven and regularisation criteria as concrete energy decisions.

smoothness, neighbourhood, joint probability

Markov Random Field, MAP-MRF problem

%----------------------------------------------------------------------------------------

\section{Energy Minimisation Algorithms and Special Cases}\label{FillingFile}

Dynamic Programming 

Binary Energies with Coherence

%----------------------------------------------------------------------------------------

\section{Thesis Overview}

The remainder of the thesis outline.\\
\\
\textbf{\hyperref[chap:Chapter2]{Chapter~\ref*{chap:Chapter2}}} is where we cover the mathematical foundation to Graph Cut image segmenation.\\
\\
\textbf{\hyperref[chap:Chapter3]{Chapter~\ref*{chap:Chapter3}}} is where we cover the mathematical foundation to Graph Cut image segmenation.\\
\\
\textbf{\hyperref[chap:Chapter4]{Chapter~\ref*{chap:Chapter4}}} is where we cover the mathematical foundation to Graph Cut image segmenation.\\
\\
\textbf{\hyperref[chap:Chapter5]{Chapter~\ref*{chap:Chapter5}}} is where we cover the mathematical foundation to Graph Cut image segmenation.\\
\\
\textbf{\hyperref[chap:Chapter6]{Chapter~\ref*{chap:Chapter6}}} is where we cover the mathematical foundation to Graph Cut image segmenation.\\
\\
\textbf{\hyperref[chap:Chapter7]{Chapter~\ref*{chap:Chapter7}}} is where we cover the mathematical foundation to Graph Cut image segmenation.\\
\\
\textbf{\hyperref[chap:Chapter8]{Chapter~\ref*{chap:Chapter8}}} concludes the thesis with suggestions for further work.
