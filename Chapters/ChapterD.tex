% Chapter Template

\chapter{Fluorescence Microscopy} % Main chapter title

\label{chap:Chapter2} % Change 2 to a consecutive number; for referencing this chapter elsewhere, use \ref{chap:Chapter2}

\citep{SpringDavisdson2016,Rice2016}
\citep{Rice2016}
\citep{Nobel2016}
\citep{AbramowitzDavidson2016}
\citep{ThermoFisher2016}
\citep{LichtmanConchello2005}
\citep{Spring2003}
\citep{Biehlmaier2013}
\citep{Svoboda2007}
\citep{Svoboda2009}
\citep{WuMerchantCastleman2008}
\citep{GonzalezWoods2002}
\citep{Pratt2001}
\citep{Soile2004}
\citep{Murphy2001}
\citep{Matula2006}
\citep{Rohr2010}
\citep{Matula2000}
\citep{Sarder2006}
\citep{Vu2008}
\citep{Kolmogorov2004}
\citep{Hubeny2008}
\citep{Kozubek2001}
\citep{Petran1985}
\citep{Tsien1998}

What is Fluorescence Microscopy?\\
What's it's purpose in the thesis?\\
Photoluminescence -> Fluorescence and Phosphorence\\
Discovery of fluorescence: Brief History and evolution\\
Brief discussion on the remainder of the chapter.

%----------------------------------------------------------------------------------------
%	SECTION 1
%----------------------------------------------------------------------------------------

\section{Physics of Fluorescence}
\label{sec:PhysicsOfFluorescence}

Excitation and Emission, Fluorophores, Jablonski Diagram, Electronic States, Stoke's Shift

%----------------------------------------------------------------------------------------
%	SECTION 2
%----------------------------------------------------------------------------------------

\section{Specimen Preparation}
\label{sec:SpecimenPreparation}

Biological Fluorescence Stains, Immunofluorescence, Fluorescent Proteins\\
FISH, IHC

%----------------------------------------------------------------------------------------
%	SECTION 3
%----------------------------------------------------------------------------------------

\section{The Epifluorescence Microscope and Image Acquisition}
\label{sec:TheEpifluorescenceMicroscope}

Schematic layout of a Fluorescence Microscope. Process, function of each component\\
Other Types of Fluorescence Microscopes: Confocal, TIRF, Epifluorescence\\
Acquisition: CCD\\
Hardware setup effect on image quality.\\
Numerical Aperture, Sub-diffraction\\

%----------------------------------------------------------------------------------------
%	SECTION 4
%----------------------------------------------------------------------------------------

\section{Image Processing in FM}
\label{sec:ImageProcessingInFM}

Factors that reduce the quality of Fluorescence Images\\
Preprocessing: Point Spread Function deconvolution, etc\\
Segmentation 

%----------------------------------------------------------------------------------------
%	SECTION 5
%----------------------------------------------------------------------------------------

\section{Measurements and Analysis in FM}
\label{sec:Measurements}

what is measured and for what?\\
Motion, number of cells, area, volume, lenght