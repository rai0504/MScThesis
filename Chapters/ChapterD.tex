% Chapter Template

\chapter{Fluorescence Microscopy} % Main chapter title

\label{chap:Chapter2} % Change 2 to a consecutive number; for referencing this chapter elsewhere, use \ref{chap:Chapter2}

%\citep{SpringDavisdson2016}
%\citep{Rice2016}
\citep{Nobel2016}
%\citep{AbramowitzDavidson2016}
%\citep{ThermoFisher2016}
%\citep{LichtmanConchello2005}
%\citep{Spring2003}
\citep{Biehlmaier2013}
\citep{Svoboda2007}
\citep{Svoboda2009}
\citep{WuMerchantCastleman2008}
\citep{GonzalezWoods2002}
\citep{Pratt2001}
\citep{Soile2004}
%\citep{Murphy2001}
\citep{Matula2006}
\citep{Rohr2010}
\citep{Matula2000}
%\citep{Sarder2006}
\citep{Vu2008}
\citep{Kolmogorov2004}
%\citep{Hubeny2008}
\citep{Kozubek2001}
\citep{Petran1985}
%\citep{Tsien1998}

\textcolor{red}{What is fluorescence microscopy? What's it's purpose in the thesis?	Photoluminescence -> fluorescence and phosphorescence. Discovery of fluorescence: Brief history and evolution. Brief discussion on the remainder of the chapter.}
Fluorescence microscopy has become an essential tool in diverse fields, such as petrology, semiconductors, etc, and has especially been established as a choice imaging technique in cellular and molecular biology for visualisation of cells and tissues \citep{Spring2003,Danek2012,Hubeny2008,Fatima2008,Matula2012}.
In this thesis we confine our attention to its use in cellular biology.

Certain substances emit radiation when irradiated with a higher intensity light, such as ultraviolet (UV), blue or green, which is off a longer wavelength than that of the exciting light, this is known as \textit{Stokes' Law}.
This phenomenon is known as \textit{photoluminescence} \citep{Koch1972,Vaughan2015,Sarder2006,AbramowitzDavidson2016}.
There are two types of photoluminescence. If emission persists at an appreciable level after the exciting light is turned off, then we call this \textit{phosphorescence}.
If emission persist only so long as the exciting light is on, then we call this \textit{fluorescence} \citep{Koch1972,SpringDavisdson2016}.

The first observance and publishing of fluorescence is credited to Sir John Frederick William Herschel around 1852.
In 1852, Sir John George Stokes published a 100 page treatise about this luminescent phenomenon and coined the term \textit{fluorescence}, over Herschel's \textit{dispersive reflection}, when he observed that the mineral \textit{fluorite} emitted red light when irradiated by ultraviolet (UV) light \citep{Dobrucki2013,Danek2012}.

In the remainder of this chapter we present the underlying principles of fluorescence, how specimens are fluorescently marked, the optical principles of microscope design, image acquisition, image processing and common analysis in cellular biology. We only go so far in depth as to present a rudimentary understanding of fluorescence microscopy as is necessary for the comprehension of this thesis.

%----------------------------------------------------------------------------------------
%	SECTION 1
%----------------------------------------------------------------------------------------

\section{Physics of Fluorescence}
\label{sec:PhysicsOfFluorescence}

Excitation and Emission, Fluorophores, Jablonski Diagram, Electronic States, Stoke's Shift
\begin{figure}[!t]
	\centering
	\includegraphics[width=\columnwidth]{Fluor532_ExcitationEmissionSpectrum.png}
	\caption{Normalised Excitation and Emmision Spectra of the Alexa Fluor 532 flurophore. The emmission maximum is at $553nm$ which is a more yellow-green than excitation maximum at $528nm$. This image was generated using FluoScout\texttrademark\, web application by Leica Microsystems for determining the optimal fluorescence filter cube set. %\url{http://www.leica-microsystems.com/fluoscout/}
	}
	\addloflink{http://www.leica-microsystems.com/fluoscout/}
	\label{fig:excitationandemissionspectra}
\end{figure}

\begin{figure}[!t]
	\centering
	\subfigure[]
	{
		\includegraphics[width=0.4\columnwidth]{molecular_absorption.png}
		\label{fig:molecularabsorption}
	}
	\subfigure[]
	{
		\includegraphics[width=0.56\columnwidth]{jablonski_diagram.png}
		\label{fig:jablonski}
	}
	\caption{\textbf{(a)} Simplified fluorescence process. \textbf{(b)} The Jab{\l}o{\'n}ski diagram depicting the electronics states from photon absorption to photo emission.}
	\label{fig:thinkoflabel}
\end{figure}

%----------------------------------------------------------------------------------------
%	SECTION 2
%----------------------------------------------------------------------------------------

\section{Specimen Labelling}
\label{sec:SpecimenLabelling}

\textcolor{red}{Why do specimens have to be stained? What is is staining?}
Many of the components of interest, such as cell nuclei, cytoplasm, genes, chrosomes, proteins, do not possess a high degree of, if not any,  autofluorescence. 
In this scenario, these components can be marked with a fluorescent dye \citep{Tsien1998}, also known as a fluorophore or fluorochrome, a substance that can bind to a specific target whose excitation and emission spectra are well known. 
This is known as staining \citep{Danek2012,Hubeny2008,Dobrucki2013}. 
Once the specimen is stained it can be indirectly observed using a fluorescence microscope.

\textcolor{red}{What are the most common staining protocols?}
The most prevalent staining techniques are fluorescence in-situ hybridisation (FISH) and immunostaining \citep{Danek2012,Fatima2008,Kozubek2001_2,Theodosiou2007}.

\begin{definition}[FISH staining]
	\textcolor{red}{What is FISH?	What is the FISH staining techniques used for?}
	FISH is a molecular cytogenetic technique that uses flourophores that bind to selected regions in nucleic acids \citep{Danek2012,Fatima2008}.
	FISH is the most frequently used staining technique used primarily for visualisation and localisation of nucleic acid sequences, chromosomes, cytplasm or organelles which contain those acids \citep{Hubeny2008}.
	This makes FISH highly attractive for finding specific features in DNA and RNA used in genetic diagnosis and research, medicine and species identification \citep{Amann2008,Fatima2008}.
	Figure \ref{fig:FISH} is a capture of mouse chromosomes using the FISH staining technique.
\end{definition}

\begin{definition}[Immunostaining]
	\textcolor{red}{What is Immunofluorescence and the two main types, what is the difference between the two, and which is more common? What is the Immunofluorescence staining techniques used for?}
	Immunofluorescence is the detection method where an antibody is used to detect an antigen in a tissue or a cell using fluorescence. Flourophores are usually conjugated onto antibodies, which are proteins that are designed bind to specific antigens, target proteins, on a cell \citep{CudeBurke2014}.
	The two types of immunofluorescent detection are immunocytofluorescence (ICF) and immunohystofluorescence (IHF).
	It must not be confused with immunocytochemistry (ICC) and immunohistochemistry (IHC).
	\textit{Immuno} refers to the immunological technique, i.e. the binding of antibodies to antigens.
	\textit{Cyto} refers to cells, i.e. cells without the extracellular membrane.
	\textit{Histo} refers to tissue i.e. cells with the extracellular membrane.
	\textit{Chemistry} refers to the chemical method of detection, e.g. a change in colour.
	\textit{Fluorescence} detection by emission of light \citep{Katikireddy2011}.
	Figure \ref{fig:IHC} shows the detection of the p53 Binding Protein 1 in perfusion fixed frozen sections of rat kidney.
\end{definition}

\begin{definition}[Live-cell staining]
	\textcolor{red}{FISH and IHC cannot stain live cells. Why? How can we stain live cells?}
	The previously discussed staining techniques are not suitable to observe living cells.
	The fluorescent dyes used are phototoxic and cause cells to die. The circumvent this problem an elegant solution has been devised.
	Instead of staining, the cells are modified to produce a fluorescent substance in the target structures.
	Derivatives of the \textit{green fluorescent protein} (GFP), isolated from the \textit{Aequorea victoria} jellyfish \citep{Tsien1998,LichtmanConchello2005,Fatima2008}, are used as it generates a strong photon emission and is non-toxic to living cells \citep{Danek2012,Hubeny2008,Dobrucki2013}.
\end{definition}

\textcolor{red}{Important notes about fluorophores and the impact on image quality?}

\begin{figure}[!t]
	\centering
	\subfigure[]
	{
		\includegraphics[width=0.495\columnwidth]{fish1.jpg}
		\label{fig:FISH}
	}
	\subfigure[]
	{
		\includegraphics[width=0.45\columnwidth]{ihc1.jpg}
		\label{fig:IHC}
	}
	\caption{\textbf{(a)} FISH (Fluorescent 'in-situ' Hybridization) in mouse chromosomes using a BAC clone labeled with Spectrum Orange. The picture shows two metaphases and one interphase with two signals in each exampling a homozygous mouse for a transgenic clone. Image Source: "All About the Human Genome Project" Genetic and Genomic Image and Illustration Database. %\addloflink{https://unlockinglifescode.org/media/images/} %\url{https://unlockinglifescode.org/media/images/}. 
	\textbf{(b)} p53 Binding Protein 1 (53BP1) was detected in perfusion fixed frozen sections of rat kidney using Goat Anti-Human 53BP1 Antigen Affinity-purified Polyclonal Antibody (Catalog \# AF1877) at 15 $\mu$g/mL overnight at 4$^{\circ}$C. Tissue was stained using the NorthernLights\texttrademark 557-conjugated Anti-Goat IgG Secondary Antibody (red; Catalog \# NL001) and counterstained with DAPI (blue). Specific staining was localized to nuclei of epithelial cells in convoluted tubules. Image Source: R\&D Systems' IHC image database.}
	%\addloflink{https://www.rndsystems.com/resources/ihc-images/53bp1}}
	%\url{https://www.rndsystems.com/resources/ihc-images/53bp1}.}
	\addloflink{https://unlockinglifescode.org/media/images/}
	\addloflink{https://www.rndsystems.com/resources/ihc-images/53bp1}
	\label{fig:stainingtechniques}
\end{figure}

%----------------------------------------------------------------------------------------
%	SECTION 3
%----------------------------------------------------------------------------------------

\section{The Epifluorescence Microscope and Image Acquisition}
\label{sec:TheEpifluorescenceMicroscope}

\begin{figure}[!t]
	\centering
	\includegraphics[width=0.4\columnwidth]{Epifluorescence_Microscope.png}
	\caption{The schematic of the epifluorescence microscope.}
	\label{fig:epifluorescencemicroscope}
\end{figure}

\textcolor{red}{What is a fluorescent microscope? Schematic layout of a fluorescence microscope? Function and purpose of each component in the fluorescent microscope?}
A fluorescence microscope is an optical microscope that is designed specifically to exploit the principle of fluorescence to allow for the observation of flourescently labelled specimens \citep{Hubeny2008,Sarder2006,Dobrucki2013,Andrews2002,Fatima2008}.
There are many types of fluorescent micrcoscopes avaialble but the favoured type among many biologists and geneticists is the epifluorescent microscope \citep{Rice2016,AbramowitzDavidson2016}.
The schematic of the epifluorescent micrcoscope is illustrated in \autoref{fig:epifluorescencemicroscope}.

\begin{definition}[Light Source]
	\textcolor{red}{What sort of light needs to be generated? What sort of lamps are used? Advantages and disadvantages of certain lamps.}
	The light source is typically a high-luminance light source e.g. Mercury or Xenon arc lamps, LEDs, lasers, etc  \citep{Danek2012,Hubeny2008,Aswani2012,Rice2016,ThermoFisher2016}.
	The primary criterion for choosing a light sources is that its characeristic peaks must coincide with the excitation spectrum of the fluorophores being used \citep{LichtmanConchello2005,Spring2003,Fatima2008}.
	Wavelength coverage spans from near infra-red to UV. Mercury and Xenon arc lamps are expensive, an inexpensive and lightweight alternative is bright LEDs \citep{Fatima2008,Dobrucki2013,Aswani2012,Koch1972}.
\end{definition}

\begin{definition}[Excitation Filter]
	\textcolor{red}{What is an excitation filter? Why is it needed?}
	The incoming light from the light source is typically mulispectral \citep{SpringDavisdson2016}. 
	The excitation filter is a wavelength selection filter which is placed in the path of the incoming light and filters through only those wavelengths in the absorption spectrum of the fluorescent dye \citep{ThermoFisher2016,Danek2012,Hubeny2008,LichtmanConchello2005,Spring2003,CudeBurke2014,Fatima2008,Dobrucki2013}.
\end{definition}

\begin{definition}[Dichroic Mirror]
	\textcolor{red}{What is a dichroic mirror? Why is it needed?}
	Also known as a \textit{dichroic beam splitter}.
	This is placed at a 45$^{\circ}$ angle and reflects the short-wavelenght light filtered through the excitation filter at a 90$^{\circ}$ angle towards the specimen \citep{Danek2012,Hubeny2008,Spring2003,CudeBurke2014} and allows the long-wavelength light from the fluorescing specimen to pass through to the detector \citep{LichtmanConchello2005,Koch1972}, thus serving as a separation filter between the absorption and emission light \citep{Fatima2008,Dobrucki2013}.
\end{definition}

\begin{definition}[Objective]
	\textcolor{red}{What is the objective? Why is it needed?}
	The incoming light reflected of the dichroic mirror passes through the objective lens before reaching the specimen \citep{Danek2012,Hubeny2008,LichtmanConchello2005,Spring2003}.
	Emission light from the fluorescing specimen is gathered in the objective lens and passed through to the dichroic mirror.
\end{definition}

\begin{definition}[Specimen]
	\textcolor{red}{Say something about the specimen, for wholeness sake.}
	The specimen is irradiated by the incoming light from the objective and emits long-wavelength light in all directions.
	The specimen is stained with a flourophore whose absorption and emission curves are well known.
	This is important since the light source and the interference filters are chosen using the peaks of these curves.
\end{definition}

\begin{definition}[Emission Filter]
	\textcolor{red}{What is an emission filter? Why is it needed?}
	Also known as a \textit{barrier filter} \citep{LichtmanConchello2005,Spring2003,Koch1972}.
	The light coming from the specimen contains multiple wavelengths and the dichroic mirror is used to filter out the shorter wavelength light.
	The emission filter is further  used to filter out the wavelengths that correspond to the emission wavelengths of the fluorophore \citep{CudeBurke2014,Danek2012,Hubeny2008,SpringDavisdson2016,ThermoFisher2016}.
\end{definition}

%\begin{definition}[Occular]
%	\textcolor{red}{What is the occular why is it needed?}
%	Purpose of the occular.
%\end{definition}

\begin{definition}[Detector]
	\textcolor{red}{What is the detector why is it needed?}
	The dector is used to capture the emission light and can further digital form the image.
	The detector is usually a CCD (charge-coupled device) camera or a photomultipler tube \citep{Danek2012,Hubeny2008,LichtmanConchello2005,Spring2003,Murphy2001}.
	It is vital that an appropriate detector be chosen as this has direct influence of image quality \citep{Fatima2008}.
\end{definition}

\textcolor{red}{Other Types of Fluorescence Microscopes: Confocal, TIRF, Epifluorescence, Acquisition: CCD, Hardware setup effect on image quality, Numerical Aperture, Sub-diffraction}


%----------------------------------------------------------------------------------------
%	SECTION 4
%----------------------------------------------------------------------------------------

\section{Image Processing in FM}
\label{sec:ImageProcessingInFM}

Limitations in Fluorescence Imaging\\
Preprocessing: Point Spread Function deconvolution, etc\\
Segmentation 

%----------------------------------------------------------------------------------------
%	SECTION 5
%----------------------------------------------------------------------------------------

\section{Measurements and Analysis in FM}
\label{sec:Measurements}

what is measured and for what?\\
Motion, number of cells, area, volume, lenght