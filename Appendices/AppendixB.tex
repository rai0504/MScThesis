% Appendix Template

\chapter{Introduction to Graph Theory} % Main appendix title

\label{AppendixB} % Change X to a consecutive letter; for referencing this appendix elsewhere, use \ref{AppendixX}

\begin{definition}[Graph]\label{def_graph}
	A graph $G$ is a pair $(V,E)$, where $V$ is the set of nodes/vertices and $E$ is the set of edges consisting of pairs $(u,v)$ where $u,v \in V$. The graph is assumed to be finite i.e. $|V| = n$ and $|E| = m$.
	
	In an \textbf{undirected graph}, the edge $(u,v)$ and $(v,u)$ are not distinct. That is, they refer to the same edge. However, in a \textbf{directed graph}, the two edge are now distinct. In a directed graph with edge $(u,v)$, $u$ is known as the \textbf{tail} and $v$ is known as the \textbf{head}. In directed graphs, edges, also known as arcs, are depicted by placing arrowheads at the head of the edge. Given an edge $e = (u,v)$, $u$ and $v$ are said to be \textbf{incident} on $e$. A graph is said to be \textbf{simple} if it does not contain any self-loops. A \textbf{self-loop} is an edge with each of its end points being the same vertex.
\end{definition}

\tikzstyle{vertex}=[circle,thick,draw]
\tikzstyle{edge} = [draw=black!24, very thick,-]
\tikzstyle{weight} = [font=\small]
\begin{figure}[!h]
	\centering
	\resizebox {\columnwidth} {!} {
%		\begin{tikzpicture}[scale=1.5, auto, swap, background rectangle/.style={fill=blue!10}, show background rectangle]
		\begin{tikzpicture}[scale=1.5, auto, swap, background rectangle/.style={fill=blue!10}]
		\draw[black] node at (-0.2,2.8) [font=\large,right,rounded corners,inner sep=1ex] {$\textbf{G}$};
		% draw the vertices
		\foreach \pos/\name in {{(0,2)/a}, {(2,2)/b}, {(4,1)/c},{(0,0)/d}, {(3,0)/e}, 
			{(4,-1)/g}, {(1,-1)/f}}
		\node[vertex] (\name) at \pos {$\name$};
		% connect vertices with edges and draw weights
		\foreach \source/ \dest /\weight in {b/a/10,c/b/3,d/a/5,d/b/7, e/b/9, e/c/7,e/d/5,
			f/d/3,f/e/1, g/e/3,g/f/5}
		\path[edge] (\source) -- node[weight] {$\weight$} (\dest);
		% info box
		\draw[yshift=0cm,xshift=4.5cm]
		node [right,text width=5cm,rounded corners,fill=red!20,inner sep=1ex]
		{
			$V_{G} = \{a,b,c,d,e,f,g\}$, \\
			$|V_{G}| = 7$\\
			$E_{G} = \{ab,ad,bc,bd,be,ce,de,df,ef,$\\$eg,fg\}$, \\ 
			$|E_{G}| = 11$
		};	
		% degrees
		\draw[orange] node at (-0.2,2.4) [font=\footnotesize,right,rounded corners,inner sep=1ex] {$d(a)=2$};
		\draw[orange] node at (1.8,2.4) [font=\footnotesize,right,rounded corners,inner sep=1ex] {$d(b)=3$};
		\draw[orange] node at (3.8,1.4) [font=\footnotesize,right,rounded corners,inner sep=1ex] {$d(c)=2$};
		\draw[orange] node at (-0.3,-0.5) [font=\footnotesize,right,rounded corners,inner sep=1ex] {$d(d)=4$};
		\draw[orange] node at (3.2,0.0) [font=\footnotesize,right,rounded corners,inner sep=1ex] {$d(e)=5$};
		\draw[orange] node at (0.8,-1.5) [font=\footnotesize,right,rounded corners,inner sep=1ex] {$d(f)=3$};
		\draw[orange] node at (3.8,-1.5) [font=\footnotesize,right,rounded corners,inner sep=1ex] {$d(g)=2$};
		\end{tikzpicture}
	}
	\caption{Undirected weighted graph \textbf{G}. The degree of each node is shown next to the corresponding node. The graph is simple. The red box shows the vertex set, $V_{G}$, and edge set, $E_{G}$, and their corresponding norm.}
	\label{fig:appG}
\end{figure}

\begin{definition}[Degree]
	The degree of a vertex $v$ is the number of edges incident on it. $deg(v) = |\{(u,v), (v,u) \in E\}|$. A self-loop counts for 2.
	
	If a graph is directed, it is also known as a \textbf{digraph}, then a node $v$ has an \textbf{in-degree} $d_{in}(v)$ and an \textbf{out-degree} $d_{out}(v)$. A digraph is said to be \textbf{balanced} if $d_{in}(v) = d_{out}(v), \forall v \in V$.
\end{definition}

\tikzstyle{vertex}=[circle,thick,draw]
\tikzstyle{edge} = [draw=black!24, very thick,<-]
\tikzstyle{weight} = [font=\small]
\begin{figure}[!h]
	\centering
	\resizebox {\columnwidth} {!} {
%		\begin{tikzpicture}[scale=1.5, auto, swap, background rectangle/.style={fill=blue!10}, show background rectangle, >={Stealth[black!24]}]
		\begin{tikzpicture}[scale=1.5, auto, swap, background rectangle/.style={fill=blue!10}, >={Stealth[black!24]}]
		\draw[black] node at (-0.2,3.1) [font=\large,right,rounded corners,inner sep=1ex] {$\textbf{D}$};
		% draw the vertices
		\foreach \pos/\name in {{(0,2)/a}, {(2,2)/b}, {(4,1)/c},{(0,0)/d}, {(3,0)/e}, 
			{(4,-1)/g}, {(1,-1)/f}}
		\node[vertex] (\name) at \pos {$\name$};
		% connect vertices with edges and draw weights
		\foreach \source/ \dest /\weight in {b/a/10,c/b/3,d/a/5,e/b/9,e/d/5,
			f/d/3,f/e/1, g/e/3,g/f/5}
		\path[edge] (\source) edge node[weight] {$\weight$} (\dest);
		% bend edges
		\path [edge] (b) edge[bend right=20] node {$7$} (d);
		\path [edge] (d) edge[bend left=-20] node {$4$} (b);
		\path [edge] (e) edge[bend right=20] node {$7$} (c); 
		\path [edge] (c) edge[bend right=20] node {$2$} (e);
		% info box
		\draw[yshift=0cm,xshift=4.5cm]
		node [right,text width=5cm,rounded corners,fill=red!20,inner sep=1ex]
		{
			$V_{D} = \{a,b,c,d,e,f,g\}$, \\
			$|V_{D}| = 7$\\
			$E_{D} = \{ab,ad,bc,bd,be,ce,db,de,df,$\\$ec,ef,eg,fg\}$, \\
			$|E_{D}| = 13$
		};
		% degrees
		% a	
		\draw[orange] node at (-0.2,2.7) [font=\footnotesize,right,rounded corners,inner sep=1ex] {$d_{in}=0$};
		\draw[orange] node at (-0.2,2.4) [font=\footnotesize,right,rounded corners,inner sep=1ex] {$d_{out}=2$};	
		% b
		\draw[orange] node at (1.8,2.7) [font=\footnotesize,right,rounded corners,inner sep=1ex] {$d_{in}=2$};
		\draw[orange] node at (1.8,2.4) [font=\footnotesize,right,rounded corners,inner sep=1ex] {$d_{out}=5$};
		% c
		\draw[orange] node at (3.8,1.7) [font=\footnotesize,right,rounded corners,inner sep=1ex] {$d_{in}=2$};
		\draw[orange] node at (3.8,1.4) [font=\footnotesize,right,rounded corners,inner sep=1ex] {$d_{out}=1$};
		% d
		\draw[orange] node at (-0.3,-0.5) [font=\footnotesize,right,rounded corners,inner sep=1ex] {$d_{in}=2$};
		\draw[orange] node at (-0.3,-0.8) [font=\footnotesize,right,rounded corners,inner sep=1ex] {$d_{out}=2$};
		% e
		\draw[orange] node at (3.2,0.08) [font=\footnotesize,right,rounded corners,inner sep=1ex] {$d_{in}=3$};
		\draw[orange] node at (3.2,-0.12) [font=\footnotesize,right,rounded corners,inner sep=1ex] {$d_{out}=3$};
		% f
		\draw[orange] node at (0.8,-1.5) [font=\footnotesize,right,rounded corners,inner sep=1ex] {$d_{in}=2$};
		\draw[orange] node at (0.8,-1.8) [font=\footnotesize,right,rounded corners,inner sep=1ex] {$d_{out}=1$};
		%g
		\draw[orange] node at (3.8,-1.5) [font=\footnotesize,right,rounded corners,inner sep=1ex] {$d_{in}=2$};
		\draw[orange] node at (3.8,-1.8) [font=\footnotesize,right,rounded corners,inner sep=1ex] {$d_{out}=0$};
		\end{tikzpicture}
	}
	\caption{Directed weighted graph (Digraph) \textbf{D}. The in-degree and out-degree is shown next to each node. The graph is simple and not balanced. The red box shows the vertex set, $V_{D}$, and edge set, $E_{D}$, and their corresponding norm.}
	\label{fig:appD}
\end{figure}

\begin{definition}[Subgraph]
	A graph $G' = (V', E')$ is said to be a sub-graph of $G = (V, E)$, denoted as $G' \subseteq G$, if $V' \subseteq V$ and $E' \subseteq E$.
\end{definition}

\tikzstyle{vertex}=[circle,thick,draw]
\tikzstyle{edge} = [draw=black!24, very thick,-]
\tikzstyle{weight} = [font=\small]
\begin{figure}[!h]
	\centering
	\resizebox {\columnwidth} {!} {
		%\begin{tikzpicture}[scale=1.5, auto, swap, background rectangle/.style={fill=blue!10}, show background rectangle]
		\begin{tikzpicture}[scale=1.5, auto, swap, background rectangle/.style={fill=blue!10}]
		\draw[black] node at (-0.2,2.8) [font=\large,right,rounded corners,inner sep=1ex] {$\textbf{H}$};
		% draw the vertices
		\foreach \pos/\name in {{(0,2)/a}, {(2,2)/b}, {(0,0)/d}, {(3,0)/e}}
		\node[vertex] (\name) at \pos {$\name$};
		% connect vertices with edges and draw weights
		\foreach \source/ \dest /\weight in {b/a/10,d/a/5,d/b/7,e/b/9,e/d/5}
		\path[edge] (\source) -- node[weight] {$\weight$} (\dest);
		% info box
		\draw[yshift=-1.6cm,xshift=0cm]
		node [right,text width=5cm,rounded corners,fill=red!20,inner sep=1ex]
		{
			$V_{H} = \{a,b,d,e\} \subseteq V_{G}$, \\
			$|V_{H}| = 4$\\
			$E_{H} = \{ab,ad,bd,be,de\} \subseteq E_{G}$, \\ 
			$|E_{H}| = 5$
		};	
		% degrees
		\draw[orange] node at (-0.2,2.4) [font=\footnotesize,right,rounded corners,inner sep=1ex] {$d(a)=2$};
		\draw[orange] node at (1.8,2.4) [font=\footnotesize,right,rounded corners,inner sep=1ex] {$d(b)=3$};
		\draw[orange] node at (-0.2,-0.4) [font=\footnotesize,right,rounded corners,inner sep=1ex] {$d(d)=4$};
		\draw[orange] node at (2.5,-0.4) [font=\footnotesize,right,rounded corners,inner sep=1ex] {$d(e)=5$};
		
		
		\draw[black] node at (4.2,2.8) [font=\large,right,rounded corners,inner sep=1ex] {$\textbf{I}$};
		% draw the vertices
		\foreach \pos/\name in {{(4.4,2)/a}, {(6.4,2)/b}, {(4.4,0)/d}, {(7.4,0)/e}}
		\node[vertex] (\name) at \pos {$\name$};
		% connect vertices with edges and draw weights
		\foreach \source/ \dest /\weight in {b/a/10,d/a/5,e/b/9,e/d/5}
		\path[draw=black!24, very thick,<-] (\source) -- node[weight] {$\weight$} (\dest);
		% bends
		\path [draw=black!24, very thick,<-] (d) edge[bend left=-20] node {$4$} (b);
		% info box
		\draw[yshift=-1.6cm,xshift=4.4cm]
		node [right,text width=5cm,rounded corners,fill=red!20,inner sep=1ex]
		{
			$V_{I} = \{a,b,d,e\} \subseteq V_{D}$, \\
			$|V_{I}| = 4$\\
			$E_{I} = \{ab,ad,bd,be,de\} \subseteq E_{D}$, \\ 
			$|E_{I}| = 5$
		};	
		% degrees
		% a	
		\draw[orange] node at (4.2,2.5) [font=\footnotesize,right,rounded corners,inner sep=1ex] {$d_{in}=0$};
		\draw[orange] node at (4.2,2.3) [font=\footnotesize,right,rounded corners,inner sep=1ex] {$d_{out}=2$};	
		% b
		\draw[orange] node at (6.2,2.5) [font=\footnotesize,right,rounded corners,inner sep=1ex] {$d_{in}=1$};
		\draw[orange] node at (6.2,2.3) [font=\footnotesize,right,rounded corners,inner sep=1ex] {$d_{out}=2$};
		% d
		\draw[orange] node at (4.1,-0.4) [font=\footnotesize,right,rounded corners,inner sep=1ex] {$d_{in}=2$};
		\draw[orange] node at (4.1,-0.6) [font=\footnotesize,right,rounded corners,inner sep=1ex] {$d_{out}=1$};
		% e
		\draw[orange] node at (7.1,-0.4) [font=\footnotesize,right,rounded corners,inner sep=1ex] {$d_{in}=2$};
		\draw[orange] node at (7.1,-0.6) [font=\footnotesize,right,rounded corners,inner sep=1ex] {$d_{out}=0$};
		\end{tikzpicture}
	}
	\caption{Undirected weighted graph \textbf{H} is a subgraph of \textbf{G} in \Cref{fig:appG}, $\textbf{H} \subseteq \textbf{G}$. Directed weighted graph \textbf{I} is a subgraph of \textbf{D} in \Cref{fig:appD}, $\textbf{I} \subseteq \textbf{D}$. The degree of each node is shown next to the corresponding node. The red box shows the vertex set, the edge set and their corresponding norms.}
\end{figure}


\begin{definition}[Clique]
	A clique is a maximal subgraph.
\end{definition}

\tikzstyle{vertex}=[circle,thick,draw]
\tikzstyle{edge} = [draw=black!24, very thick,-]
\tikzstyle{weight} = [font=\small]
\begin{figure}[!h]
	\centering
	\resizebox {\columnwidth} {!} {
%		\begin{tikzpicture}[scale=1.5, auto, swap, background rectangle/.style={fill=blue!10}, show background rectangle]
		\begin{tikzpicture}[scale=1.5, auto, swap, background rectangle/.style={fill=blue!10}]
		\draw[black] node at (-0.2,2.8) [font=\large,right,rounded corners,inner sep=1ex] {$\textbf{G}$};	
		% draw the vertices
		\foreach \pos/\name in {{(0,2)/a}, {(2,2)/b}, {(4,1)/c},{(0,0)/d}, {(3,0)/e}, 
			{(4,-1)/g}, {(1,-1)/f}}
		\node[vertex] (\name) at \pos {$\name$};
		% connect vertices with edges and draw weights
		\foreach \source/ \dest /\weight in {b/a/{},c/b/{},d/a/{},d/b/{},e/b/{},e/c/{},e/d/{},
			f/d/{},f/e/{}, g/e/{},g/f/{}}
		\path[edge] (\source) -- node[weight] {$\weight$} (\dest);
		
		% hyperedges around cliques
		% abd
		\draw[red, thick, rounded corners=1pt] (a.north) -- (a.north west) -- (a.west) -- (d.west) -- (d.south west) -- (d.south) -- (d.south east) -- (b.south east) -- (b.east) -- (b.north east) -- (b.north) -- (a.north);
		% bde
		\draw[green, thick, rounded corners=1pt] (b.north) -- (b.north west) -- (d.north west) -- (d.west) -- (d.south west) -- (d.south) -- (e.south) -- (e.south east) -- (e.east) -- (e.north east) -- (b.north east) -- (b.north);
		% def
		\draw[blue, thick, rounded corners=1pt] (d.north) -- (d.north west) -- (d.west) -- (d.south west) -- (f.south west) -- (f.south) -- (f.south east) -- (e.south east) -- (e.east) -- (e.north east) -- (e.north) -- (d.north);
		% efg
		\draw[red, thick, rounded corners=1pt] (e.north) -- (e.north west) -- (f.north west) -- (f.west) -- (f.south west) -- (f.south) -- (g.south) -- (g.south east) -- (g.east) -- (g.north east) -- (e.north east) -- (e.north);
		% bce
		\draw[orange, thick, rounded corners=1pt] (b.north) -- (b.north west) -- (b.west) -- (b.south west) -- (e.south west) -- (e.south) -- (e.south east) -- (c.south east) -- (c.east) -- (c.north east) -- (b.north east) -- (b.north);
		
		% info box
		\draw[yshift=2.5cm,xshift=3.0cm]
		node [right,text width=7cm,rounded corners,fill=red!20,inner sep=1ex]
		{
			$V_{G} = \{a,b,c,d,e,f,g\}$, \\
			$E_{G} = \{ab,ad,bc,bd,be,ce,de,df,ef,$\\$eg,fg\}$, \\ 
			$C_{G} = V_G \cup E_G \cup \{abd, bce, efg, edf, bde\}$
		};	
		\end{tikzpicture}
	}
	\caption{Cliques of the undirected weighted graph \textbf{G}. The maximal cliques are shown by the hyperedges that encompass the nodes of that clique.}
\end{figure}