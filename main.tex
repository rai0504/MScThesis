%%%%%%%%%%%%%%%%%%%%%%%%%%%%%%%%%%%%%%%%%
% Masters/Doctoral Thesis 
% LaTeX Template
% Version 2.2.2 (20/2/16)
%
% This template has been downloaded from:
% http://www.LaTeXTemplates.com
%
% Version 2.x major modifications by:
% Vel (vel@latextemplates.com)
%
% This template is based on a template by:
% Steve Gunn (http://users.ecs.soton.ac.uk/srg/softwaretools/document/templates/)
% Sunil Patel (http://www.sunilpatel.co.uk/thesis-template/)
%
% Template license:
% CC BY-NC-SA 3.0 (http://creativecommons.org/licenses/by-nc-sa/3.0/)
%
%%%%%%%%%%%%%%%%%%%%%%%%%%%%%%%%%%%%%%%%%

%----------------------------------------------------------------------------------------
%	PACKAGES AND OTHER DOCUMENT CONFIGURATIONS
%----------------------------------------------------------------------------------------

\documentclass[
11pt, % The default document font size, options: 10pt, 11pt, 12pt
%oneside, % Two side (alternating margins) for binding by default, uncomment to switch to one side
english, % ngerman for German
onehalfspacing, % Single line spacing, alternatives: singlespacing, onehalfspacing or doublespacing
%draft, % Uncomment to enable draft mode (no pictures, no links, overfull hboxes indicated)
%nolistspacing, % If the document is onehalfspacing or doublespacing, uncomment this to set spacing in lists to single
liststotoc, % Uncomment to add the list of figures/tables/etc to the table of contents
%toctotoc, % Uncomment to add the main table of contents to the table of contents
%parskip, % Uncomment to add space between paragraphs
%nohyperref, % Uncomment to not load the hyperref package
headsepline, % Uncomment to get a line under the header
]{MastersDoctoralThesis} % The class file specifying the document structure

\usepackage[utf8]{inputenc} % Required for inputting international characters
\usepackage[T1]{fontenc} % Output font encoding for international characters

\usepackage{palatino} % Use the Palatino font by default

%\usepackage{hyperref}

\usepackage[backend=bibtex,citestyle=phys,natbib=true]{biblatex} % Use the bibtex backend with the authoryear citation style (which resembles APA)

\addbibresource{references.bib} % The filename of the bibliography

\usepackage[autostyle=true]{csquotes} % Required to generate language-dependent quotes in the bibliography

\usepackage{amsmath}
\usepackage{amssymb}
\usepackage{mathtools}

\usepackage{algorithm}
\usepackage{algpseudocode}
\usepackage{pifont}

\usepackage{multirow}
\usepackage{hhline}
\usepackage{tabu}

\usepackage[dvipsnames]{xcolor}
\usepackage{color, colortbl}
\definecolor{bad}{rgb}{1,0.5,0.5}
\definecolor{acceptable}{rgb}{1,0.8,0.4}
\definecolor{closest}{rgb}{0.8,0.8,1}



%\usepackage[usenames,x11names,dvipsnames,svgnames]{xcolor}
%\usepackage{graphicx,amsmath,latexsym,amssymb,amsthm,geometry}
\usepackage{tikz}
%\usepackage{pagecolor}% http://ctan.org/pkg/{pagecolor,lipsum}
\usetikzlibrary{arrows.meta,decorations,decorations.text,backgrounds,arrows,shapes,intersections,calc,hobby}
%\usetikzlibrary{verbatim}

\usepackage{cleveref}

\usepackage{subfigure}

%\usepackage{gensymb}

%\usepackage[T3,OT2,T1]{fontenc}
\usepackage[safe]{tipa}
\usepackage{tipx}



%\newtheorem{theorem}{Theorem}[section]
%\newtheorem{lemma}[theorem]{Lemma}
%\newtheorem{proposition}[theorem]{Proposition}
%\newtheorem{corollary}[theorem]{Corollary}
%
%\newenvironment{proof}[1][Proof]{\begin{trivlist}
%		\item[\hskip \labelsep {\bfseries #1}]}{\end{trivlist}}
%\newenvironment{definition}[1][Definition]{\begin{trivlist}
%		\item[\hskip \labelsep {\bfseries #1}]}{\end{trivlist}}
%\newenvironment{example}[1][Example]{\begin{trivlist}
%		\item[\hskip \labelsep {\bfseries #1}]}{\end{trivlist}}
%\newenvironment{remark}[1][Remark]{\begin{trivlist}
%		\item[\hskip \labelsep {\bfseries #1}]}{\end{trivlist}}
%
%\newcommand{\qed}{\nobreak \ifvmode \relax \else
%	\ifdim\lastskip<1.5em \hskip-\lastskip
%	\hskip1.5em plus0em minus0.5em \fi \nobreak
%	\vrule height0.75em width0.5em depth0.25em\fi}\newtheorem{theorem}{Theorem}[section]
%\newtheorem{lemma}[theorem]{Lemma}
%\newtheorem{proposition}[theorem]{Proposition}
%\newtheorem{corollary}[theorem]{Corollary}
%
%\newenvironment{proof}[1][Proof]{\begin{trivlist}
%		\item[\hskip \labelsep {\bfseries #1}]}{\end{trivlist}}
\newenvironment{definition}[1][Definition]{\begin{trivlist}
		\item[\hskip \labelsep {\bfseries #1}]}{\end{trivlist}}
%\newenvironment{example}[1][Example]{\begin{trivlist}
%		\item[\hskip \labelsep {\bfseries #1}]}{\end{trivlist}}
%\newenvironment{remark}[1][Remark]{\begin{trivlist}
%		\item[\hskip \labelsep {\bfseries #1}]}{\end{trivlist}}
%
%\newcommand{\qed}{\nobreak \ifvmode \relax \else
%	\ifdim\lastskip<1.5em \hskip-\lastskip
%	\hskip1.5em plus0em minus0.5em \fi \nobreak
%	\vrule height0.75em width0.5em depth0.25em\fi}


\hyphenation{mo-le-cu-lar ful-le-ro-pyr-ro-li-di-ne ful-le-ro-pyr-ro-li-di-nes his-to-grams his-to-gram pla-te-aus cell-image-library}

\DeclareMathOperator*{\argmin}{arg\,min}
\DeclareMathOperator*{\argmax}{arg\,max} 

% make single figures on a new page be place at the top
\makeatletter
\setlength{\@fptop}{0pt}
\makeatother

%----------------------------------------------------------------------------------------
%	MARGIN SETTINGS
%----------------------------------------------------------------------------------------

\geometry{
	paper=a4paper,%letterpaper, % Change to letterpaper for US letter or a4paper
	inner=1.0cm,%2.5cm, % Inner margin
	outer=1.5cm,%2.5cm,%3.8cm, % Outer margin
	bindingoffset=1.5cm,%2cm, % Binding offset
	top=1.5cm, % Top margin
	bottom=1.5cm, % Bottom margin
	%showframe,% show how the type block is set on the page
}

%----------------------------------------------------------------------------------------
%	THESIS INFORMATION
%----------------------------------------------------------------------------------------

\thesistitle{A Fluorescence Image Segmentation Model Based on Improved Discrete Active Contours} % Your thesis title, this is used in the title and abstract, print it elsewhere with \ttitle
\supervisor{Prof. Jules-Raymond \textsc{Tapamo}} % Your supervisor's name, this is used in the title page, print it elsewhere with \supname
\examiner{} % Your examiner's name, this is not currently used anywhere in the template, print it elsewhere with \examname
\degree{Master of Science in Engineering} % Your degree name, this is used in the title page and abstract, print it elsewhere with \degreename
\author{Ryan \textsc{Naidoo}} % Your name, this is used in the title page and abstract, print it elsewhere with \authorname
\addresses{} % Your address, this is not currently used anywhere in the template, print it elsewhere with \addressname

\subject{Computer Vision} % Your subject area, this is not currently used anywhere in the template, print it elsewhere with \subjectname
\keywords{} % Keywords for your thesis, this is not currently used anywhere in the template, print it elsewhere with \keywordnames
\university{\href{http://www.ukzn.ac.za}{University of KwaZulu-Natal}} % Your university's name and URL, this is used in the title page and abstract, print it elsewhere with \univname
\department{\href{http://caes.ukzn.ac.za/Homepage.aspx}{School of Engineering}} % Your department's name and URL, this is used in the title page and abstract, print it elsewhere with \deptname
\group{\href{http://eece.ukzn.ac.za/Homepage.aspx}{Department of Electrical, Electronic and Computer Engineering}} % Your research group's name and URL, this is used in the title page, print it elsewhere with \groupname
\faculty{\href{http://faculty.university.com}{College of Agriculture, Engineering and Science}} % Your faculty's name and URL, this is used in the title page and abstract, print it elsewhere with \facname

\hypersetup{pdftitle=\ttitle} % Set the PDF's title to your title
\hypersetup{pdfauthor=\authorname} % Set the PDF's author to your name
\hypersetup{pdfkeywords=\keywordnames} % Set the PDF's keywords to your keywords

\makeatletter
\newcommand{\addloflink}[1]{% \addloflink{<URL>}
	\addtocontents{lof}{\begingroup\def\protect\@dotsep{10000}% Remove dots in LoF for this entry
		\protect\contentsline{figlink}{\protect\numberline{}\url{#1}}{}{}%
		\endgroup}% Restore dots in LoF for future entries
}
\newcommand{\l@figlink}{\@dottedtocline{1}{1.5em}{2.3em}}
\makeatother

%\captionsetup{width=\columnwidth}

% Define some commands to keep the formatting separated from the content 
\newcommand{\keyword}[1]{\textbf{#1}}
\newcommand{\tabhead}[1]{\textbf{#1}}
\newcommand{\code}[1]{\texttt{#1}}
\newcommand{\file}[1]{\texttt{\bfseries#1}}
\newcommand{\option}[1]{\texttt{\itshape#1}}

\begin{document}
	
\frontmatter % Use roman page numbering style (i, ii, iii, iv...) for the pre-content pages

\pagestyle{plain} % Default to the plain heading style until the thesis style is called for the body content

%----------------------------------------------------------------------------------------
%	TITLE PAGE
%----------------------------------------------------------------------------------------

\begin{titlepage}
\begin{center}

{\scshape\LARGE \univname\par}\vspace{0.5cm} % University name
{\scshape\large \facname\par}
%\large\facname \vspace{1.0cm} % College

\begin{figure}[!h]
	\centering
	\includegraphics[width=0.3\columnwidth]{Logo.png} % University/department logo - uncomment to place it
\end{figure}
\vspace{0.5cm}
%\textsc{\Large Masters Thesis}\\[0.5cm] % Thesis type

\HRule \\[0.4cm] % Horizontal line
{\huge \bfseries \ttitle\par}\vspace{0.4cm} % Thesis title
\HRule \\[1.5cm] % Horizontal line
 
\begin{minipage}[t]{0.4\textwidth}
\begin{flushleft} \large
\emph{By:}\\
%\href{http://www.johnsmith.com}
\textcolor{magenta}{\authorname} % Author name - remove the \href bracket to remove the link
\end{flushleft}
\end{minipage}
\begin{minipage}[t]{0.4\textwidth}
\begin{flushright} \large
\emph{Supervisor:} \\
%\href{http://www.jamessmith.com}
\textcolor{magenta}{\supname} % Supervisor name - remove the \href bracket to remove the link  
\end{flushright}
\end{minipage}\\[2cm]
 
\large \textit{A thesis submitted in fulfillment of the requirements\\ for the degree of \degreename}\\[0.3cm] % University requirement text
\textit{in the}\\[0.4cm]
\deptname\\[1cm] % Research group name and department name
 
 \LARGE\textbf{EXAMINER'S COPY}\\
{\large \today}%\\[4cm] % Date

\vfill
\end{center}
\end{titlepage}

%----------------------------------------------------------------------------------------
%	ABSTRACT PAGE
%----------------------------------------------------------------------------------------

\begin{abstract}
	\addchaptertocentry{\abstractname} % Add the abstract to the table of contents
	
	For multiple fields, from biosciences and pharmaceutical drug development to semiconductor and geology research and beyond, fluorescence microscopy has been established as an essential tool which allows for the visualisation of diminutive organisms and objects that escape visibility from the naked eye. The consequent insight into this world has given us a view from an indispensably unique perspective which has since played in vital role in enriching the livelihood of mankind.
	
	Fluorescence microscopy is not the sole player in this venture. It shares the limelight with another important and just as critical field, and that is image segmentation. However, the course between fluorescence images and extracting a meaningful segmentation is laden with barriers and hurdles such as noise, extreme low contrast, non-uniform illumination, etc. The aim of this research is to better understand fluorescence image properties and to leverage this knowledge to enhance and develop domain-specific segmentation methods, using discrete combinatorial techniques, for extracting meaningful results.
	
	We design a pre-processing scheme which attempts to “undo” the adverse effects inherent in the image acquisition stage. Specifically, we tackle noise removal, boundary completion and enhancement and contrast enhancement. We also modify the Chan-Vese segmentation scheme by defining a weighting process and explicit parameter relations.
	We then use the information from an initial unsupervised segmentation to estimate the correct parameter settings for the proposed segmentation method, specific to the image of interest.
	We compare this scheme with other well-known parameter estimation schemes and parameter settings. We also determine the optimal parameter setting for another two energy functions and compare them to the novel energy function for fluorescent image segmentation.
	
	The results of the proposed methods exhibit a significant boost in segmentation correctness and consistency. It also supersedes previous schemes in terms of being applicable over a wider range of images. Furthermore, it boasts a greater sense of generalisation since the parameters can be tuned for other images properties inherent in other domains.
	
	The methods and techniques presented herein will greatly improve the results and stability of higher level decision making as well as removing the need to manually fine-tune segmentation results.
\end{abstract}

%----------------------------------------------------------------------------------------
%	PREFACE
%----------------------------------------------------------------------------------------

\begin{preface}
	\addchaptertocentry{\prefacename} % Add the preface to the table of contents
	
	The research work described in this dissertation was carried out in the School of Engineering, University of KwaZulu-Natal, Durban, from July 2013 to December 2016, under the supervision of Professor Jules-Raymond Tapamo, PhD.
	
	This study is the original work of the author and has not been submitted in any form for any degree or diploma to any tertiary institution. Where use has been made of the work of others, it is duly acknowledged in the text.
	
\end{preface}

%----------------------------------------------------------------------------------------
%	DECLARATION PAGE
%----------------------------------------------------------------------------------------

%\begin{declaration}
%\addchaptertocentry{\authorshipname}
%
%\noindent I, \authorname, declare that this thesis titled, \enquote{\ttitle} and the work presented in it are my own. I confirm that:
%
%\begin{itemize} 
%\item This work was done wholly or mainly while in candidature for a research degree at this University.
%\item Where any part of this thesis has previously been submitted for a degree or any other qualification at this University or any other institution, this has been clearly stated.
%\item Where I have consulted the published work of others, this has been clearly attributed.
%\item Where I have quoted from the work of others, the source has been given. With the exception of such quotations, this thesis is entirely my own work.
%\item I have acknowledged all main sources of help.
%\item Where the thesis is based on work done by myself jointly with others, I have made clear exactly what was done by others and what I have contributed myself.\\
%\end{itemize}
% 
%\noindent Signed:\\
%\rule[0.5em]{25em}{0.5pt} % This prints a line for the signature
% 
%\noindent Date:\\
%\rule[0.5em]{25em}{0.5pt} % This prints a line to write the date
%\end{declaration}
%
%\cleardoublepage

\begin{declarationsupervisor}
	\addchaptertocentry{\decsupervisorname}
	
	As the candidates supervisor, I agree to the submission of this dissertation.
	
	\vfill
	
	\noindent \rule[0.5em]{25em}{0.5pt} % This prints a line for the signature
	\\ \noindent Prof. Jules-Raymond Tapamo
		
\end{declarationsupervisor}

\cleardoublepage

\begin{declarationplagiarism}
	\addchaptertocentry{\decplagiarismname}
	
	I, \authorname, declare that
	\begin{enumerate}
		\item The research reported in this thesis, except where otherwise indicated, is my original research.
		
		\item This thesis has not been submitted for any degree or examination at any other university.
		
		\item This thesis does not contain other persons' data, pictures, graphs or other information, unless specifically acknowledged as being sourced from other persons.
		
		\item This thesis does not contain other persons' writing, unless specifically acknowledged as being sourced from other researchers. Where other written sources have been quoted, then:
		\begin{enumerate}
			\item Their words have been re-written but the general information attributed to them has been referenced.
			
			\item Where their exact words have been used, then their writing has been placed in italics and inside quotation marks, and referenced.
		\end{enumerate}
	
		\item This thesis does not contain text, graphics or tables copied and pasted from the Internet, unless specifically acknowledged, and the source being detailed in the thesis and in the References/Bibliography sections.\\
	\end{enumerate}

	\vfill

	\noindent Signed:\\
	\rule[0.5em]{25em}{0.5pt} % This prints a line for the signature	
\end{declarationplagiarism}

\cleardoublepage

\begin{declarationpublication}
	\addchaptertocentry{\decpublicationname}
	
	Details of contribution to publications that form part and/or include research presented in this dissertation (include publications in preparation, submitted, in press and published and give details of the contributions of each author to the experimental work and writing of each publication)
%	\begin{enumerate}
%		\item PUT PUBLICATION HERE\\
%	\end{enumerate}
\begin{definition}[A Preprocessing Scheme for Fluorescence Microscopy Image Segmentation.]
	Ryan Naidoo and Jules-Raymond Tapamo. In \textit{International Journal of Imaging and Robotics}, pages 1-23, 09 August 2016.
\end{definition}
	
	\vfill
	
	\noindent Signed:\\
	\rule[0.5em]{25em}{0.5pt} % This prints a line for the signature	
\end{declarationpublication}

\cleardoublepage

%----------------------------------------------------------------------------------------
%	QUOTATION PAGE
%----------------------------------------------------------------------------------------

\vspace*{0.2\textheight}

%\noindent\enquote{\itshape Thanks to my solid academic training, today I can write hundreds of words on virtually any topic without possessing a shred of information, which is how I got a good job in journalism.}\bigbreak

%\hfill Dave Barry

\noindent\enquote{\itshape If any of you is deficient in wisdom, let him ask of the giving God [Who gives] to everyone liberally and ungrudgingly, without reproaching or faultfinding, and it will be given him.}\bigbreak

\hfill James 1.5 AMPC

%----------------------------------------------------------------------------------------
%	ACKNOWLEDGEMENTS
%----------------------------------------------------------------------------------------

\begin{acknowledgements}
\addchaptertocentry{\acknowledgementname} % Add the acknowledgements to the table of contents

I am filled with gratitude and thanksgiving to my God Jesus Christ for his favour upon my life. This dissertation would be a blank book if it was not for His grace throughout the duration of my research from the research topic to the people He placed around me to get this done.
 
I truly am grateful for being accepted by the most prolific, honourable and yet humble and patient supervisor in UKZN Howard College, Professor Jules-Raymond Tapamo, PhD. Of all his noteworthy characteristics from guidance to the calibre of work he requires, it was his patience with me for three and half years for which I am thankful.

Finally, the unwavering support I had from my family leaves me at a loss for words. I am so blessed to have a mother that doesn't get tired of encouraging me, and she had to do it very often throughout my studies. I often had to borrow her strength to get through. I think her name should really appear here...Charlotte. The quality of the language in this document would have been much lower if it wasn't for the in-house \textsf{\textipa{'gr\ae m@ 'na:tsiz}}, my father Ivan and my sister Sian. I won't forget how, for the first time, Sian searched her pencil case for the deepest, bloodiest red pen she could find. With every strike, cross and circle her smile grew wider. I am truly grateful for such a loving family.

\end{acknowledgements}

%----------------------------------------------------------------------------------------
%	LIST OF CONTENTS/FIGURES/TABLES PAGES
%----------------------------------------------------------------------------------------

\tableofcontents % Prints the main table of contents

\listoffigures % Prints the list of figures

\listoftables % Prints the list of tables

\listofalgorithms % Prints the list of algorithms

%----------------------------------------------------------------------------------------
%	ABBREVIATIONS
%----------------------------------------------------------------------------------------

\begin{abbreviations}{ll} % Include a list of abbreviations (a table of two columns)

% A
\textbf{ACM} & \textbf{A}ctive \textbf{C}ontour \textbf{M}odel\\
\textbf{ACWE} & \textbf{A}ctive \textbf{C}ontours \textbf{W}ithout \textbf{E}dges\\
\textbf{AHE} & \textbf{A}daptive \textbf{H}istogram \textbf{E}qualisation\\
\textbf{AOD} & \textbf{A}verage \textbf{O}ptical \textbf{D}ensity\\
\textbf{AIDS} & \textbf{A}cquired \textbf{I}mmune \textbf{D}eficiency \textbf{S}yndrome\\
\textbf{ARV} & \textbf{A}nti\textbf{r}etro\textbf{v}iral\\

% B
\textbf{BCC} & \textbf{B}oundary \textbf{C}hain \textbf{C}ode\\
\textbf{BFS} & \textbf{B}readth \textbf{F}irst \textbf{S}earch\\
\textbf{BP} & \textbf{B}elief \textbf{P}ropagation\\

% C
\textbf{CCD} & \textbf{C}harge-\textbf{C}oupled \textbf{D}evice\\
\textbf{CED} & \textbf{C}oherence \textbf{E}nhancing \textbf{D}iffusion\\
\textbf{CLSM} & \textbf{C}onfocal \textbf{L}aser \textbf{S}canning \textbf{M}icroscopy\\
\textbf{CML} & \textbf{C}hronic \textbf{M}yelogenous \textbf{L}eukaemia\\
\textbf{CRF} & \textbf{C}onditional \textbf{R}andom \textbf{F}ield\\

% D
\textbf{DCC} & \textbf{D}ifferential \textbf{C}hain \textbf{C}ode\\
\textbf{DFS} & \textbf{D}epth \textbf{F}irst \textbf{S}earch\\
\textbf{DNA} & \textbf{D}eoxyribo\textbf{n}ucleic \textbf{A}cid\\
\textbf{DP} & \textbf{D}ynamic \textbf{P}rogramming\\
\textbf{DT} & \textbf{D}elaunay \textbf{T}riangulation\\

% E
\textbf{EGFP} & \textbf{E}nhanced \textbf{G}reen \textbf{F}luorescent \textbf{P}rotein\\
\textbf{EM} & \textbf{E}xpectation \textbf{M}aximisation\\

% F
\textbf{FCS} & \textbf{F}luorescence \textbf{C}orrelation \textbf{S}pectroscopy\\
\textbf{FDA} & \textbf{F}ood and \textbf{D}rug \textbf{A}dministration\\
\textbf{FIFO} & \textbf{F}irst-\textbf{I}n \textbf{F}irst-\textbf{O}ut\\
\textbf{FISH} & \textbf{F}luorescence \textbf{i}n-\textbf{s}itu \textbf{H}ybridisation\\
\textbf{FLIM} & \textbf{F}luorescence \textbf{L}ifetime \textbf{I}maging \textbf{M}icroscopy\\
\textbf{FM} & \textbf{F}luorescence \textbf{M}icroscopy\\
\textbf{FN} & \textbf{F}alse \textbf{N}egatives\\
\textbf{FP} & \textbf{F}alse \textbf{P}ositives\\
\textbf{FRAP} & \textbf{F}luorescence \textbf{R}ecovery \textbf{A}fter \textbf{P}hotobleaching\\
\textbf{FRET} & \textbf{F}luorescence \textbf{R}esonance \textbf{E}nergy \textbf{T}ransfer\\

% G
\textbf{GA} & \textbf{G}enetic \textbf{A}lgorithm \\
\textbf{GC} & \textbf{G}raph \textbf{C}uts \\
\textbf{GCBLS} & \textbf{G}raph \textbf{C}ut \textbf{B}ased \textbf{L}evel \textbf{S}et\\
\textbf{GFP} & \textbf{G}reen \textbf{F}luorescent \textbf{P}rotein\\
\textbf{GLCM} & \textbf{G}ray \textbf{L}evel \textbf{C}o-occurrence \textbf{M}atrix\\
\textbf{GMM} & \textbf{G}aussian \textbf{M}ixture \textbf{M}odelling\\
\textbf{GRF} & \textbf{G}ibbs \textbf{R}andom \textbf{F}ield\\
\textbf{GVF} & \textbf{G}radient \textbf{V}ector \textbf{F}low\\

% H
\textbf{HGP} & \textbf{H}uman \textbf{G}enome \textbf{P}roject\\
\textbf{HIV} & \textbf{H}uman \textbf{I}mmune \textbf{V}irus\\
\textbf{HLF} & \textbf{H}ighest \textbf{L}evel \textbf{F}irst\\

% I
\textbf{ICC} & \textbf{I}mmuno\textbf{c}yto\textbf{c}hemistry\\
\textbf{ICF} & \textbf{I}mmuno\textbf{c}yto\textbf{f}luorescence\\
\textbf{ICM} & \textbf{I}terated \textbf{C}onditional \textbf{M}odes\\
\textbf{IHC} & \textbf{I}mmuno\textbf{h}isto\textbf{c}hemistry\\
\textbf{IHF} & \textbf{I}mmuno\textbf{h}isto\textbf{f}luorescence\\
\textbf{IOD} & \textbf{I}ntegrated \textbf{O}ptical \textbf{D}ensity\\

% J
% K
% L
\textbf{Laser} & \textbf{L}ight \textbf{A}mplification by \textbf{S}timulated \textbf{E}mission of \textbf{R}adiation\\
\textbf{LBP} & \textbf{L}oopy \textbf{B}elief \textbf{P}ropagation\\
\textbf{LED} & \textbf{L}ight \textbf{E}mitting \textbf{D}iode\\
\textbf{LoG} & \textbf{L}aplacian \textbf{o}f \textbf{G}aussian\\

% M
\textbf{MAP} & \textbf{M}aximum \textbf{A} \textbf{P}osteriori\\
\textbf{MCC} & \textbf{M}atthews \textbf{C}orrelation \textbf{C}oefficient\\
\textbf{MIS} & \textbf{M}edical \textbf{I}mage \textbf{S}egmentation\\
\textbf{MLP} & \textbf{M}ulti-\textbf{L}ayered \textbf{P}erceptron\\
\textbf{MRF} & \textbf{M}arkov \textbf{R}andom \textbf{F}ield\\
\textbf{MST} & \textbf{M}inimum \textbf{S}panning \textbf{T}ree\\

% N
\textbf{NA} & \textbf{N}umerical \textbf{A}perture\\

% O
\textbf{ORI} & \textbf{O}ptimised \textbf{R}otational \textbf{I}nvariance\\
\textbf{OTF} & \textbf{O}ptical \textbf{T}ransfer \textbf{F}unction\\

% P
\textbf{PSF} & \textbf{P}oint \textbf{S}pread \textbf{F}unction\\

% Q
% R
\textbf{RF} & \textbf{R}andom \textbf{F}ield\\
\textbf{ROC} & \textbf{R}eceiver \textbf{O}perating \textbf{C}haracteristics\\
\textbf{RNA} & \textbf{R}ibo\textbf{n}ucleic \textbf{A}cid\\

% S
\textbf{SNR} & \textbf{S}ignal-to-\textbf{N}oise \textbf{R}atio\\
\textbf{SVM} & \textbf{S}upport \textbf{V}ector \textbf{M}achine\\

% T
\textbf{TN} & \textbf{T}rue \textbf{N}egatives\\
\textbf{TP} & \textbf{T}rue \textbf{P}ositives\\
\textbf{TV} & \textbf{T}otal \textbf{V}ariation\\
%\textbf{TRW} & \textbf{T}ree \textbf{R}e\textbf{w}eighted\\

% U
\textbf{UV} & \textbf{U}ltra\textbf{v}iolet\\

% V
% X
% Y
% Z

\end{abbreviations}

%----------------------------------------------------------------------------------------
%	PHYSICAL CONSTANTS/OTHER DEFINITIONS
%----------------------------------------------------------------------------------------

%\begin{constants}{lr@{${}={}$}l} % The list of physical constants is a three column table
%
%% The \SI{}{} command is provided by the siunitx package, see its documentation for instructions on how to use it
%
%	Speed of Light & $c$ & \SI{2.99792458e8}{\meter\per\second} (exact)\\
%	Planck's Constant & $h$ & \SI{6.62607004e-34}{\meter\squared\kilogram\per\second} (exact)\\
%%Constant Name & $Symbol$ & $Constant Value$ with units\\
%
%\end{constants}

%----------------------------------------------------------------------------------------
%	SYMBOLS
%----------------------------------------------------------------------------------------

%\begin{symbols}{lll} % Include a list of Symbols (a three column table)
%
%$a$ & distance & \si{\meter} \\
%$P$ & power & \si{\watt} (\si{\joule\per\second}) \\
%%Symbol & Name & Unit \\
%
%\addlinespace % Gap to separate the Roman symbols from the Greek
%
%$\omega$ & angular frequency & \si{\radian} \\
%
%\end{symbols}

%----------------------------------------------------------------------------------------
%	DEDICATION
%----------------------------------------------------------------------------------------

\dedicatory{To God be the glory} 

%----------------------------------------------------------------------------------------
%	THESIS CONTENT - CHAPTERS
%----------------------------------------------------------------------------------------

\mainmatter % Begin numeric (1,2,3...) page numbering

\pagestyle{thesis} % Return the page headers back to the "thesis" style

% Include the chapters of the thesis as separate files from the Chapters folder
% Uncomment the lines as you write the chapters

% Chapter 1

\chapter{Introduction} % Main chapter title

\label{chap:Chapter1} % For referencing the chapter elsewhere, use \ref{Chapter1} 

%----------------------------------------------------------------------------------------

% Define some commands to keep the formatting separated from the content 
\newcommand{\keyword}[1]{\textbf{#1}}
\newcommand{\tabhead}[1]{\textbf{#1}}
\newcommand{\code}[1]{\texttt{#1}}
\newcommand{\file}[1]{\texttt{\bfseries#1}}
\newcommand{\option}[1]{\texttt{\itshape#1}}

%----------------------------------------------------------------------------------------

\section{What is Image Segmentation}

definition\\
types: region, edge\\
human segmentation -> Gestalt Groupings\\
Machine segmentation\\
Good segmentation vs Bad Segmentation\\
Goal of Image segmentation\\
Differences between bottom-up and top-down image segmentation\\
number of labels (2 labels -> binarization or binary segmentation, etc)

%----------------------------------------------------------------------------------------

\section{Gestalt Theory of Visual Perception}

A psychological view of visual perception. The aim here is to give a brief realisation of the current understandings of human visual perception. Since image segmentation is predominantly guided by subjective human perception, it is wholesome to understand, at least briefly, what human perception is all about; at least to our current understanding.

[It would be good to get a few experts to manually segment the same images and do a similarity comparison. This will prove that even experts in the same field are subject to their own interpretation of an image. Also compare the manual segmentation to people that are not experts in the field. Discuss, how trustworthy are manual segmentations?]

Gestalt - movement in experimental psychology. Developed in Germany We percieve objects as well-origanised patterns rather than seperate components. Gestalt is a theory that the brain operates wholistically, with self-organising tendencies. "The whole is greater than the sum of its parts." Illusory Contours - The Kanisza triangle as figure ground illusory contours. Three main principles: Grouping(proximity, similarity, continuity, closure), Goodness of Figures, Figure/Ground Relationships.

Gestalt - The German word mean Organised Whole

Goodness of Figure, or the Law of Pragnaz. Pragnaz is the German word for Pregnant, but in the sense of pregnant with meaning, not with child.

Figure/Ground Relationships: Figure-Foreground, Ground-Background, Contours-"belong" to the Figure. Reversible Figure/Ground Relationship.

Problems with Gestalt Theory: 
- It is a phenomenological approach.
- Some terms are vague. E.g What is the simplest organisation?

\begin{definition}[Contrast]
	When perception is influenced by comparison. Three types:\\
	Brightness Contrast\\
	Colour Contrast\\
	Size Contrast
\end{definition}

\begin{definition}[Context]
	When a stimulus can be interpreted in more than one way, the context resolves the ambiguity.
\end{definition}

\begin{definition}[Figure Ground]
	Separation of an image into figure and ground.
\end{definition}

\begin{definition}[Closure]
	We tend to see figures as whole even though lines enclosing them are incomplete.
\end{definition}

\begin{definition}[Good Continuation, or Good Figure]
	Where lines intersect, we tend to see them as continuing along their previous course, rather than suddenly changing direction. As a result we tend to decompose figures into their simplest components.
\end{definition}

\begin{definition}[Perceptual Consistancies]
	4 types:\\
	Shape Consitency: We tend to see object as holding its essential shape even though the shape of its image changes with our view of it.\\
	Size Consistency: An object appears to retain its essential size event though its image size changes with distance.\\
	Brightness/Lightness Consistency: Object seem to retain about the same brightness or lightness under widely differing levels of illumination.\\
	Colour Consistency: Object appear to retain their essential colour even though illuminated by somewhat differently coloured lights.
\end{definition}

\begin{definition}[Principles of Grouping]
	The principles by which  you recognise object as belonging to the same group. They include:\\
	Similarity: Object are recognised as belonging to the same group when they have a similar appearance.\\
	Proximity: We percieve object as belonging to the same group based on their relative distances from one another. \\
	Common Fate: We percieve object as belonging to the same group when the same things are happening to them.
\end{definition}

%----------------------------------------------------------------------------------------

\section{Energy Minimisation}

Brief energy minimisation and what it is. Huge number of problems in vision are inference problems which can be found from energy minimisation. The relation of the concept of energy in physics is not important but useful.

Vision problems are much more complex involving hundreds or even millions of interdependant variables. Some energies precisely model the desired inference problem while some are coarse approximations. Some energies are easy to optimise while others are known to be NP-hard. Once an accurate energy and satisfying algorithm are available, the associated inference problem is essentially solved.

Many of the most important developments in computer vision began with a proposal for a better energy, better algorithm or a combination of both.

%----------------------------------------------------------------------------------------

\section{Labelling Problems}

What is a labelling problem? Type of labelling problems. What is needed for a labelling problem. What is a discrete labelling problem in vision? What are the type of labels (semantic or related to geometry).

Data driven criteria - Data influences the outcome. These preferences are dereived from machine learning.

Regularisation criteria - When we explicitely prefer some kinds  of labelling over others, these criteria are called regularisers. The most prominent in computer vision are spatially coherent labellings. The idea of smoothness in preference to noisy. We know from experience that object and medical data correpsonds to coherent labels more often than not - varies from application to application but has now become a rule of thumb. Because data in computer vision tends to be highly correlated in space.

%----------------------------------------------------------------------------------------

\section{Labelling Problems as Energy Minimisation}

General case of expressing data-driven and regularisation criteria as concrete energy decisions.

smoothness, neighbourhood, joint probability

Markov Random Field, MAP-MRF problem

%----------------------------------------------------------------------------------------

\section{Energy Minimisation Algorithms and Special Cases}\label{FillingFile}

Dynamic Programming 

Binary Energies with Coherence

%----------------------------------------------------------------------------------------

\section{Thesis Overview}

The remainder of the thesis outline.\\
\\
\textbf{\hyperref[chap:Chapter2]{Chapter~\ref*{chap:Chapter2}}} is where we cover the mathematical foundation to Graph Cut image segmenation.\\
\\
\textbf{\hyperref[chap:Chapter3]{Chapter~\ref*{chap:Chapter3}}} is where we cover the mathematical foundation to Graph Cut image segmenation.\\
\\
\textbf{\hyperref[chap:Chapter4]{Chapter~\ref*{chap:Chapter4}}} is where we cover the mathematical foundation to Graph Cut image segmenation.\\
\\
\textbf{\hyperref[chap:Chapter5]{Chapter~\ref*{chap:Chapter5}}} is where we cover the mathematical foundation to Graph Cut image segmenation.\\
\\
\textbf{\hyperref[chap:Chapter6]{Chapter~\ref*{chap:Chapter6}}} is where we cover the mathematical foundation to Graph Cut image segmenation.\\
\\
\textbf{\hyperref[chap:Chapter7]{Chapter~\ref*{chap:Chapter7}}} is where we cover the mathematical foundation to Graph Cut image segmenation.\\
\\
\textbf{\hyperref[chap:Chapter8]{Chapter~\ref*{chap:Chapter8}}} concludes the thesis with suggestions for further work.
 % Introduction
% Chapter Template

\chapter{Fluorescence Microscopy} % Main chapter title

\label{chap:Chapter2} % Change 2 to a consecutive number; for referencing this chapter elsewhere, use \ref{chap:Chapter2}

%\citep{SpringDavisdson2016}
%\citep{Rice2016}
\citep{Nobel2016}
%\citep{AbramowitzDavidson2016}
%\citep{ThermoFisher2016}
%\citep{LichtmanConchello2005}
%\citep{Spring2003}
\citep{Biehlmaier2013}
\citep{Svoboda2007}
\citep{Svoboda2009}
\citep{WuMerchantCastleman2008}
\citep{GonzalezWoods2002}
\citep{Pratt2001}
\citep{Soile2004}
%\citep{Murphy2001}
\citep{Matula2006}
\citep{Rohr2010}
\citep{Matula2000}
%\citep{Sarder2006}
\citep{Vu2008}
\citep{Kolmogorov2004}
%\citep{Hubeny2008}
\citep{Kozubek2001}
\citep{Petran1985}
%\citep{Tsien1998}

\textcolor{red}{What is fluorescence microscopy? What's it's purpose in the thesis?	Photoluminescence -> fluorescence and phosphorescence. Discovery of fluorescence: Brief history and evolution. Brief discussion on the remainder of the chapter.}
Fluorescence microscopy has become an essential tool in diverse fields, such as petrology, semiconductors, etc, and has especially been established as a choice imaging technique in cellular and molecular biology for visualisation of cells and tissues \citep{Spring2003,Danek2012,Hubeny2008,Fatima2008,Matula2012}.
In this thesis we confine our attention to its use in cellular biology.

Certain substances emit radiation when irradiated with a higher intensity light, such as ultraviolet (UV), blue or green, which is off a longer wavelength than that of the exciting light, this is known as \textit{Stokes' Law}.
This phenomenon is known as \textit{photoluminescence} \citep{Koch1972,Vaughan2015,Sarder2006,AbramowitzDavidson2016}.
There are two types of photoluminescence. If emission persists at an appreciable level after the exciting light is turned off, then we call this \textit{phosphorescence}.
If emission persist only so long as the exciting light is on, then we call this \textit{fluorescence} \citep{Koch1972,SpringDavisdson2016}.

The first observance and publishing of fluorescence is credited to Sir John Frederick William Herschel around 1852.
In 1852, Sir John George Stokes published a 100 page treatise about this luminescent phenomenon and coined the term \textit{fluorescence}, over Herschel's \textit{dispersive reflection}, when he observed that the mineral \textit{fluorite} emitted red light when irradiated by ultraviolet (UV) light \citep{Dobrucki2013,Danek2012}.

In the remainder of this chapter we present the underlying principles of fluorescence, how specimens are fluorescently marked, the optical principles of microscope design, image acquisition, image processing and common analysis in cellular biology. We only go so far in depth as to present a rudimentary understanding of fluorescence microscopy as is necessary for the comprehension of this thesis.

%----------------------------------------------------------------------------------------
%	SECTION 1
%----------------------------------------------------------------------------------------

\section{Physics of Fluorescence}
\label{sec:PhysicsOfFluorescence}

Excitation and Emission, Fluorophores, Jablonski Diagram, Electronic States, Stoke's Shift
\begin{figure}[!t]
	\centering
	\includegraphics[width=\columnwidth]{Fluor532_ExcitationEmissionSpectrum.png}
	\caption{Normalised Excitation and Emmision Spectra of the Alexa Fluor 532 flurophore. The emmission maximum is at $553nm$ which is a more yellow-green than excitation maximum at $528nm$. This image was generated using FluoScout\texttrademark\, web application by Leica Microsystems for determining the optimal fluorescence filter cube set. %\url{http://www.leica-microsystems.com/fluoscout/}
	}
	\addloflink{http://www.leica-microsystems.com/fluoscout/}
	\label{fig:excitationandemissionspectra}
\end{figure}

\begin{figure}[!t]
	\centering
	\subfigure[]
	{
		\includegraphics[width=0.4\columnwidth]{molecular_absorption.png}
		\label{fig:molecularabsorption}
	}
	\subfigure[]
	{
		\includegraphics[width=0.56\columnwidth]{jablonski_diagram.png}
		\label{fig:jablonski}
	}
	\caption{\textbf{(a)} Simplified fluorescence process. \textbf{(b)} The Jab{\l}o{\'n}ski diagram depicting the electronics states from photon absorption to photo emission.}
	\label{fig:thinkoflabel}
\end{figure}

%----------------------------------------------------------------------------------------
%	SECTION 2
%----------------------------------------------------------------------------------------

\section{Specimen Labelling}
\label{sec:SpecimenLabelling}

\textcolor{red}{Why do specimens have to be stained? What is is staining?}
Many of the components of interest, such as cell nuclei, cytoplasm, genes, chrosomes, proteins, do not possess a high degree of, if not any,  autofluorescence. 
In this scenario, these components can be marked with a fluorescent dye \citep{Tsien1998}, also known as a fluorophore or fluorochrome, a substance that can bind to a specific target whose excitation and emission spectra are well known. 
This is known as staining \citep{Danek2012,Hubeny2008,Dobrucki2013}. 
Once the specimen is stained it can be indirectly observed using a fluorescence microscope.

\textcolor{red}{What are the most common staining protocols?}
The most prevalent staining techniques are fluorescence in-situ hybridisation (FISH) and immunostaining \citep{Danek2012,Fatima2008,Kozubek2001_2,Theodosiou2007}.

\begin{definition}[FISH staining]
	\textcolor{red}{What is FISH?	What is the FISH staining techniques used for?}
	FISH is a molecular cytogenetic technique that uses flourophores that bind to selected regions in nucleic acids \citep{Danek2012,Fatima2008}.
	FISH is the most frequently used staining technique used primarily for visualisation and localisation of nucleic acid sequences, chromosomes, cytplasm or organelles which contain those acids \citep{Hubeny2008}.
	This makes FISH highly attractive for finding specific features in DNA and RNA used in genetic diagnosis and research, medicine and species identification \citep{Amann2008,Fatima2008}.
	Figure \ref{fig:FISH} is a capture of mouse chromosomes using the FISH staining technique.
\end{definition}

\begin{definition}[Immunostaining]
	\textcolor{red}{What is Immunofluorescence and the two main types, what is the difference between the two, and which is more common? What is the Immunofluorescence staining techniques used for?}
	Immunofluorescence is the detection method where an antibody is used to detect an antigen in a tissue or a cell using fluorescence. Flourophores are usually conjugated onto antibodies, which are proteins that are designed bind to specific antigens, target proteins, on a cell \citep{CudeBurke2014}.
	The two types of immunofluorescent detection are immunocytofluorescence (ICF) and immunohystofluorescence (IHF).
	It must not be confused with immunocytochemistry (ICC) and immunohistochemistry (IHC).
	\textit{Immuno} refers to the immunological technique, i.e. the binding of antibodies to antigens.
	\textit{Cyto} refers to cells, i.e. cells without the extracellular membrane.
	\textit{Histo} refers to tissue i.e. cells with the extracellular membrane.
	\textit{Chemistry} refers to the chemical method of detection, e.g. a change in colour.
	\textit{Fluorescence} detection by emission of light \citep{Katikireddy2011}.
	Figure \ref{fig:IHC} shows the detection of the p53 Binding Protein 1 in perfusion fixed frozen sections of rat kidney.
\end{definition}

\begin{definition}[Live-cell staining]
	\textcolor{red}{FISH and IHC cannot stain live cells. Why? How can we stain live cells?}
	The previously discussed staining techniques are not suitable to observe living cells.
	The fluorescent dyes used are phototoxic and cause cells to die. The circumvent this problem an elegant solution has been devised.
	Instead of staining, the cells are modified to produce a fluorescent substance in the target structures.
	Derivatives of the \textit{green fluorescent protein} (GFP), isolated from the \textit{Aequorea victoria} jellyfish \citep{Tsien1998,LichtmanConchello2005,Fatima2008}, are used as it generates a strong photon emission and is non-toxic to living cells \citep{Danek2012,Hubeny2008,Dobrucki2013}.
\end{definition}

\textcolor{red}{Important notes about fluorophores and the impact on image quality?}

\begin{figure}[!t]
	\centering
	\subfigure[]
	{
		\includegraphics[width=0.495\columnwidth]{fish1.jpg}
		\label{fig:FISH}
	}
	\subfigure[]
	{
		\includegraphics[width=0.45\columnwidth]{ihc1.jpg}
		\label{fig:IHC}
	}
	\caption{\textbf{(a)} FISH (Fluorescent 'in-situ' Hybridization) in mouse chromosomes using a BAC clone labeled with Spectrum Orange. The picture shows two metaphases and one interphase with two signals in each exampling a homozygous mouse for a transgenic clone. Image Source: "All About the Human Genome Project" Genetic and Genomic Image and Illustration Database. %\addloflink{https://unlockinglifescode.org/media/images/} %\url{https://unlockinglifescode.org/media/images/}. 
	\textbf{(b)} p53 Binding Protein 1 (53BP1) was detected in perfusion fixed frozen sections of rat kidney using Goat Anti-Human 53BP1 Antigen Affinity-purified Polyclonal Antibody (Catalog \# AF1877) at 15 $\mu$g/mL overnight at 4$^{\circ}$C. Tissue was stained using the NorthernLights\texttrademark 557-conjugated Anti-Goat IgG Secondary Antibody (red; Catalog \# NL001) and counterstained with DAPI (blue). Specific staining was localized to nuclei of epithelial cells in convoluted tubules. Image Source: R\&D Systems' IHC image database.}
	%\addloflink{https://www.rndsystems.com/resources/ihc-images/53bp1}}
	%\url{https://www.rndsystems.com/resources/ihc-images/53bp1}.}
	\addloflink{https://unlockinglifescode.org/media/images/}
	\addloflink{https://www.rndsystems.com/resources/ihc-images/53bp1}
	\label{fig:stainingtechniques}
\end{figure}

%----------------------------------------------------------------------------------------
%	SECTION 3
%----------------------------------------------------------------------------------------

\section{The Epifluorescence Microscope and Image Acquisition}
\label{sec:TheEpifluorescenceMicroscope}

\begin{figure}[!t]
	\centering
	\includegraphics[width=0.4\columnwidth]{Epifluorescence_Microscope.png}
	\caption{The schematic of the epifluorescence microscope.}
	\label{fig:epifluorescencemicroscope}
\end{figure}

\textcolor{red}{What is a fluorescent microscope? Schematic layout of a fluorescence microscope? Function and purpose of each component in the fluorescent microscope?}
A fluorescence microscope is an optical microscope that is designed specifically to exploit the principle of fluorescence to allow for the observation of flourescently labelled specimens \citep{Hubeny2008,Sarder2006,Dobrucki2013,Andrews2002,Fatima2008}.
There are many types of fluorescent micrcoscopes avaialble but the favoured type among many biologists and geneticists is the epifluorescent microscope \citep{Rice2016,AbramowitzDavidson2016}.
The schematic of the epifluorescent micrcoscope is illustrated in \autoref{fig:epifluorescencemicroscope}.

\begin{definition}[Light Source]
	\textcolor{red}{What sort of light needs to be generated? What sort of lamps are used? Advantages and disadvantages of certain lamps.}
	The light source is typically a high-luminance light source e.g. Mercury or Xenon arc lamps, LEDs, lasers, etc  \citep{Danek2012,Hubeny2008,Aswani2012,Rice2016,ThermoFisher2016}.
	The primary criterion for choosing a light sources is that its characeristic peaks must coincide with the excitation spectrum of the fluorophores being used \citep{LichtmanConchello2005,Spring2003,Fatima2008}.
	Wavelength coverage spans from near infra-red to UV. Mercury and Xenon arc lamps are expensive, an inexpensive and lightweight alternative is bright LEDs \citep{Fatima2008,Dobrucki2013,Aswani2012,Koch1972}.
\end{definition}

\begin{definition}[Excitation Filter]
	\textcolor{red}{What is an excitation filter? Why is it needed?}
	The incoming light from the light source is typically mulispectral \citep{SpringDavisdson2016}. 
	The excitation filter is a wavelength selection filter which is placed in the path of the incoming light and filters through only those wavelengths in the absorption spectrum of the fluorescent dye \citep{ThermoFisher2016,Danek2012,Hubeny2008,LichtmanConchello2005,Spring2003,CudeBurke2014,Fatima2008,Dobrucki2013}.
\end{definition}

\begin{definition}[Dichroic Mirror]
	\textcolor{red}{What is a dichroic mirror? Why is it needed?}
	Also known as a \textit{dichroic beam splitter}.
	This is placed at a 45$^{\circ}$ angle and reflects the short-wavelenght light filtered through the excitation filter at a 90$^{\circ}$ angle towards the specimen \citep{Danek2012,Hubeny2008,Spring2003,CudeBurke2014} and allows the long-wavelength light from the fluorescing specimen to pass through to the detector \citep{LichtmanConchello2005,Koch1972}, thus serving as a separation filter between the absorption and emission light \citep{Fatima2008,Dobrucki2013}.
\end{definition}

\begin{definition}[Objective]
	\textcolor{red}{What is the objective? Why is it needed?}
	The incoming light reflected of the dichroic mirror passes through the objective lens before reaching the specimen \citep{Danek2012,Hubeny2008,LichtmanConchello2005,Spring2003}.
	Emission light from the fluorescing specimen is gathered in the objective lens and passed through to the dichroic mirror.
\end{definition}

\begin{definition}[Specimen]
	\textcolor{red}{Say something about the specimen, for wholeness sake.}
	The specimen is irradiated by the incoming light from the objective and emits long-wavelength light in all directions.
	The specimen is stained with a flourophore whose absorption and emission curves are well known.
	This is important since the light source and the interference filters are chosen using the peaks of these curves.
\end{definition}

\begin{definition}[Emission Filter]
	\textcolor{red}{What is an emission filter? Why is it needed?}
	Also known as a \textit{barrier filter} \citep{LichtmanConchello2005,Spring2003,Koch1972}.
	The light coming from the specimen contains multiple wavelengths and the dichroic mirror is used to filter out the shorter wavelength light.
	The emission filter is further  used to filter out the wavelengths that correspond to the emission wavelengths of the fluorophore \citep{CudeBurke2014,Danek2012,Hubeny2008,SpringDavisdson2016,ThermoFisher2016}.
\end{definition}

%\begin{definition}[Occular]
%	\textcolor{red}{What is the occular why is it needed?}
%	Purpose of the occular.
%\end{definition}

\begin{definition}[Detector]
	\textcolor{red}{What is the detector why is it needed?}
	The dector is used to capture the emission light and can further digital form the image.
	The detector is usually a CCD (charge-coupled device) camera or a photomultipler tube \citep{Danek2012,Hubeny2008,LichtmanConchello2005,Spring2003,Murphy2001}.
	It is vital that an appropriate detector be chosen as this has direct influence of image quality \citep{Fatima2008}.
\end{definition}

\textcolor{red}{Other Types of Fluorescence Microscopes: Confocal, TIRF, Epifluorescence, Acquisition: CCD, Hardware setup effect on image quality, Numerical Aperture, Sub-diffraction}


%----------------------------------------------------------------------------------------
%	SECTION 4
%----------------------------------------------------------------------------------------

\section{Image Processing in FM}
\label{sec:ImageProcessingInFM}

Limitations in Fluorescence Imaging\\
Preprocessing: Point Spread Function deconvolution, etc\\
Segmentation 

%----------------------------------------------------------------------------------------
%	SECTION 5
%----------------------------------------------------------------------------------------

\section{Measurements and Analysis in FM}
\label{sec:Measurements}

what is measured and for what?\\
Motion, number of cells, area, volume, lenght % Fluorescence Microscopy
% Chapter Template

\chapter{Mathematical Background} % Main chapter title

\label{chap:Chapter3} % Change to a consecutive number; for referencing this chapter elsewhere, use \ref{chap:Chapter2}

\textcolor{red}{Optimisation approach in vision, "machinery/mechanics", literature review, ill-posed inverse problems.}

Image segmentation falls under the mathemtical classification as being an \textit{ill-posed inverse problem} \citep{Poggio1985,Terzopoulos1986}.
It is an inverse problem since we require a model from the observation, this simply means given the results, what are the causes.
In image segmentation this translates to, given a 2D matrix of intensity values, which pixels belong to the object and which belong to the background.
Image segmentation is also an ill-posed problem since their is a lack of uniqueness or stability of a solution \citep{Kabanikhin2008}, which are two of the three requirements for a solution to be \textit{well-posed}, the other being existence.
Image segmentation is an ill-posed because immense amount of information is suppressed in the acquisition processed \citep{Tarantola2005,Bertero1998,Bertero2006}.
Many tasks in vision are inherently or can be reformulated as ill-posed inverse problems e.g. scene reconstruction, stereo matching, image restoration, image deconvolution, etc.
Computer vision is used heavily in industry, medicine and life science fields included, hence there is a need for robust, enviromentally resistant approach.
The \textit{optimisation approach} is an elegant way to obtain a solution.
In computer vision, a problem can be posed as an optimisation problem as follows: We are given a coarse, discrete and noisy, approximation of the visual data, $d$, we aim to infer some hidden quantities $x$, labels, depth, probable pixel intensty, etc, based on it.
We then have to design an \textit{objective function}, also known as an \textit{energy function} or \textit{cost function},

\begin{equation}
	E:(x,d) \rightarrow \Re,
\end{equation}

which has to be optimised such that the optimsation of the function provides a solution to the problem.
$E(x,d)$ assigns an energy or a cost to each combination $(x,d)$ of the input and hidden quantities.
$E$ provides a measure of goodness to how well the candidate solution $x$ fits the expectation given the data $d$.
In optimisation of this function we seek a minimum energy,

\begin{equation}
	x^{*} = \argmin_{x}E(x,d),
\end{equation}

which has roots in Statistical Physics where lower energies correspond to more stable solutions.
This gives us a general idea of how we should assign energies to solutions; better a solution, the lower an energy we should assign to it.
In this case, a huge number of inference problems in vision can be solved by minimising the associated energy.
A solution is only as good as the energy model and the optimisation technique. Once a precice energy and minimising algorithm are found, the problem is essentially solved \citep{Delong2011}. 

Early attempts in computer vision would solve problems like these using iteration or relaxation methods \citep{Waltz1975,Rosenfeld1976}.
In these attempts the problems are solved in a Calculus of variations framework, this is still a popular approach to optimisation in vision since Poggio et al. \citep{Poggio1985} proposed an integrated framework to regularisation theory for vision \citep{Sakaue1999}.
Many important advancements in computer vision are proposals for a better energy, a better algorithm or both \citep{Delong2011,Boykov2001,Kolmogorov2005,Mumford1989,Shi1997}.
In this thesis we focus on discrete energy optimisation using graph cuts for image segmentation.

\textcolor{red}{Plan for the chapter.}
Image segmentation falls under a broader catergory of problems known as \textit{labelling problems}.
The aim is to find the best label, foregound/object or background, for each pixel.
In \Cref{sec:LabellingProblems} we briefly discuss labelling problems and its formulation as an energy minimisation problem.

%----------------------------------------------------------------------------------------
%	SECTION 1
%----------------------------------------------------------------------------------------

\section{Labelling Problems}
\label{sec:LabellingProblems}

Among the many computer vision problems, image segmentation is the most easy to understand labelling problem.
A labelling problem is simply assigning, to an observation, a label that most accurately explains it.
An observation can be anything that we wish to classify e.g. pixels, features, salient points, depth measurement, etc.
A label is a description of that observation.
There are two types of labels: \textit{semantic labelling} (person, car, tree, sky, face, eye, etc) or \textit{pixel-wise labelling} (texture, shape, colour, background/object, etc) \citep{Delong2011,Athanasiadis2007}.

To formulate a labelling problem we need a set of \textit{cites}, intuitively known as observations, and a set of \textit{labels}, a set of explanations.
The goal is to find the best explanation given the observations.
In computer vision, the observations, can be features, image segments, etc.
However, they will typically represent pixels in an image with some natual structure or ordering.
Let 

\begin{equation*}
	\mathcal{P} = \{1, 2, \ldots, n\}
\end{equation*}

be the set of $n$ cites and 

\begin{equation*}
\mathcal{L} = \{l_1, l_2, \ldots, l_k\}
\end{equation*}

be the set of $k$ labels.
A discrete labelling is a map $f: \mathcal{P} \rightarrow \mathcal{L}$ that assigns each discrete variable $f_p$ one value from $\mathcal{L}$ and $f=\{f_p\}_{p \in \mathcal{P}}$ which is known as a \textit{configuration}.
We are interested in binary segmentation, also known as \textit{binarization}, which implies we have two explanations in our label set, $k=2$.
The labels of interest are the \textit{background} and the \textit{object}.
Although the solution space is finite, it is very large and grows exponentially as the image size increases or as the number of labels increases.
The number of possible configurations is given by $|\mathcal{L}|^{|\mathcal{P}|}$.
Table \ref{tab:configuration} shows the largeness of the solution space even for very small images and a few labels.
In pratice, the image sizes used in Table \ref{tab:configuration} is too small, hence finding a solution is not easy.
Most often we have to settle for an approximate solution, one that is "good enough".

\def\arraystretch{1.2}
\begin{table}[ht]
	\caption{The impact of the number of cites and labels on the solution space}
	\label{tab:configuration}
	\begin{tabular}{|c|c|c|c|}
		\hline 
		Image ($\mathcal{P}$) & Number of cites ($|\mathcal{P}|$) & Number of labes ($|\mathcal{L}|$) & Number of configurations $|\mathcal{L}|^{|\mathcal{P}|}$ \\ 
		\hline 
		$64 \times 64$ & $2^{12}=4096$ & $2$ & $2^{2^{12}} = 2^{4096}=n$ \\
		\hline 
		$128 \times 128$ & $2^{14}=16384$ & $2$ & $2^{2^{14}} = 2^{16384} = n^4$ \\ 
		\hline 
		$64 \times 64$ & $2^{12}=4096$ & $3$ & $3^{2^{12}} = 3^{4096} \approx n^{1.585}$ \\ 
		\hline 
	\end{tabular}
\end{table}
 

%----------------------------------------------------------------------------------------
%	SECTION 2
%----------------------------------------------------------------------------------------

\section{MAP Estimation for Discrete Models}
\label{sec:MAPEstimates}

%----------------------------------------------------------------------------------------
%	SUBSECTION 1
%----------------------------------------------------------------------------------------

\subsection{Random Fields}
\label{sec:RandomFields}

%----------------------------------------------------------------------------------------
%	SUBSECTION 2
%----------------------------------------------------------------------------------------

\subsection{Markov Random Fields}
\label{sec:MarkovRandomFields}

%----------------------------------------------------------------------------------------
%	SUBSECTION 3
%----------------------------------------------------------------------------------------

\subsection{Conditional Random Fields}
\label{sec:ConditionalRandomFields}

%----------------------------------------------------------------------------------------
%	SUBSECTION 4
%----------------------------------------------------------------------------------------

\subsection{CRFs for Image Segmentation}
\label{sec:ConditionalRandomFieldsForImageSegmentation}

%----------------------------------------------------------------------------------------
%	SUBSECTION 5
%----------------------------------------------------------------------------------------

\subsection{MAP-MRF Estimation}
\label{sec:MAPMRFEstimation}

%----------------------------------------------------------------------------------------
%	SECTION 3
%----------------------------------------------------------------------------------------

\section{Graph Cuts}
\label{sec:GraphCuts}

%----------------------------------------------------------------------------------------
%	SUBSECTION 1
%----------------------------------------------------------------------------------------

\subsection{Network Theory and the Min-cut Problem}
\label{sec:NetworkTheory}

In this section we cover Graph Theory and specifically Flow Networks, which is a branch of Graph Theory, which is fundamental to the understanding of image segmentation via graph cuts. With it roots in Germany where Euler tried to find the solution to the Konigsberg bridge problem, graph theory has since blossomed into a rich field of Mathematics with seemingly endless amounts of application. Graph Theory is a huge topic in mathematics and can be applied to many other sciences. Graph theory is part of another more encompassing field of Mathematics known as Combinatronics. Graph theory and applications are more useful than the average person would recognise. They're used in Google Maps to find shortest routes to destinations, in Molecular Chemistry to model the structure of atoms, and the list goes on for quite a while. It is no surprise that it is also found to be useful in image segmentation.

\begin{definition}[Network]
	A network $N = (V,E)$ is a directed graph with a source node $s$, a sink node $t$ and a strictly positive capacity on every edge. That is, for each edge $e \in E$, the capacity, $c(.)$, obeys $c(e) \in \Re^{+}$.
\end{definition}

\tikzstyle{vertex}=[circle,thick,draw]
\tikzstyle{edge} = [draw=black!24, very thick,->]
\tikzstyle{weight} = [font=\small]
\begin{figure}[!h]
	\centering
	\resizebox {\columnwidth} {!} {
		\begin{tikzpicture}[scale=1.5, auto, swap, background rectangle/.style={fill=blue!10}, show background rectangle, >={Stealth[black!24]}]
		\draw[black] node at (-0.2,1.5) [font=\large,right,rounded corners,inner sep=1ex] {$\textbf{N}$};
		
		\draw node[circle,thick,right,fill=red!20,text=red!20] at (-0.2,0) {$s$};
		\draw node[circle,thick,right,fill=red!20,text=red!20] at (5.8,0) {$t$};
		\draw[dashed,draw=red!24,very thick] (0.0,0) -- (0.5,-1.5);
		\draw[dashed,draw=red!24,very thick] (6.0,0) -- (6.5,-1.5);
		\draw[dashed,draw=red!24,very thick] (2.65,1.25) -- (4,1.5);
		
		% draw the vertices
		\foreach \pos/\name in {{(0,0)/s},{(2,1)/a},{(4,1)/b},{(2,-1)/c},{(4,-1)/d}, {(6,0)/t}}
		\node[vertex] (\name) at \pos {$\name$};
		% connect vertices with edges and draw weights
		\foreach \source/ \dest /\weight in {s/a/7,s/c/2,a/b/{},c/d/2,b/t/5,d/t/3}
		\path[edge] (\source) edge node[weight] {$\weight$} (\dest);
		% bends
		\path [draw=black!24, very thick,->] (a) edge[bend left=-30] node {$3$} (d);
		\path [draw=black!24, very thick,->] (c) edge[bend left=30] node {$3$} (b);
		% info box		
		\draw node[circle,right,fill=red!20] at (2.65,1.25) {$5$};
		\draw node[right,rounded corners,fill=red!20,inner sep=1ex] at (4,1.5){$capacity$};
		\draw node[right,rounded corners,fill=red!20,inner sep=1ex] at (0,-1.5){$source$};
		\draw node[right,rounded corners,fill=red!20,inner sep=1ex] at (6,-1.5){$sink$};
		% s	
		\draw[orange] node at (-0.2,0.4) [font=\footnotesize,right,rounded corners,inner sep=1ex] {$d_{in}=0$};
		\draw[orange] node at (-0.2,0.7) [font=\footnotesize,right,rounded corners,inner sep=1ex] {$d_{out}=2$};	
		% t
		\draw[orange] node at (5.8,0.4) [font=\footnotesize,right,rounded corners,inner sep=1ex] {$d_{in}=2$};
		\draw[orange] node at (5.8,0.7) [font=\footnotesize,right,rounded corners,inner sep=1ex] {$d_{out}=0$};
		\end{tikzpicture}
	}
	\caption{Network \textbf{N} with no flow. The in-degree and out-degree for the source, \textbf{s}, and the sink, \textbf{t}, are shown next to the corresponding node.}
\end{figure}

The \textbf{source node} only has out-going edges, $d_{in}(s) = 0$ and $d_{out}(s) \geq 0$. The \textbf{sink node} only has incoming edges, $d_{in} \geq 0$ and $d_out = 0$.

\begin{definition}[Flow]
	A flow $f : V^2 \longrightarrow \Re^{+}$ is associated with each edge $e = (u,v)$ such that:
	\begin{enumerate}
		\item for each edge $e \in E$ we have $0 \leq f(e) \leq c(e)$. That is, the flow is positive and cannot excees the capacity of the edge.
		
		\item for each intermediate node $v \in V\setminus \{s,t\}$ the in- and out-flow of that node $\sum_{u \in V^-(v)} f(u,v) = \sum_{u \in V^+(v)} f(v,u)$.
	\end{enumerate}
\end{definition}

The \textbf{total flow} $F$ of a network is then what leave the source $s$ or reaches the sink $t$:
\begin{equation}
F(N) := \sum_{u \in V} f(s,u) - \sum_{u \in V}f(u,s) = \sum_{u \in V} f(u,t) - \sum_{u \in V}f(t,u)
\end{equation}

\tikzstyle{vertex}=[circle,thick,draw]
\tikzstyle{edge} = [draw=black!24, very thick,->]
\tikzstyle{weight} = [font=\small]
\begin{figure}[!h]
	\centering
	\resizebox {\columnwidth} {!} {
		\begin{tikzpicture}[scale=1.5, auto, swap, background rectangle/.style={fill=blue!10}, show background rectangle, >={Stealth[black!24]}]
		\draw[black] node at (-0.2,1.5) [font=\large,right,rounded corners,inner sep=1ex] {$\textbf{N}$};
		\draw node[circle,thick,right,fill=red!20,text=red!20] at (-0.2,0) {$s$};
		\draw node[circle,thick,right,fill=red!20,text=red!20] at (5.8,0) {$t$};
		\draw[dashed,draw=red!24,very thick] (2.65,1.25) -- (4,1.5);
		\draw[dashed,draw=red!24,very thick] (0.0,0) -- (0.5,-1.5);
		\draw[dashed,draw=red!24,very thick] (6.0,0) -- (6.5,-1.5);
		% draw the vertices
		\foreach \pos/\name in {{(0,0)/s},{(2,1)/a},{(4,1)/b},{(2,-1)/c},{(4,-1)/d}, {(6,0)/t}}
		\node[vertex] (\name) at \pos {$\name$};
		% connect vertices with edges and draw weights
		\foreach \source/ \dest /\weight in {s/a/{5/7},s/c/{2/2},a/b/{},c/d/{1/2},b/t/{4/5},d/t/{3/3}}
		\path[edge] (\source) edge node[weight] {$\weight$} (\dest);
		% bends
		\path [draw=black!24, very thick,->] (a) edge[bend left=-30] node {$1/3$} (d);
		\path [draw=black!24, very thick,->] (c) edge[bend left=30] node {$2/3$} (b);
		% info box		
		\draw node[rounded corners,right,fill=red!20] at (2.65,1.25) {$3/5$};
		\draw node[right,rounded corners,fill=red!20,inner sep=1ex] at (4,1.5){$flow/capacity$};
		% s	
		\draw node[right,rounded corners,fill=red!20,inner sep=1ex] at (0,-1.5) [font=\footnotesize,right,rounded corners,inner sep=1ex] {$\sum\limits_{u \in V_N} f(s,u)=7$};
		% t
		\draw node[right,rounded corners,fill=red!20,inner sep=1ex] at (6,-1.5) [font=\footnotesize,right,rounded corners,inner sep=1ex] {$\sum\limits_{u \in V_N} f(u,t)=7$};
		\end{tikzpicture}
	}
	\caption{Network \textbf{N} with flow. The flow out of the source node, \textbf{s}, is equal to the flow into the sink node, \textbf{t}. For all other nodes, the flow-in is equal to the flow-out. This is the conservation of flow principle. This is only part of the network. The remaining part is the residual graph which shows the amount of reverse flow is available on an edge.}
\end{figure}

\begin{definition}[Cut]
	A cut of a network $N = (V,E)$ is a partitioning of the vertex set $V = P \bigcup \bar{P}$ into two disjoint sets $P$ containining the source node $s$ and $\bar{P}$ containing the sink node $t$. $P \bigcap \bar{P} = \emptyset$.
\end{definition}

\tikzstyle{vertex}=[circle,thick,draw]
\tikzstyle{edge} = [draw=black!24, very thick,->]
\tikzstyle{weight} = [font=\small]
\begin{figure}[!h]
	\centering
	\resizebox {\columnwidth} {!} {
		\begin{tikzpicture}[scale=1.5, auto, swap, background rectangle/.style={fill=blue!10}, show background rectangle, >={Stealth[black!24]}]
		\draw[black] node at (-0.2,1.5) [font=\large,right,rounded corners,inner sep=1ex] {$\textbf{N}$};
		\path[fill=red!20,,use Hobby shortcut,closed=true] (-0.3,-0.1) .. (1,1) .. (2.3,1.2) .. (1.3,0.2);
		%		\draw node[text=blue] at (-0.3,0) {$1$};
		%		\draw node[text=blue] at (1,1) {$2$};
		%		\draw node[text=blue] at (2.5,1.2) {$3$};
		%		\draw node[text=blue] at (1,0.0) {$4$};
		\path[fill=blue!20,,use Hobby shortcut,closed=true] (1.5,-1) .. (2.5,0.5) .. (3.5,1.2) .. (5.2,1.0) .. (6.3,0.0) .. (5.0,-1);
		%		\draw node[text=blue] at (1,-1) {$1$};
		%		\draw node[text=blue] at (3.2,1.2) {$2$};
		%		\draw node[text=blue] at (5.0,1.2) {$3$};
		%		\draw node[text=blue] at (6.5,0.0) {$4$};
		% draw the vertices
		\foreach \pos/\name in {{(0,0)/s},{(2,1)/a},{(4,1)/b},{(2,-1)/c},{(4,-1)/d}, {(6,0)/t}}
		\node[vertex] (\name) at \pos {$\name$};
		% connect vertices with edges and draw weights
		\foreach \source/ \dest /\weight in {s/a/{},s/c/{},a/b/{},c/d/{},b/t/{},d/t/{}}
		{
			\path[edge] (\source) edge node[weight] {$\weight$} (\dest);
		}		
		% bends
		\path [draw=black!24,very thick,->,name path=curve1] (a) edge[bend left=-30] node {} (d);
		\path [draw=black!24,very thick,->,name path=curve2] (c) edge[bend left=30] node {} (b);
		% cut
		\draw node [text=orange] at (0,-0.7) {$C$};
		\path[dashed, very thick, draw=orange, name path=C] (0,-1) -- (3.5,1.5);
		% intersections
		\path [draw=black!24,name path=line1] (s) -- (c);		
		\path [draw=black!24,name path=line2] (a) -- (b);		
		\path [draw=blue!24,name path=line3] (a) edge [bend left=-30] (d);
		\fill [color=orange, name intersections={of=C and line1}] (intersection-1) circle (2pt);
		\fill [color=orange, name intersections={of=C and line2}] (intersection-1) circle (2pt);
		\fill [color=orange, name intersections={of=C and line3}] (2.15,0.53) circle (2pt);	
		% info box
		\draw[yshift=-2.1cm,xshift=-0.5cm]
		node [right,rounded corners,fill=orange!20,inner sep=1ex]
		{
			$S = \{s,a\}$, \, $T = \{t,b,c,d\}$, \,$C = \{(sc), (ab), (ad)\}$
		};
		\end{tikzpicture}
	}
	\caption{Network \textbf{N} with with a valid cut \textbf{C}. The nodes within the red region are reachable from the source and the nodes within the blue region are able to reach the sink. The cut set, \textbf{C}, is show in the orange filled block.}
\end{figure}

\tikzstyle{vertex}=[circle,thick,draw]
\tikzstyle{edge} = [draw=black!24, very thick,->]
\tikzstyle{weight} = [font=\small]
\begin{figure}[!h]
	\centering
	\resizebox {\columnwidth} {!} {
		\begin{tikzpicture}[scale=1.5, auto, swap, background rectangle/.style={fill=blue!10}, show background rectangle, >={Stealth[black!24]}]
		\draw[black] node at (-0.2,1.5) [font=\large,right,rounded corners,inner sep=1ex] {$\textbf{N}$};
		\path[fill=red!20,,use Hobby shortcut,closed=true] (-0.3,-0.1) .. (2,1.5) .. (4,1.5) .. (6.4,-0.1) .. (4,0) .. (2,0) ;
		%				\draw node[text=blue] at (-0.3,0) {$1$};
		%				\draw node[text=blue] at (2,1.5) {$2$};
		%				\draw node[text=blue] at (4,1.5) {$3$};
		%				\draw node[text=blue] at (6.3,0.0) {$4$};
		%				\draw node[text=blue] at (4,0.0) {$5$};
		%				\draw node[text=blue] at (2,0.0) {$6$};
		\path[fill=blue!20,,use Hobby shortcut,closed=true] (1.5,-1) .. (3.0,-0.7) .. (4.5,-1.0) .. (3.0,-1.2);
		%				\draw node[text=blue] at (1.5,-1) {$1$};
		%				\draw node[text=blue] at (3.0,-0.7) {$2$};
		%				\draw node[text=blue] at (4.5,-1.0) {$3$};
		%				\draw node[text=blue] at (3.0,-1.2) {$4$};			
		% draw the vertices
		\foreach \pos/\name in {{(0,0)/s},{(2,1)/a},{(4,1)/b},{(2,-1)/c},{(4,-1)/d}, {(6,0)/t}}
		\node[vertex] (\name) at \pos {$\name$};
		% connect vertices with edges and draw weights
		\foreach \source/ \dest /\weight in {s/a/{},s/c/{},a/b/{},c/d/{},b/t/{},d/t/{}}
		{
			\path[edge] (\source) edge node[weight] {$\weight$} (\dest);
		}		
		% bends
		\path [draw=black!24,very thick,->,name path=curve1] (a) edge[bend left=-30] node {} (d);
		\path [draw=black!24,very thick,->,name path=curve2] (c) edge[bend left=30] node {} (b);
		% cut
		\draw node [text=orange] at (0,-0.7) {$C$};
		\path[dashed, very thick, draw=orange, name path=C] (0,-1) edge[bend left=30] (6,-1);
		% intersections
		\fill [color=orange] (1.0,-0.5) circle (2pt);
		\fill [color=orange] (5.0,-0.5) circle (2pt);
		\fill [color=orange] (2.28,-0.18) circle (2pt);
		\fill [color=orange] (2.52,-0.165) circle (2pt);
		% info box
		%		\draw[yshift=-2.1cm,xshift=-0.5cm]
		%		node [right,rounded corners,fill=orange!20,inner sep=1ex]
		%		{
		%			$S = \{s,a\}$, \, $T = \{t,b,c,d\}$, \,$C = \{(sc), (ab), (ad)\}$
		%		};
		\end{tikzpicture}
	}
	\caption{Network \textbf{N} with with a invalid cut \textbf{C}. The cut does not partition source node \textbf{s} and sink node \textbf{t} into distinct sets.}
\end{figure}

\tikzstyle{vertex}=[circle,thick,draw]
\tikzstyle{edge} = [draw=black!24, very thick,->]
\tikzstyle{weight} = [font=\small]
\begin{figure}[!h]
	\centering
	\resizebox {\columnwidth} {!} {
		\begin{tikzpicture}[scale=1.5, auto, swap, background rectangle/.style={fill=blue!10}, show background rectangle, >={Stealth[black!24]}]
		\draw[black] node at (-0.2,1.5) [font=\large,right,rounded corners,inner sep=1ex] {$\textbf{N}$};
		%		\path[fill=red!20,,use Hobby shortcut,closed=true] (-0.3,-0.1) .. (2,1.5) .. (4,1.5) .. (6.4,-0.1) .. (4,0) .. (2,0) ;
		%				\draw node[text=blue] at (-0.3,0) {$1$};
		%				\draw node[text=blue] at (2,1.5) {$2$};
		%				\draw node[text=blue] at (4,1.5) {$3$};
		%				\draw node[text=blue] at (6.3,0.0) {$4$};
		%				\draw node[text=blue] at (4,0.0) {$5$};
		%				\draw node[text=blue] at (2,0.0) {$6$};
		%		\path[fill=blue!20,,use Hobby shortcut,closed=true] (1.5,-1) .. (3.0,-0.7) .. (4.5,-1.0) .. (3.0,-1.2);
		%				\draw node[text=blue] at (1.5,-1) {$1$};
		%				\draw node[text=blue] at (3.0,-0.7) {$2$};
		%				\draw node[text=blue] at (4.5,-1.0) {$3$};
		%				\draw node[text=blue] at (3.0,-1.2) {$4$};			
		% draw the vertices
		\foreach \pos/\name in {{(0,0)/s},{(2,1)/a},{(4,1)/b},{(2,-1)/c},{(4,-1)/d}, {(6,0)/t}}
		\node[vertex] (\name) at \pos {$\name$};
		% connect vertices with edges and draw weights
		\foreach \source/ \dest /\weight in {s/a/{},s/c/{},a/b/{},c/d/{},b/t/{},d/t/{}}
		{
			\path[edge] (\source) edge node[weight] {$\weight$} (\dest);
		}		
		% bends
		\path [draw=black!24,very thick,->,name path=curve1] (a) edge[bend left=-30] node {} (d);
		\path [draw=black!24,very thick,->,name path=curve2] (c) edge[bend left=30] node {} (b);
		% cut
		\draw node [text=orange] at (0,-0.7) {$C$};
		\draw [dashed, orange, very thick] plot [smooth] coordinates {(0,-1) (2.5,1) (3,1.2) (3.5,-1.5)};
		% intersections
		\fill [color=orange] (0.77,-0.37) circle (2pt);
		\fill [color=orange] (2.09,0.68) circle (2pt);
		\fill [color=orange] (2.5,1) circle (2pt);
		\fill [color=orange] (3.05,1) circle (2pt);
		\fill [color=orange] (3.12,0.7) circle (2pt);
		\fill [color=orange] (3.40,-0.82) circle (2pt);
		\fill [color=orange] (3.42,-1) circle (2pt);
		% info box
		%		\draw[yshift=-2.1cm,xshift=-0.5cm]
		%		node [right,rounded corners,fill=orange!20,inner sep=1ex]
		%		{
		%			$S = \{s,a\}$, \, $T = \{t,b,c,d\}$, \,$C = \{(sc), (ab), (ad)\}$
		%		};
		\end{tikzpicture}
	}
	\caption{Network \textbf{N} with with a invalid cut \textbf{C}. The cut partition partitions the graph into more than two sets and the cut intersects the edges $ab$ and $ad$ twice.}
\end{figure}

The \textbf{capacity} of a cut is the sum of the edges $(u,v) \in V$ where $u \in P$ and $v \in \bar{P}$:
\begin{equation}
\kappa (P, \bar{P}) = \sum_{u \in P; v \in \bar{P}} c(u,v)
\end{equation}

\begin{definition}[Maximal Flow]
	The largest amount of flow that can be sent through the source that is able to reach the sink is known as the maximal flow.
\end{definition}

\tikzstyle{vertex}=[circle,thick,draw]
\tikzstyle{edge} = [draw=black!24, very thick,->]
\tikzstyle{weight} = [font=\small]
\begin{figure}[!h]
	\centering
	\resizebox {\columnwidth} {!} {
		\begin{tikzpicture}[scale=1.5, auto, swap, background rectangle/.style={fill=blue!10}, show background rectangle, >={Stealth[black!24]}]
		\draw[black] node at (-0.2,1.5) [font=\large,right,rounded corners,inner sep=1ex] {$\textbf{N}$};
		\draw node[circle,thick,right,fill=red!20,text=red!20] at (-0.2,0) {$s$};
		\draw node[circle,thick,right,fill=red!20,text=red!20] at (5.8,0) {$t$};
		\draw[dashed,draw=red!24,very thick] (0.0,0) -- (0.5,-1.5);
		\draw[dashed,draw=red!24,very thick] (6.0,0) -- (6.5,-1.5);
		% draw the vertices
		\foreach \pos/\name in {{(0,0)/s},{(2,1)/a},{(4,1)/b},{(2,-1)/c},{(4,-1)/d}, {(6,0)/t}}
		\node[vertex] (\name) at \pos {$\name$};
		% connect vertices with edges and draw weights
		\foreach \source/ \dest /\weight in {s/a/{6/7},s/c/{2/2},a/b/{5/5},c/d/{2/2},b/t/{5/5},d/t/{3/3}}
		\path[edge] (\source) edge node[weight] {$\weight$} (\dest);
		% bends
		\path [draw=black!24, very thick,->] (a) edge[bend left=-30] node {$0/3$} (d);
		\path [draw=black!24, very thick,->] (c) edge[bend left=30] node {$1/3$} (b);
		%info
		% s	
		\draw node[right,rounded corners,fill=red!20,inner sep=1ex] at (0,-1.5) [font=\footnotesize,right,rounded corners,inner sep=1ex] {$\sum\limits_{u \in V_N} f(s,u)=8$};
		% t
		\draw node[right,rounded corners,fill=red!20,inner sep=1ex] at (6,-1.5) [font=\footnotesize,right,rounded corners,inner sep=1ex] {$\sum\limits_{u \in V_N} f(u,t)=8$};
		\end{tikzpicture}
	}
	\caption{Network \textbf{N} with maximum flow. There is no way to push more flow out of the source into the sink without breaking the rules for conservation of flow.}
\end{figure}

\begin{definition}[Minimal Cut]
	A cut $C$ on a network $N = (V,E)$ is a minimal cut if there exists no other cut $C'$ where $\kappa (C') < \kappa(C)$.
\end{definition}

\tikzstyle{vertex}=[circle,thick,draw]
\tikzstyle{edge} = [draw=black!24,very thick,->]
\tikzstyle{weight} = [font=\small]
\begin{figure}[!h]
	\centering
	\resizebox {\columnwidth} {!} {
		\begin{tikzpicture}[scale=1.5, auto, swap, background rectangle/.style={fill=blue!10}, show background rectangle, >={Stealth[black!24]}]
		\draw[black] node at (-0.2,1.5) [font=\large,right,rounded corners,inner sep=1ex] {$\textbf{N}$};	
		\path[fill=red!20,,use Hobby shortcut,closed=true] (-0.3,-0.1) .. (2,1.5) .. (4,1.5) .. (4.5,-0.1) .. (4,-1.5) .. (2,-1.5) ;
		%			\draw node[text=blue] at (-0.3,0) {$1$};
		%			\draw node[text=blue] at (2,1.5) {$2$};
		%			\draw node[text=blue] at (4,1.5) {$3$};
		%			\draw node[text=blue] at (6.3,0.0) {$4$};
		%			\draw node[text=blue] at (4,-1.5) {$5$};
		%			\draw node[text=blue] at (2,-1.5) {$6$};
		\path[fill=blue!20,use Hobby shortcut,closed=true] (5.5,0) .. (6.0,0.5) .. (6.5,0) .. (6.0,-0.5);
		%			\draw node[text=blue] at (5.5,0) {$1$};
		%			\draw node[text=blue] at (6.0,0.5) {$2$};
		%			\draw node[text=blue] at (6.5,0) {$3$};
		%			\draw node[text=blue] at (6.0,-0.5) {$4$};
		% draw the vertices
		\foreach \pos/\name in {{(0,0)/s},{(2,1)/a},{(4,1)/b},{(2,-1)/c},{(4,-1)/d}, {(6,0)/t}}
		\node[vertex] (\name) at \pos {$\name$};
		% connect vertices with edges and draw weights
		\foreach \source/ \dest /\weight in {s/a/7,s/c/2,a/b/{5},c/d/2,b/t/5,d/t/3}
		\path[edge] (\source) edge node[weight] {$\weight$} (\dest);
		% bends
		\path [draw=black!24, very thick,->] (a) edge[bend left=-30] node {$3$} (d);
		\path [draw=black!24, very thick,->] (c) edge[bend left=30] node {$3$} (b);
		% cut
		\draw node [text=orange] at (4.8,-1.0) {$C$};
		\path[dashed, very thick, draw=orange, name path=C] (5,-1.2) -- (5,1.2);
		% intersections
		\path [draw=black!24,name path=line1] (b) -- (t);
		\path [draw=black!24,name path=line2] (d) -- (t);
		\fill [color=orange, name intersections={of=C and line1}] (intersection-1) circle (2pt);
		\fill [color=orange, name intersections={of=C and line2}] (intersection-1) circle (2pt);
		%info
		% minimum capacity	
		\draw node[right,rounded corners,fill=orange!20,inner sep=1ex] at (5,-1.5) [font=\footnotesize,right,rounded corners,inner sep=1ex] {$\sum\limits_{e \in C} c(e)=8$};
		\end{tikzpicture}
	}
	\caption{Network \textbf{N} with minimal cut \textbf{C}. The sum of the capacity of all the edges in the cut set is the minimum of all possible valid cuts on the network \textbf{N}.}
\end{figure}

In the next section we show that the Maximal Flow problem and the Minimal Cut problem are duals of each other, commonly known as the Max-Flow/Min-Cut problem.

%----------------------------------------------------------------------------------------
%	SUBSECTION 2
%----------------------------------------------------------------------------------------

\subsection{Image Segmentation Framework/Graph Construction Images}
\label{sec:GraphCutFramework}

%----------------------------------------------------------------------------------------
%	SECTION 4
%----------------------------------------------------------------------------------------

\section{Energy Minimisation}
\label{sec:EnergyMinimisation}

%----------------------------------------------------------------------------------------
%	SUBSECTION 1
%----------------------------------------------------------------------------------------

\subsection{General Form of Energy Functions}
\label{sec:GeneralEnergyFunctions}

%----------------------------------------------------------------------------------------
%	SUBSECTION 2
%----------------------------------------------------------------------------------------

\subsection{Submodular Functions}
\label{sec:Submodular Functions}

%----------------------------------------------------------------------------------------
%	SECTION 3
%----------------------------------------------------------------------------------------

\subsection{Energy Minimisation Algorithms}
\label{sec:MaxFlowMinCutAlgoithms}

Lorem ipsum dolor sit amet, consectetur adipiscing elit. Aliquam ultricies lacinia euismod. Nam tempus risus in dolor rhoncus in interdum enim tincidunt. Donec vel nunc neque. In condimentum ullamcorper quam non consequat. Fusce sagittis tempor feugiat. Fusce magna erat, molestie eu convallis ut, tempus sed arcu. Quisque molestie, ante a tincidunt ullamcorper, sapien enim dignissim lacus, in semper nibh erat lobortis purus. Integer dapibus ligula ac risus convallis pellentesque.

%-----------------------------------
%	SUBSECTION 1
%-----------------------------------
\subsubsection{Ford-Fulkerson}

Nunc posuere quam at lectus tristique eu ultrices augue venenatis. Vestibulum ante ipsum primis in faucibus orci luctus et ultrices posuere cubilia Curae; Aliquam erat volutpat. Vivamus sodales tortor eget quam adipiscing in vulputate ante ullamcorper. Sed eros ante, lacinia et sollicitudin et, aliquam sit amet augue. In hac habitasse platea dictumst.

\begin{algorithm}
	\caption{Euclid’s algorithm}\label{alg:euclid}
	\begin{algorithmic}[1]
		\Procedure{Euclid}{$a,b$}\Comment{The g.c.d. of a and b}
		\State $r\gets a\bmod b$
		\While{$r\not=0$}\Comment{We have the answer if r is 0}
		\State $a\gets b$
		\State $b\gets r$
		\State $r\gets a\bmod b$
		\EndWhile\label{euclidendwhile}
		\State \textbf{return} $b$\Comment{The gcd is b}
		\EndProcedure
	\end{algorithmic}
\end{algorithm}

%-----------------------------------
%	SUBSECTION 2
%-----------------------------------

\subsubsection{Dinic/Edmond-Karp}
Morbi rutrum odio eget arcu adipiscing sodales. Aenean et purus a est pulvinar pellentesque. Cras in elit neque, quis varius elit. Phasellus fringilla, nibh eu tempus venenatis, dolor elit posuere quam, quis adipiscing urna leo nec orci. Sed nec nulla auctor odio aliquet consequat. Ut nec nulla in ante ullamcorper aliquam at sed dolor. Phasellus fermentum magna in augue gravida cursus. Cras sed pretium lorem. Pellentesque eget ornare odio. Proin accumsan, massa viverra cursus pharetra, ipsum nisi lobortis velit, a malesuada dolor lorem eu neque.

%-----------------------------------
%	SUBSECTION 3
%-----------------------------------

\subsubsection{Push-Relabel}
Originally developed by Andrew V. Goldberg and Robert E. Tarjan. Previous algorithms, such as Ford-Fulkerson, used the concept of residual networks and augmenting paths to determine max-flow.
Push-Relabel used the concept of preflow to determine  max-flow instead of augmenting paths. Sometimes referred as the Preflow-Push Algorithm.
Preflow is a concept originally developed by A.V. Karzanov.

The algorithm works at converting a preflow, $f$, into a normal flow and then terminates. This flow also turns out to be the maximum flow. Goldberg and Tarjan defined a generic Push-Relabel algorithm  which solves the maximum flow problem.

\begin{definition}[Preflow]
	A preflow is a real-valued function, $f$, on vertice pairs. The total flow into a vertex can exceed the flow out of a vertex but not vice versa.
\end{definition}
A preflow where all $v \in V-\{s, t\}$ has a flow excess of zero, $e_f(v) = 0$, is a normal flow. The preflow function is also referred to as the \textbf{s-t preflow}.

Preflow must satisfy:
\begin{enumerate}
	\item Capacity Constraint\\
	$\forall u,v \in V, f(u,v) \leq c(u,v)$
	
	\item Antisymmetry/Skew Symmetry\\
	$\forall u, v, \in V, f(u,v) = -f(v,u)$
	
	\item Nonnegative Constrain\\
	The flow into $v \in V-\{s\}$ must be greater than or equal to the flow out of $v$. $\forall u \in V, v \in V-\{s\}, \sum f(u,v)>0$
\end{enumerate}

\begin{definition}[Flow Excess]
	Flow excess, $e_f(v)$, is the net flow into $v$ where $v \in V$ for some preflow $f$.
\end{definition}

\[
e_f(v) =
\begin{cases} 
\hfill \infty \hfill & \text{ if $v=s$} \\
\hfill \sum_{u \in V}f(u,v) \hfill & \text{ if $v \in V-\{s\}$} \\
\end{cases}
\]

\begin{definition}[Active Vertex]
	An active vertex/node is a vertex $v$ which satisfies all of the properties:
	\begin{enumerate}
		\item Not a source or sink, $v \in V-\{s,t\}$
		\item Positive flow excess, $e_f(v) > 0$
		\item Has a valid label, $d(v) < \infty$
	\end{enumerate}
\end{definition}

Push-Relabel also uses the concept of a residual graph, $G_f=(V, E_f)$.

\begin{definition}[Residual Capacity]
	The residual capacity of a preflow is defined as $r_f(v,w) = c(v,w)-f(v,w)$.
\end{definition}

\begin{definition}[Residual Edges]
	The residual edges for a preflow $f$ is defined as the set of edges with positive residual capacity. $E_f = \{(v,w)\} | r-f(v,w) > 0$.
\end{definition}

\begin{definition}[Labelling]
	Push-Relabel also use a valid labelling function, $d$, to determine which vertex pairs should be selected for the push operation.
\end{definition}
A valid labelling , $d$, is a nonnegative integer function applied to all vertices to denote a label. The labelling is often referred as the height or distance from the sink node, $t$. This function is sometimes compared to the physical intuition that liquids naturally flow downhill.

A valid labelling for a preflow consists of:
\begin{enumerate}
	\item For $v \in V, 0 \leq d(v) \leq \infty$
	\item $d(s) = |V| \text{ (source condition)}$
	\item $d(t) = 0 \text{ (sink condition)}$
	\item $d(v) = d(w) + 1$ for every residual edge $(v,u) \in E_f$
\end{enumerate}
A labelling $d$ and a preflow $f$ are said to be compatible id $d$ adheres to the properties above.

The algorithm pushes flow excess starting at the source, $s$, along all vertices towards the sink, $t$. The algorithm maintains a compatible vertex labelling function, $d$, to the preflow, $f$. The labelling is usedto determine where to puch the flow excess. The algorithm repeatedly performs either a push or a relabel operation so long as there is an active vertex in $G_f$.

\begin{definition}[Push Operation]
	The push operation is used to move flow from one vertex to another. The transfer of excess can be performed across the vertex pair $(v,w) \in E_f$ if:
	\begin{enumerate}
		\item $v$ is an active vertex
		\item the edge has positive residual capacity, $r_f(v,w)>0$
		\item the label distance $d(v) = d(w)+1$
	\end{enumerate}
\end{definition}
This allows the algorithm to move $\delta$ excess flow: $\delta = min (e_f(v), r_f(v,w))$ from $v$ to $w$. A push is considered \textbf{saturating} if no more flow can be sent over the edge, $\delta = r_f(v,w)$. A push is considered to be \textbf{non-saturating} if all the excess from $v$ the push over the dge and the edge still has some cpacity, $\delta = e_f(v)$.

\begin{algorithm}
	\caption{Push Operation}\label{alg:push}
	\textbf{Input:} Preflow $f$, labels $d$, and $(v,w)$ where $v,w \in V$\\
	\textbf{Output:} Preflow $f$\\
	\textbf{Applicable:} if $v \in V-\{s,t\}$, $d(v) < \infty$, $e_f(v)>0$, $r_f(v,w)>0$ and $d(v)=d(w)+1$
	\begin{algorithmic}[1]
		\Procedure{Push}{$v,w$}
		\State $\delta \coloneqq min(e_f(v), r_f(v,w)$
		\State $f(v,w) \coloneqq f(v,w) + \delta$
		\State $f(w,v) \coloneqq f(w,v) - \delta$
		\State $e_f(v) \coloneqq e_f(v) - \delta$
		\State $e_f(w) \coloneqq e_f(w) + \delta$
		\State \textbf{return} $f$
		\EndProcedure
	\end{algorithmic}
\end{algorithm}
	
\begin{definition}[Relabel Operation]
	The relabel operation is used to increase the label value of a single active vertex so that excess flow can be pushed out of the active vertex. The relabel operation is performed when all the residual edges of the active vertex have positive residual capacity, $r_f(v,w)>0$. This implies that $v$'s label is less than or equal to all vertices, $d(v) \leq d(w)$, meaning that no push operation across the edges is possible given the push condition $d(v) = d(w)+1$.
\end{definition}
The relabel operation for some vertex $v$ selects the smallest label for the vertices with positive residual edges, $r_f(v,w)>0$. The active vertex is then assigned the smallest label value $+1$ such that $d(v) := min{d(v)+1 | (v,w) \in E_f}$. This will alow the vertex $v$ to potentially push its excess flow to atleast one of the othe vertices during the algorithm's next iteration.

\begin{algorithm}
	\caption{Relabel Operation}\label{alg:relabel}
	\textbf{Input:} Preflow $f$, labels $d$, and $v \in V-\{s,t\}$\\
	\textbf{Output:} Labels $d$\\
	\textbf{Applicable:} if $v \in V-\{s,t\}$, $d(v) < \infty$, $e_f(v)>0$, and $\forall w \in V, r_f(v,w)>0$ which implies $d(v) \leq d(w)$
	\begin{algorithmic}[1]
		\Procedure{Relabel}{$v$}
		\If {$\{(v,w) \in E_f\} \neq 0$}
		\State $d(v) \coloneqq min(d(w)+1 | (v,w) \in E_f)$
		\Else 
		\State $d(v) \coloneqq \infty$
		\EndIf
		\State \textbf{return} $d$
		\EndProcedure
	\end{algorithmic}
\end{algorithm}

The algorithm initialises the following values in the residual graph before the push andrelabel operations in the main loop.
\begin{enumerate}
	\item Initialise the preflow of all edges in the  residual graph
	
	\item Initialise the labellings such that:
	\begin{enumerate}
		\item $d(s) = |V|$
		\item $d(v) = 0 \text{for } v \in V-\{s\}$
	\end{enumerate}
	
	\item Performs saturation, pushes along all residual edges out of the source $(s,v) \in E_f$ and $v \in V$.
\end{enumerate}
Once complete the algorithm repeatedly performs either a push or a relabel operation against all vertices. The algorithm continues until no operation can be performed. The algorithm terminates when there are no more active vertices.

\begin{algorithm}
	\caption{Push-Relabel Main-loop}\label{alg:main_no_optimisation}
	\textbf{Input:} Network flow graph $G=(V,E)$, $s$, $t$ and $c$\\
	\textbf{Output:} Maximum flow $f$
	\begin{algorithmic}[1]
		\Procedure{Main}{$v$}
		\ForAll {$(v,w) \in (V-\{s\})(V-\{s\})$}
		\State $f(v,w) \gets 0$
		\State $f(w,v) \gets 0$
		\EndFor
		\item[]
		\ForAll {$v \in V$}
		\State $f(s,v) \gets r_f(s,v)$
		\State $f(v,s) \gets -r_f(s,v)$
		\EndFor
		\item[]
		\State $d(s) \gets |V|$
		\item[]
		\ForAll {$v \in V-\{s\}$}
		\State $d(v) \gets 0$
		\State $e_f(v) \gets f(s,v)$
		\EndFor
		\item[]
		\State While there exists an active vertex
		\While {$\exists v \in V-\{s,t\}$} \Comment{with either applicable PUSH() or RELABEL() operation}
		\State Perform either a PUSH or a RELABEL operation on $v$
		\EndWhile
		\item[]
		\State \textbf{return} $f$
		\EndProcedure
	\end{algorithmic}
\end{algorithm}

The analaysis and the proof of correctness of the Push-Relabel algorithm can be found in \autoref{AppendixB}.

%-----------------------------------
%	SUBSUBSECTION 1
%-----------------------------------

\subsubsection{Push-Relabel Speed Optimisation Heuristics}
\begin{definition}[Discharge]
	Push-Relabel also use a valid labelling function ,$d$, to determine which vertex pairs should be selected for the push operation.
\end{definition}

\begin{definition}[FIFO]
	Push-Relabel also use a valid labelling function ,$d$, to determine which vertex pairs should be selected for the push operation.
\end{definition}

\begin{definition}[Highest Label First]
	Push-Relabel also use a valid labelling function ,$d$, to determine which vertex pairs should be selected for the push operation.
\end{definition}

\begin{definition}[Global Relabel]
	Push-Relabel also use a valid labelling function ,$d$, to determine which vertex pairs should be selected for the push operation.
\end{definition}

\begin{definition}[Gap Relabel]
	Push-Relabel also use a valid labelling function ,$d$, to determine which vertex pairs should be selected for the push operation.
\end{definition}
 % Mathematical Background
% Chapter Template

\chapter{Literature Review} % Main chapter title

\label{chap:Chapter10} % Change X to a consecutive number; for referencing this chapter elsewhere, use \ref{ChapterX}
 % Literature Review
% Chapter Template

\chapter{Pre-processing Scheme for Fluorescence Microscopy Images} % Main chapter title

\label{chap:Chapter4} % Change X to a consecutive number; for referencing this chapter elsewhere, use \ref{ChapterX}

%\textcolor{red}{There are many factors that degrade the quality of images. The knock-on effect is sub-optimal segmentation results. List the problems and show some images of them. What techniques exist to curb these negative effects? Although it is very common for preprocessing to be done before any further analysis, lack of a scheme. Where do we take this up from (after PSF deconvolution). Why is it non-trivial to design criteria for an ideal image to segment? What do we plan to accomplish in this section?}

%Fluorescence microscopy has become an indispensable tool \citep{Matula2012} in myriad of scientific disciplines such as chemistry, biology, neurobiology and medicine to name a few. It gives scientists visual access to physiological processes and other cellular and sub-cellular activities.
%Fluorescence microscopy coupled with the tremendous technological advancement of image acquisition has lead to an explosion of raw image data such that the sheer volume of images acquired presents too much of a burden on manual analysis \citep{Matula2012}.
%Consequently, there is a need for highly accurate automated analysis.
%In such image analysis, image segmentation has established itself as a key process in identifying and extracting objects of interest which can thereafter be used in higher level image analysis.
%The ever increasing demand on image segmentation is to extract the objects of interest with greater accuracy in a shorter time.

Often enough, the poor results of a segmentation are not because the segmentation method is not tuned properly, or flawed in some way, it is because the image quality is extremely poor which renders acuurate object extraction virtually impossible. 
The fluorescence image acquisition process is ridden with image degradation factors at almost every step.
Yet still, the ever-increasing demand on image segmentation is to extract the objects of interest with greater accuracy in a shorter time.
%There are many factors that degrade the quality of fluorescence images and these can result in sub-optimal segmentations.
%affect the accuracy of the segmentation.
%These factors have been studied independently to a great deal and have been successfully applied to real-world data.
Some of these issues are poor contrast, photo-bleaching, not black enough background, non-fluorescing samples, improper excitation, etc.

Literature is rich with techniques that are able to, a high degree, negate the effect of these factors \cite{Lysaker2004,Wang2008,Zhou2013}. Given the frequency of occurrence of these problems in fluorescence images, it is surprising that there is lack of definition of a scheme prior to segmentation that prepares an image such that accurate segmentation can be achieved.

The task of defining a set of criteria an image has to meet for reliable segmentation results is not a trivial one.
However, segmentation algorithms are desgined to work with certain image charateristics and we can design pre-processing schemes that enhance these characteristics such that  we would have a "better" image to segment. Enhancement of the original image can facilitate higher level analysis e.g. contrast enhancement.
%segmentation qualities. The criteria of this scheme is
In this chapter, we aim to design a hybrid algorithm that builds on highly efficient algorithms to produce better results. A typically fluorescence image segmentation scheme is shown in \Cref{fig:wholescheme}.
In this scheme, a proxy image is generated. This image has the enhanced characteristics required for accurate segmentation. In the next step, segmentation is performed on the proxy image which yields a segmentation mask. The segmentation mask oftens contains speckle and other faux segmetnation areas which are then removed in the clean-up stage.
Thereafter, depending on the next level of image analysis, just the final segmentation mask may be all that is required, otherwise, the mask is overlayed on the original image and sent through for high level analysis.

\begin{figure}[!h]
	\centering
	\includegraphics[width=1\columnwidth]{mypreprocess/whole_process}
	\caption{Segmentation Scheme.}
	\label{fig:wholescheme}
\end{figure}

The step that we are concerned with, in this chapter, is the Proxy Image Generation stage. Most segmentation schemes in fluorescence image segmentation would segment on intensity, therefore, a typical pre-processing scheme would emphasise the following main steps:

\begin{enumerate}
	\item Noise reduction
	\item Object data enhancement
	\item Edge completion and enhancement
	\item Reduction of intra-region variance
\end{enumerate}

We base our scheme on this framework. The step through process of the proposed scheme is illustrated in \Cref{fig:flowchartproposedscheme}. For multi-channel images, we first split the image into its channel components and process each on its own. It is at this stage that extraneous channels are discarded. Then noise removal is performed on the remaining channels. The next step involves object data enhancement by suppressing non-object data and amplifying object data. We then combine the channels into a single image and perform intra-region smoothing to reduce intensity variation.

\begin{figure}[!h]
	\centering
	%\includegraphics[width=0.25\columnwidth]{mypreprocess/proposed_scheme}
	\includegraphics[width=1\columnwidth]{mypreprocess/preprocess_flow}
	\caption{Proxy image generation.}
	\label{fig:flowchartproposedscheme}
\end{figure}

%----------------------------------------------------------------------------------------
%	SECTION 2
%----------------------------------------------------------------------------------------

\section{Pre-processing Scheme}
\label{sec:preprocessscheme}


%----------------------------------------------------------------------------------------
%	SUBSECTION 1
%----------------------------------------------------------------------------------------

\subsection{Denoising}
\label{sec:PoissonDenoising}

\begin{definition}[Removing useless channels]
	Many segmentation algorithms are designed to work on gray scale images. If we have a colour image, it is first converted to gray scale. Previously, each fluorescing sample was captured in a its own grayscale image. The final colour image was composed on a computer. Through the advancements in optical engineering it is now possible for fluorescence images to be obtained in colour, however not all channels add object data to the image.
	%They in fact degrade the image due to the noise in that channel. 
	Often enough, these redundant channels will just be "data-less and noisy", so eliminating this channel will often yield a higher quality image.
	
	For example, a direct conversion of the colour image in \ref{fig:originalgrayscale} to grayscale, shown in \ref{fig:averaginggrayscale}, has lesser brightness, lower contrast and more noise than the grayscale image in \ref{fig:specificgrayscale}, which discarded the data-less blue channel. This is because in a direct grayscale conversion, all channels are averaged to produce the final single-channel image. Hence, the redundant blue channel is suppressing important data.
	
	\begin{figure}[!t]
		\centering
		\subfigure[]
		{
			
			\includegraphics[width=0.31\columnwidth]{mypreprocess/2colr.jpg}
			\label{fig:originalgrayscale}
		}	
		\subfigure[]
		{
			\includegraphics[width=0.31\columnwidth]{cell_database_gray/2gray.jpg}
			\label{fig:averaginggrayscale}
		}
		\subfigure[]
		{
			\includegraphics[width=0.31\columnwidth]{mypreprocess/2gr.jpg}
			\label{fig:specificgrayscale}
		}
		\caption{Comparison of grayscale conversions. \textbf{(a)} Original image. \textbf{(b)} Direct grayscale conversion. \textbf{(c)} Grayscale conversion discarding the blue channel.}
		\label{fig:grayscaleconversion}
	\end{figure}
\end{definition}

\begin{definition}[Poisson Noise Reduction]
	In \Cref{sec:ImageProcessing} we mentioned that the primary form of noise in fluorescent images is Poisson noise. It is important to remove as much noise as possible without dampening the boundary information. To this end, we have used the Total-Variation anisotropic denoising (Bregman Split). Although it was developed to remove Gaussian noise, it was shown by Rodriguez \textit{et al.}\citep{Rodriguez2008} that it supersedes other state-of-the-art Poisson noise removal methods, e.g.  wavelets \citep{Timmermann1999}, platelets\citep{Willett2004}, minimum description length \citep{Nowak1999}, while maintaining signal integrity. 
	
	The Poisson distribution, which has equal mean and standard deviation i.e. $\mu = \sigma$, is defined by
	\begin{equation}
	P(n,\mu) = \frac{e^{-\mu}\mu^{n}}{n!}
	\label{eq:poissondist}
	\end{equation}
	
	Let $y = \lbrace y_i:i=1, \cdots, N \rbrace$ and $x = \lbrace x_i:i=1, \cdots, N \rbrace$ be the observed and the true image, respectively. The sample $y_i$ is a Poisson contaminated form of $x_i$. We desire to recover the signal $x$ from the observed signal $y$. From Bayes' Law, we get
	\begin{equation}
	P(x \vert y) = \frac{P(y \vert x)P(x)}{P(y)}
	\label{eq:bayeslaw}
	\end{equation}
	Therefore, we wish to find the maximum of $P(y \vert x)P(x)$. If all samples are affected by Poisson noise we have
	\begin{equation}
	P(y \vert x) = P(y,x) = \frac{e^{-x_i}x_i^{y_i}}{y_i!}
	\label{eq:poissonafect}
	\end{equation}
	Thus the likelihood of observing $y$ given the true image $x$ is given by
	\begin{equation}
	P(y \vert x) = \prod_{i=1}^{N} \frac{e^{-x_i}x_i^{y_i}}{y_i!}
	\label{eq:poissonlikelihood}
	\end{equation}
	
	In anisotropic TV denoising we wish to recover the original image given the noisy image by minimising the constrained problem
	
	\begin{equation}
	\underset{u} {\mathrm{min}} \left| \left| \frac{du}{dx} \right| \right|_1 + \left| \left| \frac{du}{dy} \right| \right|_1 + \frac{\gamma}{2} \left| \left| u-f \right| \right|^2_2
	\label{equ:anisotropic_tv_constrained}
	\end{equation}
	
	where $\gamma > 0$ is the regularisation parameter which affects the balance between noise removal and signal preservation \citep{Getreuer2012}, $u$ is the true image and $f$ is the noisy image. For computational efficiency issues, we actually solve the unconstrained problem
	
	\begin{equation}
	\underset{u, dx, dy} {\mathrm{min}} \left| \left|dx_1 \right| \right|_1 + \left| \left| dy_1 \right| \right|_1 + \frac{\gamma}{2} \left| \left| u-f \right| \right|^2_2 +
	\frac{\lambda}{2} \left| \left| dx-u_x \right| \right|^2_2 + \frac{\lambda}{2} \left| \left| dy-u_y \right| \right|^2_2
	\label{equ:anisotropic_tv_unconstrained}
	\end{equation}
	
	This can be solved using the Bregman Split algorithm \citep{Wei2010}.
	The algorithm runs iteratively until the error, $e = \frac{\vert u'-u \vert}{\vert u \vert}$,  is less than a user-defined tolerance factor, $\epsilon$, where $u'$ is the image obtained after denoising the input image $u$. 
	%We used  $\mu=20$, $\lambda=1$ and $\epsilon=1 \times 10^{-3}$, the original and the denoised channels are shown in Figures \ref{fig:gchannel_1_orig}, \ref{fig:rchannel_1_orig}, \ref{fig:gchannel_1_denoised} and \ref{fig:rchannel_1_denoised}
\end{definition}
%----------------------------------------------------------------------------------------
%	SUBSECTION 2
%----------------------------------------------------------------------------------------

\subsection{Object Data Enhancement}
\label{sec:contrastcorrection}

The fundamental task of binarization of an image is to split up an image into homogenous regions of two gray levels, $l_0$ and $l_1$. All pixels labelled $l_0$ share similar values with regard to the feature under consideration. This also means that there is high dissimilarity with pixels that are labelled $l_1$.
For grayscale images, such as the type we are concerned with, this feature is generally intensity.
For segmentation regarding a specific feature, an ideal image is one where there is no overlap in the feature space between the background and the object, as illustrated in \Cref{fig:featuredataset}. In real images (non-synthetic) this is almost never the case as there is no known feature space or it does not exist.

\begin{figure}[!h]
	\centering
	\subfigure[]
	{
		\includegraphics[width=0.48\columnwidth]{mypreprocess/feature_data_set_ideal.png}
		\label{fig:featuredataset_ideal}
	}	
	\subfigure[]
	{
		\includegraphics[width=0.48\columnwidth]{mypreprocess/feature_data_set_real.png}
		\label{fig:featuredataset_real}
	}
	\caption{Feature Distribution for ideal and real images. \textbf{(a)} Ideal image. The background and the object occupy distinct non-overlapping partitions of the feature space. The non-overlapping region is called the "Region of Disparity". \textbf{(b)} Real Image. The background and the object partially overlap in the feature space. The overlapping section is called the "Region of Uncertainty".}
	\label{fig:featuredataset}
\end{figure}

Grayscale medical image segmentation (MIS) typically uses intensity as the dominant feature on which the application of the segmentation algorithms is biased. Hence, it is common to precede segmentation by contrast enhancement \citep{Kim2003,Subr2005}. The problem therein lies with the fact that the aim of contrast enhancement is to bring out the details more clearly that are otherwise obscured due to limited dynamic range, non-uniform illumination, etc. In fluorescence image this does not necessarily mean that the contrast enhanced image possesses better "segmentation qualities", in terms of intensity.

In this section, we present a novel mapping function whose aim is to shrink the "region of uncertainty" by non-linearly widening the gap between background pixels and object pixels. This function is composed of two piece-wise sub-functions; one for data attenuation and one for data amplification. We first design the properties the mapping function should have. 

\begin{definition}[Remapping Function Properties]
	We denote the remap function as $\widetilde{x_i} = R(x_i)$, where $\widetilde{x_i}$ is the new value which remaps the input pixel, $x_i$, using the function $R$. The range on which $R$ works is $[0,L]$. For 8-bit gray scale images the highest value is usually $L=255$. 
	Let $N$ be the value that contains the greatest classification uncertainty, as illustrated in \Cref{fig:featuredataset_real}.
	$R$ must have the following properties:
\end{definition}

\begin{enumerate}
	\item{}
	\textit{$R$ must be non-decreasing in the interval $[0, L]$}\\
	This is a trivial criterion arising from the context in which our problem is defined. We specifically focus black background fluorescence images. Consequently, it is not possible for a lower intensity pixel to have a higher probability of belonging to the object compared to a pixel of a higher gray level intensity. 
	
	\item
	\textit{$\widetilde{x_i} = R(x_i=N) = N$}\\
	This value has no bias as to whether it tends more to the background or the object. It is best left unaltered. This is marked in \Cref{fig:remapfunctionspace} as "2".
	
	\item
	\textit{Attenuation: $\widetilde{x_i} < x_i, \, \forall x_i<N$}\\
	$R$ must remap gray-level intensities below $N$, according to $0 \leq \widetilde{x_i} \leq x_i$. This is marked in \Cref{fig:remapfunctionspace} as "3".
	
	\item
	\textit{Amplification: $\widetilde{x_i} > x_i, \, \forall x_i>N$}\\
	$R$ must remap gray-level intensities above $N$, according to $x_i \leq \widetilde{x_i} \leq L$. This is marked in \Cref{fig:remapfunctionspace} as "4".
	
	\item
	\textit{$R'$ must be non-decreasing in the interval $[0,N]$}\\
	Given two pixels, $p_1$ and $p_2$, with values $x_1$ and $x_2$ respectively where $x_1<x_2$. It is more probable for $p_2$ to belong to the object since it has a higher value. Also, pixels with gray levels intensities closer to $0$ do not need to be attenuated as much.
	
	\item
	\textit{$R'$ must be non-increasing in the interval $[N,L]$}\\
	As the pixels values approach $L$, less amplification is needed since the pixels are already more likely to be classified as belonging to the object.
\end{enumerate}

%\begin{figure}[!t]
%	\centering
%	\includegraphics[width=0.45\columnwidth]{mypreprocess/remapfcn_assist.png}
%	\caption{Remapping function space.}
%	\label{fig:remapfunctionspace}
%\end{figure}
%\begin{figure}[!t]
%	\centering
%	\includegraphics[width=0.45\columnwidth]{remapfcn.pdf}
%	\caption{Plot of remapping function.}
%	\label{fig:remapfcn}
%\end{figure}


\begin{figure}[!t]
	\centering
	\begin{minipage}{.5\textwidth}
		\centering
		\includegraphics[width=\linewidth]{mypreprocess/remapfcn_assist.png}
		\captionof{figure}{Remapping function space.}
		\label{fig:remapfunctionspace}
	\end{minipage}%
	\begin{minipage}{.5\textwidth}
		\centering
		\includegraphics[width=1\linewidth]{remapfcn.pdf}
		\captionof{figure}{Plot of remapping function.}
		\label{fig:remapfcn}
	\end{minipage}
\end{figure}


\begin{definition}[Function Design]
	Given the criteria presented, many functions can be designed. We have decided that one of the better solutions is to make the mapping function a piece-wise quadratic Bezier curve. This is to maintain intuitive tuning of the sub-functions. The two functions are $R_{att}$, for the attenuation section, and $R_{amp}$, for the amplification section as shown in \Cref{fig:remapfcn}.
	
	There are three anchor points the function must pass through. These are $p_0(0,0)$, $p_2(N, N)$ and $p_4(L,L)$. If the curve between $p_0$ and $p_4$ is a straight line, then there is no change between the output and input image since the gradient of the line is equal to $1$. Additional control points are needed to bend the curve.
	There is a particular class of curves whose shapes are useful for our purpose. This places constraints on the position of the control points. Relative to the straight line, $y=x, \, x \in [0,L]$, the amplification curve would be above the relaxed line, $p_{3y} > p_{3x}$, and similarly the attenuation curve would be below the line, $p_{1y} < p_{1x}$. Also, given that the function is to be a one-to-one function, the control point for the attenuation function $p_{1x}<N$; similarly, the control point for the amplification function $p_{3x}>N$.
	
	It is more geometrically intuitive to represent the control points in polar coordinates with the centre at $p_2(N,N)$. Therefore, control point $c$ is represented as $p_c(R_c, \theta_c)$; where $\theta_c \in [0, \frac{\pi}{4}]$ and $R_c \in \mathbb{R}^+$. The angular deviation is the measure off the straight line which is counter-clockwise for the amplification function and clockwise for the attenuation function. The deviation, $\theta_c$, where $c \in {1,2}$, away from the straight line is further implicitly represented as a range $\kappa_c \in [0,1]$ where $\kappa_c=0$ implies $\theta_c = 0$ would mean no deviation, and $\kappa_c = 1$ implies $\theta_c = \frac{\pi}{4}$ would mean maximum deviation.
	As the function approaches the ends of its domain, less attenuation or amplification occurs, which meets criteria 5 and 6.
	
	\begin{figure}[!t]
		\centering
		\subfigure[Amplification function controls points.]
		{
			
			\includegraphics[width=0.48\columnwidth]{attenutation_control_points.pdf}
			\label{fig:ampcontrolpoints}
		}	
		\subfigure[Amplification function curve.]
		{
			\includegraphics[width=0.48\columnwidth]{attenutation_curve.pdf}
			\label{fig:ampcurve}
		}
		\caption{Amplification function control points and curve.}
		\label{fig:ampcalculation}
	\end{figure}

	The position of the attenuation control point is calculated as
	\begin{equation*}
		{p_1} = \left(
		p_{2x} - R_1\cos[\frac{\pi}{4}(1+\kappa_1)],
		p_{2y} - R_1\sin[\frac{\pi}{4}(1+\kappa_1)]
		\right)
	\end{equation*}
	Similarly, the position of the amplification control point, as shown in \Cref{fig:ampcalculation}, can be calculated as
	\begin{equation*}
		{p_3} = \left(
		p_{2x} + R_2\cos[\frac{\pi}{4}(1+\kappa_2)],
		p_{2y} + R_2\sin[\frac{\pi}{4}(1+\kappa_2)]
		\right)
	\end{equation*}
	
	The piecewise functions that define the curve are given by	
	\begin{eqnarray}
		R_{att}(t) = \sum_{i=0}^2 p_i J_i^n \\
		R_{amp}(t) = \sum_{i=2}^4 p_i J_i^n
	\end{eqnarray}
	Where $J_i^n$ is the Bernstein basis function, defined as
	\begin{equation}
	J_i^n = \begin{pmatrix}
	n \\
	i
	\end{pmatrix}	
	t^i(1-t)^{n-i}, \,\, t \in [0,1]
	\label{eq:bernsteinbasis}
	\end{equation}
	
	For each input gray level intensity $x \in [0, L]$ we require corresponding output gray level intensity $y \in [0, L]$.
	The curve is parameterised with respect to $t$ for each of its $x$ and $y$ parameters.
	For a curve defined by three control points $q_0$, $q_1$ and $q_2$ where $q_0 < q_1 < q_2$, the equation for the curve determine by these points is the quadratic Bezier given by
	\begin{equation}
		R = \sum_{i=0}^2 q_i J_i^n
	\end{equation}
	
	For a quadratic Bezier curve, this simplifies to
	\begin{align*}
		R &= q_0(1-t)^2 + 2q_1(1-t)t + q_2t^2 \\
		0 &= (q_0-2q_1+q_2)t^2 + (-2q_0+2q_1)t + (q_0-R)
	\end{align*}
	
	Therefore for the parametric function $R_x$, we require the set of values of $t$ for which $R_x \in [q_{0x},q_{2x}]$. The value of $t$ for a given $R_x$ is determined by the solution to the quadratic equation.\\
	Let $a=q_{0x}-2q_{1x}+q_{2x}$, $b=-2q_{0x}+2q_{1x}$, and $c=q_{0x}-R_x$.\\
	It is seen that $b>0$, since $q_{1x} > q_{0x}$, and $c \leq 0$, since $R_x \geq q_{0x}$.\\
	Rewrite $a$ as $a=(q_{0x}-q_{1x}) + (q_{2x}-q_{1x})$.
	For the case of $a$, there are two cases.
	
	\begin{enumerate}
		\item
		For $q_{1x} \in [q_{0x}, \frac{q_{2x}-q_{0x}}{2}]$, $a>0$.\\
		In this case $-4ac>0$ hence $\sqrt{b^2-4ac}>b$.\\
		$\therefore$ the only value of $t$ which is positive and must be the solution is given by
		\begin{equation}\label{eq:qt1}
			t =  \frac{(q_{1x}-q_{0x}) + \sqrt{R_x(q_{0x}-2q_{1x}+q_{2x})+(q_{1x}^2-q_{0x}q_{2x}})}{q_{0x}-2q_{1x}+q_{2x}}
		\end{equation}

		\item
		For $q_{1x}>\frac{q_{2x}-q_{0x}}{2}$, $a<0$.\\
		It is known that a real solution exists. This means that
		\begin{align*}
			b^2 -4ac &\geq 0 \\
			b^2 &\geq 4ac \\
			\implies \sqrt{b^2-4ac} &< b
		\end{align*}
		$\therefore$ the only value of $t$ which is positive and must be the solution is given by
		\begin{equation}\label{eq:qt2}
			t =  \frac{(q_{1x}-q_{0x}) + \sqrt{R_x(q_{0x}-2q_{1x}+q_{2x})+(q_{1x}^2-q_{0x}q_{2x}})}{q_{0x}-2q_{1x}+q_{2x}}
		\end{equation}
	\end{enumerate}

	In both cases the solution to $t$ can be calculated using the same formula.
	It is possible for the remapping function to exceed the range, in this case any value that maps to a value higher than the maximum value will be assigned the maximum value i.e. $\widetilde{x_i} = min(R(x_i),L), \,  x_i \in (N,L]$; similarly any value that maps to a value lower than the minimum value will be assigned the minimum value i.e. $\widetilde{x_i} = max(R(x_i),0), \, x_i \in [0,N)$.
\end{definition}

\begin{definition}[Function Properties and Constraints]
This function presents several properties within the constraints defined as follows:
\begin{enumerate}
	\item
	The curves obey the convex hull property. The sub-functions will always be contained within the control polygon determined by the control points \citep{Vince2006,Marsh2005}.
	
	\item
	No attenuation when $\kappa_1=0$ and no amplification when $\kappa_2=0$.
	
	\item
	$R$ is continuous on $[0,L]$.
	
	\item
	$R$, is weakly monotonically increasing on $[0,L]$.
	
	\item
	$R'_{att}$, is monotonically increasing on $[0, N]$.
	
	\item
	$R'_{amp}$, is monotonically decreasing on $[N, L]$.
	
	\item
	The ends of the curve are coincident with the first and last control points of the control polygon.
	
	\item
	The direction of the tangent vectors from the end points of the curve are the same as the direction of the vector anchored at the control point and along the line that joins the end point and the centre control point of the control polygon \citep{Vince2006,Marsh2005}.
\end{enumerate}
\end{definition}


%----------------------------------------------------------------------------------------
%	SUBSECTION 3
%----------------------------------------------------------------------------------------

\subsection{Channel Mixing}
\label{sec:channelmixing}

To reconstitute an image from the updated channels we perform channel mixing. In an equi-weighted mixing system, each channel contributes equally to the final image. This simplistic method of channel mixing does not always produce the best image.
A consequence, of equi-weighted mixing is that some channels might become very suppressed and may be disregarded by the segmentation algorithm. Hence, it is necessary to assign weights which are channel dependant. These weightings must sum to one, $\sum_{i \in C} w_i = 1$, where $C$ is the set of channels to be mixed-down. The mixed-down image is then calculated as $y = \sum_{i \in C}w_iC_i$, where $C_i$ is channel $i$. Channels with very low gray level values are given a greater weight.

%We used $w_R=0.5$, $w_G=0.5$, and $w_B=0.0$ to produce Figure \ref{fig:mixingchannels}, where $w_X$ is the contribution of channel X to the mixed-down image. 


%----------------------------------------------------------------------------------------
%	SUBSECTION 4
%----------------------------------------------------------------------------------------

\subsection{Intra-Region Smoothing and Edge Completion and Enhancement}
\label{sec:Diffusion}

One of the criteria which is used in identifying an image is that objects tend to have little intra-region variance.
It is common for fluorescence images to be plagued with various degrees of lighting and contrast even within objects. This is primarily due to improper excitation, non-fluorescence of particles and ubiquitous measurement errors during acquisition.
%In some cases it is easy to visualise the completed edge of the object by joining the disconnected edge components where the extension is along the direction of the edge.
We used a coherence enhancing diffusion filter with optimised rotational invariance (CED-ORI) presented in \citep{Weickert1999,Weickert2002,Weickert2003}, which very successfully reduces intra-region variance and joins closely-disconnected edges.

The diffusion works by evolving the image, $u$, over a time using $n$ discrete time steps, $t$, called the diffusion time. The evolution equation is defined as:

\begin{equation}
\frac{\partial u}{\partial t} = \nabla \cdot (D\nabla u)
\end{equation}

where $D = \begin{pmatrix}
a & b \\
b & c
\end{pmatrix}$ is the diffusion tensor which can be adapted to the local image structure measure known as the structure tensor. The structure tensor is given by:

\begin{equation}
J_{\rho}(\nabla u_{\sigma}) = G_{\rho} \ast (\nabla u_{\sigma} \nabla u_{\sigma}^T)
\end{equation}

Where $G_{\rho}$ is the Gaussian kernel with standard deviation $\rho$, and $u_{\sigma} := G_{\sigma} \ast u$ where $G_{\sigma}$ is the Gaussian kernel with standard deviation $\sigma$.
The eigenvalues of $J_{\rho}=\begin{pmatrix}
J_{11} & J_{12} \\
J_{12} & J_{22}
\end{pmatrix}$ are

\begin{eqnarray}
\mu_{1} = \frac{1}{2}\left( J_{11} + J_{12} + \sqrt{(J_{11}-J_{22})^2+4J_{12}^2} \right) \\
\mu_{2} = \frac{1}{2}\left( J_{11} + J_{12} - \sqrt{(J_{11}-J_{22})^2+4J_{12}^2} \right)
\end{eqnarray}

Where the normalised first eigenvector satisfies

\begin{equation}
\begin{pmatrix}
cos \alpha \\
sin \alpha
\end{pmatrix} \parallel
\begin{pmatrix}
2J_{12} \\
J_{22}-J_{11}+\sqrt{(J_{11}-J_{22})^2 + 4J_{12}^2}
\end{pmatrix}
\end{equation}

The diffusion tensor's, $D$, eigenvectors are obtained from the structure tensor eigenvectors using:

\begin{eqnarray}
\lambda_1 &=& c_1 \\
\lambda_2 &=& \left\lbrace \begin{matrix}
c_1 & \text{if } \mu_1=\mu_2 \\
c_1+(1-c_1)e^{\frac{c_2}{(\mu_1-\mu_2)^2}}& \text{otherwise}
\end{matrix}
\right.
\end{eqnarray}

where $c_1 \in (0,1)$, $c_2>0$. The elements of $D$ are then calculated as:

\begin{eqnarray}
a = \lambda_1 cos^2 \alpha + \lambda_2 sin^2 \alpha \\
b = (\lambda_1 - \lambda_2)sin \alpha cos \alpha \\
c = \lambda_1 sin^2 \alpha + \lambda_2 cos^2 \alpha
\end{eqnarray}

Further details on the coherence enhancing diffusion filter with optimised rotational invariance is found in \cite{Weickert2002}. In Masaka \textit{et al.} \citep{Maska2013} and Kroon \textit{et al.} \citep{Kroon2009} this filter was used primarily for noise removal while preserving edge detail, here we use it for its edge completion and smoothing properties.
%The comparison between the original image and the final image is shown in Figure  \ref{Org_dif}, the diffused image appears very distorted however the data necessary for the object extraction is present with less intra-region variance and better edge coherence. These added qualities are vital to yield optimal segmentations.

%----------------------------------------------------------------------------------------
%	SECTION 3
%----------------------------------------------------------------------------------------

\section{Experimental Results}
\label{sec:preprocessschemeexperimentalresults}

In this section we present the results of the  proposed pre-processing scheme compared against other very commonly used fluorescence image enhancement methods. We have used the graph cut version of the Chan-Vese segmenation model, and kept the parameters the same over the same images after different pre-processing methods. The parameter settings used in each scheme is given under the image.

We have compiled a label for label comparison on the segmentation mask and the ground truth in \Cref{tab:preprocessresults} along with the efficiency measures \textit{precision}, \textit{recall}, \textit{accuracy} and \textit{Matthews Correlation Coefficient (MCC)}. The overview results are shown in \Cref{tab:overallpreprocessingsegmentationefficiency}. For each image, we have highlighted the method which performs the best in blue, and the worst in red.

We differentiate between methods on the same image as follows:

\textbf{[imageno]-[method]}, 

where \textit{imageno} goes from $1$ to $25$ and \textit{method} is defined as follows:
\begin{enumerate}
	\item [\textbf{o}] - Original image without any pre-processing.
	\item [\textbf{c}] - Image after Coherence Enhancing Diffusion with Optimised Rotataional Invariance (CEDORI).
	\item [\textbf{t}] - Image after Total Variation denoising.
	\item [\textbf{p}] - Image after Proposed pre-processing scheme.
\end{enumerate}

\clearpage
%%%%%%%%%%%%%%%%%%%%%%%%%%%%%%%%%%%%%%%%%%%%%%%%%%%%%%%
% 188
\begin{figure}[!h]
	\centering
	\subfigure[Image 1 without any pre-processing.]
	{
		\includegraphics[width=0.22\columnwidth]{mypreprocess/orig/188gray}
		\label{fig:188none}
	}
	\subfigure[Image 1 after CEDORI. $\sigma=1.0$, $\rho=1.5$, $n=2$, $\tau=0.0015$.]
	{
		\includegraphics[width=0.22\columnwidth]{mypreprocess/cedori/188ced}
		\label{fig:188ced}
	}
	\subfigure[Image 1 after TV denoising. $\lambda=1.0$, $\epsilon=1\times10^{-3}$, $\gamma=20$.]
	{
		\includegraphics[width=0.22\columnwidth]{mypreprocess/tv/188graytv}
		\label{fig:188tv}
	}
	\subfigure[Image 1 after Proposed Pre-processing scheme. $\kappa_1=0.17$, $\kappa_2=0.55$, $R_1=50$, $R_2=225$,$N=50$.]
	{
		\includegraphics[width=0.22\columnwidth]{mypreprocess/proposed/188ced}
		\label{fig:188prop}
	}

	\subfigure[Original segmentation.]
	{
		\includegraphics[width=0.22\columnwidth]{mypreprocess/origseg/def188}
		\label{fig:188noneseg}
	}
	\subfigure[CEDORI segmentation.]
	{
		\includegraphics[width=0.22\columnwidth]{mypreprocess/cedseg/def188}
		\label{fig:188cedseg}
	}
	\subfigure[Total Variation Denoising segmentation.]
	{
		\includegraphics[width=0.22\columnwidth]{mypreprocess/tvseg/def188}
		\label{fig:188tvseg}
	}
	\subfigure[Proposed Pre-process segmentation.]
	{
		\includegraphics[width=0.22\columnwidth]{mypreprocess/proposedseg/def188}
		\label{fig:188propseg}
	}
	\caption{Image 1 from test set \Cref{AppendixA} pre-processing segmentation results. $\mu=1$, $\lambda_0=240$, $\lambda_1=8$.}
	\label{fig:preprocessschemeresult188}
\end{figure}

%%%%%%%%%%%%%%%%%%%%%%%%%%%%%%%%%%%%%%%%%%%%%%%%%%%%%%%
% 195
\begin{figure}[!h]
	\centering
	\subfigure[Image 2 without any pre-processing.]
	{
		\includegraphics[width=0.22\columnwidth]{mypreprocess/orig/195gray}
		\label{fig:195none}
	}
	\subfigure[Image 2 after CEDORI. $\sigma=0.5$, $\rho=1.0$, $n=1$, $\tau=0.0005$.]
	{
		\includegraphics[width=0.22\columnwidth]{mypreprocess/cedori/195ced}
		\label{fig:195ced}
	}
	\subfigure[Image 2 after TV denoising. $\lambda=1.0$, $\epsilon=1\times10^{-3}$, $\gamma=5$.]
	{
		\includegraphics[width=0.22\columnwidth]{mypreprocess/tv/195graytv}
		\label{fig:195tv}
	}
	\subfigure[Image 2 after Proposed Pre-processing scheme. $\kappa_1=0.17$, $\kappa_2=0.55$, $R_1=50$, $R_2=225$,$N=50$.]
	{
		\includegraphics[width=0.22\columnwidth]{mypreprocess/proposed/195ced}
		\label{fig:195prop}
	}
	
	\subfigure[Original segmentation.]
	{
		\includegraphics[width=0.22\columnwidth]{mypreprocess/origseg/def195}
		\label{fig:195noneseg}
	}
	\subfigure[CEDORI segmentation.]
	{
		\includegraphics[width=0.22\columnwidth]{mypreprocess/cedseg/def195}
		\label{fig:195cedseg}
	}
	\subfigure[Total Variation Denoising segmentation.]
	{
		\includegraphics[width=0.22\columnwidth]{mypreprocess/tvseg/def195}
		\label{fig:195tvseg}
	}
	\subfigure[Proposed Pre-process segmentation.]
	{
		\includegraphics[width=0.22\columnwidth]{mypreprocess/proposedseg/def195}
		\label{fig:195propseg}
	}
	\caption{Image 2 from test set \Cref{AppendixA} pre-processing segmentation results. $\mu=1$, $\lambda_0=5$, $\lambda_1=1$.}
	\label{fig:preprocessschemeresult195}
\end{figure}

%%%%%%%%%%%%%%%%%%%%%%%%%%%%%%%%%%%%%%%%%%%%%%%%%%%%%%%
% 228
\begin{figure}[!h]
	\centering
	\subfigure[Image 3 without any pre-processing.]
	{
		\includegraphics[width=0.22\columnwidth]{mypreprocess/orig/228gray}
		\label{fig:228none}
	}
	\subfigure[Image 3 after CEDORI. $\sigma=0.5$, $\rho=1.5$, $n=2$, $\tau=0.0005$.]
	{
		\includegraphics[width=0.22\columnwidth]{mypreprocess/cedori/228ced}
		\label{fig:228ced}
	}
	\subfigure[Image 3 after TV denoising. $\lambda=1.0$, $\epsilon=1\times10^{-3}$, $\gamma=20$.]
	{
		\includegraphics[width=0.22\columnwidth]{mypreprocess/tv/228graytv}
		\label{fig:228tv}
	}
	\subfigure[Image 3 after Proposed Pre-processing scheme. $\kappa_1=0.17$, $\kappa_2=0.55$, $R_1=50$, $R_2=225$,$N=50$.]
	{
		\includegraphics[width=0.22\columnwidth]{mypreprocess/proposed/228ced}
		\label{fig:228prop}
	}
	
	\subfigure[Original segmentation.]
	{
		\includegraphics[width=0.22\columnwidth]{mypreprocess/origseg/def228}
		\label{fig:228noneseg}
	}
	\subfigure[CEDORI segmentation.]
	{
		\includegraphics[width=0.22\columnwidth]{mypreprocess/cedseg/def228}
		\label{fig:228cedseg}
	}
	\subfigure[Total Variation Denoising segmentation.]
	{
		\includegraphics[width=0.22\columnwidth]{mypreprocess/tvseg/def228}
		\label{fig:228tvseg}
	}
	\subfigure[Proposed Pre-process segmentation.]
	{
		\includegraphics[width=0.22\columnwidth]{mypreprocess/proposedseg/def228}
		\label{fig:228propseg}
	}
	\caption{Image 3 from test set \Cref{AppendixA} pre-processing segmentation results. $\mu=1$, $\lambda_0=240$, $\lambda_1=8$.}
	\label{fig:preprocessschemeresult228}
\end{figure}

%%%%%%%%%%%%%%%%%%%%%%%%%%%%%%%%%%%%%%%%%%%%%%%%%%%%%%%
% 1057
\begin{figure}[!h]
	\centering
	\subfigure[Image 4 without any pre-processing.]
	{
		\includegraphics[width=0.22\columnwidth]{mypreprocess/orig/1057gray}
		\label{fig:1057none}
	}
	\subfigure[Image 4 after CEDORI. $\sigma=1.5$, $\rho=2.0$, $n=5$, $\tau=0.0015$.]
	{
		\includegraphics[width=0.22\columnwidth]{mypreprocess/cedori/1057ced}
		\label{fig:1057ced}
	}
	\subfigure[Image 4 after TV denoising. $\lambda=1.0$, $\epsilon=1\times10^{-3}$, $\gamma=20$.]
	{
		\includegraphics[width=0.22\columnwidth]{mypreprocess/tv/1057graytv}
		\label{fig:1057tv}
	}
	\subfigure[Image 4 after Proposed Pre-processing scheme. $\kappa_1=0.17$, $\kappa_2=0.55$, $R_1=50$, $R_2=225$,$N=50$.]
	{
		\includegraphics[width=0.22\columnwidth]{mypreprocess/proposed/1057ced}
		\label{fig:1057prop}
	}
	
	\subfigure[Original segmentation.]
	{
		\includegraphics[width=0.22\columnwidth]{mypreprocess/origseg/def1057}
		\label{fig:1057noneseg}
	}
	\subfigure[CEDORI segmentation.]
	{
		\includegraphics[width=0.22\columnwidth]{mypreprocess/cedseg/def1057}
		\label{fig:1057cedseg}
	}
	\subfigure[Total Variation Denoising segmentation.]
	{
		\includegraphics[width=0.22\columnwidth]{mypreprocess/tvseg/def1057}
		\label{fig:1057tvseg}
	}
	\subfigure[Proposed Pre-process segmentation.]
	{
		\includegraphics[width=0.22\columnwidth]{mypreprocess/proposedseg/def1057}
		\label{fig:1057propseg}
	}
	\caption{Image 4 from test set \Cref{AppendixA} pre-processing segmentation results. $\mu=1$, $\lambda_0=300$, $\lambda_1=5$.}
	\label{fig:preprocessschemeresult1057}
\end{figure}

%%%%%%%%%%%%%%%%%%%%%%%%%%%%%%%%%%%%%%%%%%%%%%%%%%%%%%%
% 1265
\begin{figure}[!h]
	\centering
	\subfigure[Image 5 without any pre-processing.]
	{
		\includegraphics[width=0.22\columnwidth]{mypreprocess/orig/1265gray}
		\label{fig:1265none}
	}
	\subfigure[Image 5 after CEDORI. $\sigma=1.5$, $\rho=2.0$, $n=5$, $\tau=0.0015$.]
	{
		\includegraphics[width=0.22\columnwidth]{mypreprocess/cedori/1265ced}
		\label{fig:1265ced}
	}
	\subfigure[Image 5 after TV denoising. $\lambda=1.0$, $\epsilon=1\times10^{-3}$, $\gamma=20$.]
	{
		\includegraphics[width=0.22\columnwidth]{mypreprocess/tv/1265graytv}
		\label{fig1265tv}
	}
	\subfigure[Image 5 after Proposed Pre-processing scheme. $\kappa_1=0.17$, $\kappa_2=0.55$, $R_1=50$, $R_2=225$,$N=50$.]
	{
		\includegraphics[width=0.22\columnwidth]{mypreprocess/proposed/1265ced}
		\label{fig:1265prop}
	}
	
	\subfigure[Original segmentation.]
	{
		\includegraphics[width=0.22\columnwidth]{mypreprocess/origseg/def1265}
		\label{fig:1265noneseg}
	}
	\subfigure[CEDORI segmentation.]
	{
		\includegraphics[width=0.22\columnwidth]{mypreprocess/cedseg/def1265}
		\label{fig:1265cedseg}
	}
	\subfigure[Total Variation Denoising segmentation.]
	{
		\includegraphics[width=0.22\columnwidth]{mypreprocess/tvseg/def1265}
		\label{fig:1265tvseg}
	}
	\subfigure[Proposed Pre-process segmentation.]
	{
		\includegraphics[width=0.22\columnwidth]{mypreprocess/proposedseg/def1265}
		\label{fig:1265propseg}
	}
	\caption{Image 5 from test set \Cref{AppendixA} pre-processing segmentation results. $\mu=1$, $\lambda_0=220$, $\lambda_1=11$.}
	\label{fig:preprocessschemeresult1265}
\end{figure}

%%%%%%%%%%%%%%%%%%%%%%%%%%%%%%%%%%%%%%%%%%%%%%%%%%%%%%%
% 10093
\begin{figure}[!h]
	\centering
	\subfigure[Image 6 without any pre-processing.]
	{
		\includegraphics[width=0.22\columnwidth]{mypreprocess/orig/10093gray}
		\label{fig:10093none}
	}
	\subfigure[Image 6 after CEDORI. $\sigma=3.0$, $\rho=5.0$, $n=20$, $\tau=0.00075$.]
	{
		\includegraphics[width=0.22\columnwidth]{mypreprocess/cedori/10093ced}
		\label{fig:10093ced}
	}
	\subfigure[Image 6 after TV denoising. $\lambda=1.0$, $\epsilon=1\times10^{-3}$, $\gamma=20$.]
	{
		\includegraphics[width=0.22\columnwidth]{mypreprocess/tv/10093graytv}
		\label{fig:10093tv}
	}
	\subfigure[Image 6 after Proposed Pre-processing scheme. $\kappa_1=0.17$, $\kappa_2=0.65$, $R_1=12$, $R_2=225$,$N=12$.]
	{
		\includegraphics[width=0.22\columnwidth]{mypreprocess/proposed/10093ced}
		\label{fig:10093prop}
	}
	
	\subfigure[Original segmentation.]
	{
		\includegraphics[width=0.22\columnwidth]{mypreprocess/origseg/def10093}
		\label{fig:10093noneseg}
	}
	\subfigure[CEDORI segmentation.]
	{
		\includegraphics[width=0.22\columnwidth]{mypreprocess/cedseg/def10093}
		\label{fig:10093cedseg}
	}
	\subfigure[Total Variation Denoising segmentation.]
	{
		\includegraphics[width=0.22\columnwidth]{mypreprocess/tvseg/def10093}
		\label{fig:10093tvseg}
	}
	\subfigure[Proposed Pre-process segmentation.]
	{
		\includegraphics[width=0.22\columnwidth]{mypreprocess/proposedseg/def10093}
		\label{fig:10093propseg}
	}
	\caption{Image 6 from test set \Cref{AppendixA} pre-processing segmentation results. $\mu=1$, $\lambda_0=1600$, $\lambda_1=80$.}
	\label{fig:preprocessschemeresult10093}
\end{figure}

%%%%%%%%%%%%%%%%%%%%%%%%%%%%%%%%%%%%%%%%%%%%%%%%%%%%%%%
% 10093
\begin{figure}[!h]
	\centering
	\subfigure[Image 7 without any pre-processing.]
	{
		\includegraphics[width=0.22\columnwidth]{mypreprocess/orig/10102gray}
		\label{fig:10102none}
	}
	\subfigure[Image 7 after CEDORI. $\sigma=3.0$, $\rho=3.0$, $n=20$, $\tau=0.00075$.]
	{
		\includegraphics[width=0.22\columnwidth]{mypreprocess/cedori/10102ced}
		\label{fig:10102ced}
	}
	\subfigure[Image 7 after TV denoising. $\lambda=1.0$, $\epsilon=1\times10^{-3}$, $\gamma=20$.]
	{
		\includegraphics[width=0.22\columnwidth]{mypreprocess/tv/10102graytv}
		\label{fig:10102tv}
	}
	\subfigure[Image 7 after Proposed Pre-processing scheme. $\kappa_1=0.17$, $\kappa_2=0.65$, $R_1=12$, $R_2=225$,$N=12$.]
	{
		\includegraphics[width=0.22\columnwidth]{mypreprocess/proposed/10102ced}
		\label{fig:10102prop}
	}
	
	\subfigure[Original segmentation.]
	{
		\includegraphics[width=0.22\columnwidth]{mypreprocess/origseg/def10102}
		\label{fig:10102noneseg}
	}
	\subfigure[CEDORI segmentation.]
	{
		\includegraphics[width=0.22\columnwidth]{mypreprocess/cedseg/def10102}
		\label{fig:10102cedseg}
	}
	\subfigure[Total Variation Denoising segmentation.]
	{
		\includegraphics[width=0.22\columnwidth]{mypreprocess/tvseg/def10102}
		\label{fig:10102tvseg}
	}
	\subfigure[Proposed Pre-process segmentation.]
	{
		\includegraphics[width=0.22\columnwidth]{mypreprocess/proposedseg/def10102}
		\label{fig:10102propseg}
	}
	\caption{Image 7 from test set \Cref{AppendixA} pre-processing segmentation results. $\mu=1$, $\lambda_0=10$, $\lambda_1=1$.}
	\label{fig:preprocessschemeresult101023}
\end{figure}

%%%%%%%%%%%%%%%%%%%%%%%%%%%%%%%%%%%%%%%%%%%%%%%%%%%%%%%
% 10104
\begin{figure}[!h]
	\centering
	\subfigure[Image 8 without any pre-processing.]
	{
		\includegraphics[width=0.22\columnwidth]{mypreprocess/orig/10104gray}
		\label{fig:10104none}
	}
	\subfigure[Image 8 after CEDORI. $\sigma=3.0$, $\rho=3.0$, $n=20$, $\tau=0.00075$.]
	{
		\includegraphics[width=0.22\columnwidth]{mypreprocess/cedori/10104ced}
		\label{fig:10104ced}
	}
	\subfigure[Image 8 after TV denoising. $\lambda=1.0$, $\epsilon=1\times10^{-3}$, $\gamma=20$.]
	{
		\includegraphics[width=0.22\columnwidth]{mypreprocess/tv/10104graytv}
		\label{fig:10104tv}
	}
	\subfigure[Image 8 after Proposed Pre-processing scheme. $\kappa_1=0.17$, $\kappa_2=0.65$, $R_1=12$, $R_2=225$,$N=12$.]
	{
		\includegraphics[width=0.22\columnwidth]{mypreprocess/proposed/10104ced}
		\label{fig:10104prop}
	}
	
	\subfigure[Original segmentation.]
	{
		\includegraphics[width=0.22\columnwidth]{mypreprocess/origseg/def10104}
		\label{fig:10104noneseg}
	}
	\subfigure[CEDORI segmentation.]
	{
		\includegraphics[width=0.22\columnwidth]{mypreprocess/cedseg/def10104}
		\label{fig:10104cedseg}
	}
	\subfigure[Total Variation Denoising segmentation.]
	{
		\includegraphics[width=0.22\columnwidth]{mypreprocess/tvseg/def10104}
		\label{fig:10104tvseg}
	}
	\subfigure[Proposed Pre-process segmentation.]
	{
		\includegraphics[width=0.22\columnwidth]{mypreprocess/proposedseg/def10104}
		\label{fig:10104propseg}
	}
	\caption{Image 8 from test set \Cref{AppendixA} pre-processing segmentation results. $\mu=1$, $\lambda_0=10$, $\lambda_1=1$.}
	\label{fig:preprocessschemeresult10104}
\end{figure}

%%%%%%%%%%%%%%%%%%%%%%%%%%%%%%%%%%%%%%%%%%%%%%%%%%%%%%%
% 12294
\begin{figure}[!h]
	\centering
	\subfigure[Image 9 without any pre-processing.]
	{
		\includegraphics[width=0.22\columnwidth]{mypreprocess/orig/12294gray}
		\label{fig:12294none}
	}
	\subfigure[Image 9 after CEDORI. $\sigma=0.5$, $\rho=1.0$, $n=2$, $\tau=0.0005$.]
	{
		\includegraphics[width=0.22\columnwidth]{mypreprocess/cedori/12294ced}
		\label{fig:12294ced}
	}
	\subfigure[Image 9 after TV denoising. $\lambda=1.0$, $\epsilon=1\times10^{-3}$, $\gamma=5$.]
	{
		\includegraphics[width=0.22\columnwidth]{mypreprocess/tv/12294graytv}
		\label{fig:12294tv}
	}
	\subfigure[Image 9 after Proposed Pre-processing scheme. $\kappa_1=0.17$, $\kappa_2=0.65$, $R_1=12$, $R_2=225$,$N=12$.]
	{
		\includegraphics[width=0.22\columnwidth]{mypreprocess/proposed/12294ced}
		\label{fig:12294prop}
	}
	
	\subfigure[Original segmentation.]
	{
		\includegraphics[width=0.22\columnwidth]{mypreprocess/origseg/def12294}
		\label{fig:12294noneseg}
	}
	\subfigure[CEDORI segmentation.]
	{
		\includegraphics[width=0.22\columnwidth]{mypreprocess/cedseg/def12294}
		\label{fig:12294cedseg}
	}
	\subfigure[Total Variation Denoising segmentation.]
	{
		\includegraphics[width=0.22\columnwidth]{mypreprocess/tvseg/def12294}
		\label{fig:12294tvseg}
	}
	\subfigure[Proposed Pre-process segmentation.]
	{
		\includegraphics[width=0.22\columnwidth]{mypreprocess/proposedseg/def12294}
		\label{fig:12294propseg}
	}
	\caption{Image 9 from test set \Cref{AppendixA} pre-processing segmentation results. $\mu=1$, $\lambda_0=2700$, $\lambda_1=90$.}
	\label{fig:preprocessschemeresult12294}
\end{figure}

%%%%%%%%%%%%%%%%%%%%%%%%%%%%%%%%%%%%%%%%%%%%%%%%%%%%%%%
% 12627
\begin{figure}[!h]
	\centering
	\subfigure[Image 10 without any pre-processing.]
	{
		\includegraphics[width=0.22\columnwidth]{mypreprocess/orig/12627gray}
		\label{fig:12627none}
	}
	\subfigure[Image 10 after CEDORI. $\sigma=3.0$, $\rho=5.0$, $n=10$, $\tau=0.0005$.]
	{
		\includegraphics[width=0.22\columnwidth]{mypreprocess/cedori/12627ced}
		\label{fig:12627ced}
	}
	\subfigure[Image 10 after TV denoising. $\lambda=1.0$, $\epsilon=1\times10^{-3}$, $\gamma=20$.]
	{
		\includegraphics[width=0.22\columnwidth]{mypreprocess/tv/12627graytv}
		\label{fig:12627tv}
	}
	\subfigure[Image 10 after Proposed Pre-processing scheme. $\kappa_1=0.17$, $\kappa_2=0.65$, $R_1=12$, $R_2=225$,$N=12$.]
	{
		\includegraphics[width=0.22\columnwidth]{mypreprocess/proposed/12627ced}
		\label{fig:12627prop}
	}
	
	\subfigure[Original segmentation.]
	{
		\includegraphics[width=0.22\columnwidth]{mypreprocess/origseg/def12627}
		\label{fig:12627noneseg}
	}
	\subfigure[CEDORI segmentation.]
	{
		\includegraphics[width=0.22\columnwidth]{mypreprocess/cedseg/def12627}
		\label{fig:12627cedseg}
	}
	\subfigure[Total Variation Denoising segmentation.]
	{
		\includegraphics[width=0.22\columnwidth]{mypreprocess/tvseg/def12627}
		\label{fig:12627tvseg}
	}
	\subfigure[Proposed Pre-process segmentation.]
	{
		\includegraphics[width=0.22\columnwidth]{mypreprocess/proposedseg/def12627}
		\label{fig:12627propseg}
	}
	\caption{Image 10 from test set \Cref{AppendixA} pre-processing segmentation results. $\mu=1$, $\lambda_0=4050$, $\lambda_1=135$.}
	\label{fig:preprocessschemeresult12627}
\end{figure}

%%%%%%%%%%%%%%%%%%%%%%%%%%%%%%%%%%%%%%%%%%%%%%%%%%%%%%%
% 13432
\begin{figure}[!h]
	\centering
	\subfigure[Image 11 without any pre-processing.]
	{
		\includegraphics[width=0.22\columnwidth]{mypreprocess/orig/13432gray}
		\label{fig:13432none}
	}
	\subfigure[Image 11 after CEDORI. $\sigma=1.0$, $\rho=1.0$, $n=1$, $\tau=0.0005$.]
	{
		\includegraphics[width=0.22\columnwidth]{mypreprocess/cedori/13432ced}
		\label{fig:13432ced}
	}
	\subfigure[Image 11 after TV denoising. $\lambda=1.0$, $\epsilon=1\times10^{-3}$, $\gamma=20$.]
	{
		\includegraphics[width=0.22\columnwidth]{mypreprocess/tv/13432graytv}
		\label{fig:13432tv}
	}
	\subfigure[Image 11 after Proposed Pre-processing scheme. $\kappa_1=0.17$, $\kappa_2=0.55$, $R_1=50$, $R_2=225$,$N=50$.]
	{
		\includegraphics[width=0.22\columnwidth]{mypreprocess/proposed/13432ced}
		\label{fig:13432prop}
	}
	
	\subfigure[Original segmentation.]
	{
		\includegraphics[width=0.22\columnwidth]{mypreprocess/origseg/def13432}
		\label{fig:13432noneseg}
	}
	\subfigure[CEDORI segmentation.]
	{
		\includegraphics[width=0.22\columnwidth]{mypreprocess/cedseg/def13432}
		\label{fig:13432cedseg}
	}
	\subfigure[Total Variation Denoising segmentation.]
	{
		\includegraphics[width=0.22\columnwidth]{mypreprocess/tvseg/def13432}
		\label{fig:13432tvseg}
	}
	\subfigure[Proposed Pre-process segmentation.]
	{
		\includegraphics[width=0.22\columnwidth]{mypreprocess/proposedseg/def13432}
		\label{fig:13432propseg}
	}
	\caption{Image 11 from test set \Cref{AppendixA} pre-processing segmentation results. $\mu=1$, $\lambda_0=10$, $\lambda_1=1$.}
	\label{fig:preprocessschemeresult13432}
\end{figure}

%%%%%%%%%%%%%%%%%%%%%%%%%%%%%%%%%%%%%%%%%%%%%%%%%%%%%%%
% 13438
\begin{figure}[!h]
	\centering
	\subfigure[Image 12 without any pre-processing.]
	{
		\includegraphics[width=0.22\columnwidth]{mypreprocess/orig/13438gray}
		\label{fig:13438none}
	}
	\subfigure[Image 12 after CEDORI. $\sigma=0.5$, $\rho=1.0$, $n=10$, $\tau=0.0005$.]
	{
		\includegraphics[width=0.22\columnwidth]{mypreprocess/cedori/13438ced}
		\label{fig:13438ced}
	}
	\subfigure[Image 12 after TV denoising. $\lambda=1.0$, $\epsilon=1\times10^{-3}$, $\gamma=20$.]
	{
		\includegraphics[width=0.22\columnwidth]{mypreprocess/tv/13438graytv}
		\label{fig:13438tv}
	}
	\subfigure[Image 12 after Proposed Pre-processing scheme. $\kappa_1=0.17$, $\kappa_2=0.55$, $R_1=50$, $R_2=225$,$N=50$.]
	{
		\includegraphics[width=0.22\columnwidth]{mypreprocess/proposed/13438ced}
		\label{fig:13438prop}
	}
	
	\subfigure[Original segmentation.]
	{
		\includegraphics[width=0.22\columnwidth]{mypreprocess/origseg/def13438}
		\label{fig:13438noneseg}
	}
	\subfigure[CEDORI segmentation.]
	{
		\includegraphics[width=0.22\columnwidth]{mypreprocess/cedseg/def13438}
		\label{fig:13438cedseg}
	}
	\subfigure[Total Variation Denoising segmentation.]
	{
		\includegraphics[width=0.22\columnwidth]{mypreprocess/tvseg/def13438}
		\label{fig:13438tvseg}
	}
	\subfigure[Proposed Pre-process segmentation.]
	{
		\includegraphics[width=0.22\columnwidth]{mypreprocess/proposedseg/def13438}
		\label{fig:13438propseg}
	}
	\caption{Image 12 from test set \Cref{AppendixA} pre-processing segmentation results. $\mu=1$, $\lambda_0=10$, $\lambda_1=1$.}
	\label{fig:preprocessschemeresult13438}
\end{figure}

%%%%%%%%%%%%%%%%%%%%%%%%%%%%%%%%%%%%%%%%%%%%%%%%%%%%%%%
% 13899
\begin{figure}[!h]
	\centering
	\subfigure[Image 13 without any pre-processing.]
	{
		\includegraphics[width=0.22\columnwidth]{mypreprocess/orig/13899gray}
		\label{fig:13899none}
	}
	\subfigure[Image 13 after CEDORI. $\sigma=0.5$, $\rho=1.0$, $n=20$, $\tau=0.0001$.]
	{
		\includegraphics[width=0.22\columnwidth]{mypreprocess/cedori/13899ced}
		\label{fig:13899ced}
	}
	\subfigure[Image 13 after TV denoising. $\lambda=1.0$, $\epsilon=1\times10^{-3}$, $\gamma=20$.]
	{
		\includegraphics[width=0.22\columnwidth]{mypreprocess/tv/13899graytv}
		\label{fig:13899tv}
	}
	\subfigure[Image 13 after Proposed Pre-processing scheme. $\kappa_1=0.17$, $\kappa_2=0.55$, $R_1=50$, $R_2=225$,$N=50$.]
	{
		\includegraphics[width=0.22\columnwidth]{mypreprocess/proposed/13899ced}
		\label{fig:13899prop}
	}
	
	\subfigure[Original segmentation.]
	{
		\includegraphics[width=0.22\columnwidth]{mypreprocess/origseg/def13899}
		\label{fig:13899noneseg}
	}
	\subfigure[CEDORI segmentation.]
	{
		\includegraphics[width=0.22\columnwidth]{mypreprocess/cedseg/def13899}
		\label{fig:13899cedseg}
	}
	\subfigure[Total Variation Denoising segmentation.]
	{
		\includegraphics[width=0.22\columnwidth]{mypreprocess/tvseg/def13899}
		\label{fig:13899tvseg}
	}
	\subfigure[Proposed Pre-process segmentation.]
	{
		\includegraphics[width=0.22\columnwidth]{mypreprocess/proposedseg/def13899}
		\label{fig:13899propseg}
	}
	\caption{Image 13 from test set \Cref{AppendixA} pre-processing segmentation results. $\mu=1$, $\lambda_0=10$, $\lambda_1=1$.}
	\label{fig:preprocessschemeresult13899}
\end{figure}

%%%%%%%%%%%%%%%%%%%%%%%%%%%%%%%%%%%%%%%%%%%%%%%%%%%%%%%
% 13901
\begin{figure}[!h]
	\centering
	\subfigure[Image 14 without any pre-processing.]
	{
		\includegraphics[width=0.22\columnwidth]{mypreprocess/orig/13901gray}
		\label{fig:13901none}
	}
	\subfigure[Image 14 after CEDORI. $\sigma=1.0$, $\rho=1.0$, $n=5$, $\tau=0.0001$.]
	{
		\includegraphics[width=0.22\columnwidth]{mypreprocess/cedori/13901ced}
		\label{fig:13901ced}
	}
	\subfigure[Image 14 after TV denoising. $\lambda=1.0$, $\epsilon=1\times10^{-3}$, $\gamma=20$.]
	{
		\includegraphics[width=0.22\columnwidth]{mypreprocess/tv/13901graytv}
		\label{fig:13901tv}
	}
	\subfigure[Image 14 after Proposed Pre-processing scheme. $\kappa_1=0.17$, $\kappa_2=0.55$, $R_1=50$, $R_2=225$,$N=50$.]
	{
		\includegraphics[width=0.22\columnwidth]{mypreprocess/proposed/13901ced}
		\label{fig:13901prop}
	}
	
	\subfigure[Original segmentation.]
	{
		\includegraphics[width=0.22\columnwidth]{mypreprocess/origseg/def13901}
		\label{fig:13901noneseg}
	}
	\subfigure[CEDORI segmentation.]
	{
		\includegraphics[width=0.22\columnwidth]{mypreprocess/cedseg/def13901}
		\label{fig:13901cedseg}
	}
	\subfigure[Total Variation Denoising segmentation.]
	{
		\includegraphics[width=0.22\columnwidth]{mypreprocess/tvseg/def13901}
		\label{fig:13901tvseg}
	}
	\subfigure[Proposed Pre-process segmentation.]
	{
		\includegraphics[width=0.22\columnwidth]{mypreprocess/proposedseg/def13901}
		\label{fig:13901propseg}
	}
	\caption{Image 14 from test set \Cref{AppendixA} pre-processing segmentation results. $\mu=1$, $\lambda_0=3$, $\lambda_1=1$.}
	\label{fig:preprocessschemeresult13901}
\end{figure}

%%%%%%%%%%%%%%%%%%%%%%%%%%%%%%%%%%%%%%%%%%%%%%%%%%%%%%%
% 21749
\begin{figure}[!h]
	\centering
	\subfigure[Image 15 without any pre-processing.]
	{
		\includegraphics[width=0.22\columnwidth]{mypreprocess/orig/21749gray}
		\label{fig:21749none}
	}
	\subfigure[Image 15 after CEDORI. $\sigma=1.0$, $\rho=1.0$, $n=20$, $\tau=0.0005$.]
	{
		\includegraphics[width=0.22\columnwidth]{mypreprocess/cedori/21749ced}
		\label{fig:21749ced}
	}
	\subfigure[Image 15 after TV denoising. $\lambda=1.0$, $\epsilon=1\times10^{-3}$, $\gamma=20$.]
	{
		\includegraphics[width=0.22\columnwidth]{mypreprocess/tv/21749graytv}
		\label{fig:21749tv}
	}
	\subfigure[Image 15 after Proposed Pre-processing scheme. $\kappa_1=0.17$, $\kappa_2=0.65$, $R_1=12$, $R_2=225$,$N=12$.]
	{
		\includegraphics[width=0.22\columnwidth]{mypreprocess/proposed/21749ced}
		\label{fig:21749prop}
	}
	
	\subfigure[Original segmentation.]
	{
		\includegraphics[width=0.22\columnwidth]{mypreprocess/origseg/def21749}
		\label{fig:21749noneseg}
	}
	\subfigure[CEDORI segmentation.]
	{
		\includegraphics[width=0.22\columnwidth]{mypreprocess/cedseg/def21749}
		\label{fig:21749cedseg}
	}
	\subfigure[Total Variation Denoising segmentation.]
	{
		\includegraphics[width=0.22\columnwidth]{mypreprocess/tvseg/def21749}
		\label{fig:21749tvseg}
	}
	\subfigure[Proposed Pre-process segmentation.]
	{
		\includegraphics[width=0.22\columnwidth]{mypreprocess/proposedseg/def21749}
		\label{fig:21749propseg}
	}
	\caption{Image 15 from test set \Cref{AppendixA} pre-processing segmentation results. $\mu=1$, $\lambda_0=10$, $\lambda_1=1$.}
	\label{fig:preprocessschemeresult21749}
\end{figure}

%%%%%%%%%%%%%%%%%%%%%%%%%%%%%%%%%%%%%%%%%%%%%%%%%%%%%%%
% 21759
\begin{figure}[!h]
	\centering
	\subfigure[Image 16 without any pre-processing.]
	{
		\includegraphics[width=0.22\columnwidth]{mypreprocess/orig/21759gray}
		\label{fig:21759none}
	}
	\subfigure[Image 16 after CEDORI. $\sigma=3.0$, $\rho=1.0$, $n=20$, $\tau=0.0005$.]
	{
		\includegraphics[width=0.22\columnwidth]{mypreprocess/cedori/21759ced}
		\label{fig:21759ced}
	}
	\subfigure[Image 16 after TV denoising. $\lambda=1.0$, $\epsilon=1\times10^{-3}$, $\gamma=20$.]
	{
		\includegraphics[width=0.22\columnwidth]{mypreprocess/tv/21759graytv}
		\label{fig:21759tv}
	}
	\subfigure[Image 16 after Proposed Pre-processing scheme. $\kappa_1=0.17$, $\kappa_2=0.55$, $R_1=50$, $R_2=225$,$N=50$.]
	{
		\includegraphics[width=0.22\columnwidth]{mypreprocess/proposed/21759ced}
		\label{fig:21759prop}
	}
	
	\subfigure[Original segmentation.]
	{
		\includegraphics[width=0.22\columnwidth]{mypreprocess/origseg/def21759}
		\label{fig:21759noneseg}
	}
	\subfigure[CEDORI segmentation.]
	{
		\includegraphics[width=0.22\columnwidth]{mypreprocess/cedseg/def21759}
		\label{fig:21759cedseg}
	}
	\subfigure[Total Variation Denoising segmentation.]
	{
		\includegraphics[width=0.22\columnwidth]{mypreprocess/tvseg/def21759}
		\label{fig:21759tvseg}
	}
	\subfigure[Proposed Pre-process segmentation.]
	{
		\includegraphics[width=0.22\columnwidth]{mypreprocess/proposedseg/def21759}
		\label{fig:21759propseg}
	}
	\caption{Image 16 from test set \Cref{AppendixA} pre-processing segmentation results. $\mu=1$, $\lambda_0=10$, $\lambda_1=1$.}
	\label{fig:preprocessschemeresult21759}
\end{figure}

%%%%%%%%%%%%%%%%%%%%%%%%%%%%%%%%%%%%%%%%%%%%%%%%%%%%%%%
% 32140
\begin{figure}[!h]
	\centering
	\subfigure[Image 17 without any pre-processing.]
	{
		\includegraphics[width=0.22\columnwidth]{mypreprocess/orig/32140gray}
		\label{fig:32140none}
	}
	\subfigure[Image 17 after CEDORI. $\sigma=3.0$, $\rho=3.0$, $n=20$, $\tau=0.0005$.]
	{
		\includegraphics[width=0.22\columnwidth]{mypreprocess/cedori/32140ced}
		\label{fig:32140ced}
	}
	\subfigure[Image 17 after TV denoising. $\lambda=1.0$, $\epsilon=1\times10^{-3}$, $\gamma=20$.]
	{
		\includegraphics[width=0.22\columnwidth]{mypreprocess/tv/32140graytv}
		\label{fig:32140tv}
	}
	\subfigure[Image 17 after Proposed Pre-processing scheme. $\kappa_1=0.17$, $\kappa_2=0.55$, $R_1=50$, $R_2=225$,$N=50$.]
	{
		\includegraphics[width=0.22\columnwidth]{mypreprocess/proposed/32140ced}
		\label{fig:32140prop}
	}
	
	\subfigure[Original segmentation.]
	{
		\includegraphics[width=0.22\columnwidth]{mypreprocess/origseg/def32140}
		\label{fig:32140noneseg}
	}
	\subfigure[CEDORI segmentation.]
	{
		\includegraphics[width=0.22\columnwidth]{mypreprocess/cedseg/def32140}
		\label{fig:32140cedseg}
	}
	\subfigure[Total Variation Denoising segmentation.]
	{
		\includegraphics[width=0.22\columnwidth]{mypreprocess/tvseg/def32140}
		\label{fig:32140tvseg}
	}
	\subfigure[Proposed Pre-process segmentation.]
	{
		\includegraphics[width=0.22\columnwidth]{mypreprocess/proposedseg/def32140}
		\label{fig:32140propseg}
	}
	\caption{Image 17 from test set \Cref{AppendixA} pre-processing segmentation results. $\mu=1$, $\lambda_0=5$, $\lambda_1=1$.}
	\label{fig:preprocessschemeresult32140}
\end{figure}

%%%%%%%%%%%%%%%%%%%%%%%%%%%%%%%%%%%%%%%%%%%%%%%%%%%%%%%
% 35278
\begin{figure}[!h]
	\centering
	\subfigure[Image 18 without any pre-processing.]
	{
		\includegraphics[width=0.22\columnwidth]{mypreprocess/orig/35278gray}
		\label{fig:35278none}
	}
	\subfigure[Image 18 after CEDORI. $\sigma=1.0$, $\rho=1.0$, $n=20$, $\tau=0.0005$.]
	{
		\includegraphics[width=0.22\columnwidth]{mypreprocess/cedori/35278ced}
		\label{fig:35278ced}
	}
	\subfigure[Image 18 after TV denoising. $\lambda=1.0$, $\epsilon=1\times10^{-3}$, $\gamma=5$.]
	{
		\includegraphics[width=0.22\columnwidth]{mypreprocess/tv/35278graytv}
		\label{fig:35278tv}
	}
	\subfigure[Image 18 after Proposed Pre-processing scheme. $\kappa_1=0.17$, $\kappa_2=0.65$, $R_1=12$, $R_2=225$,$N=12$.]
	{
		\includegraphics[width=0.22\columnwidth]{mypreprocess/proposed/35278ced}
		\label{fig:35278prop}
	}
	
	\subfigure[Original segmentation.]
	{
		\includegraphics[width=0.22\columnwidth]{mypreprocess/origseg/def35278}
		\label{fig:35278noneseg}
	}
	\subfigure[CEDORI segmentation.]
	{
		\includegraphics[width=0.22\columnwidth]{mypreprocess/cedseg/def35278}
		\label{fig:35278cedseg}
	}
	\subfigure[Total Variation Denoising segmentation.]
	{
		\includegraphics[width=0.22\columnwidth]{mypreprocess/tvseg/def35278}
		\label{fig:35278tvseg}
	}
	\subfigure[Proposed Pre-process segmentation.]
	{
		\includegraphics[width=0.22\columnwidth]{mypreprocess/proposedseg/def35278}
		\label{fig:35278propseg}
	}
	\caption{Image 18 from test set \Cref{AppendixA} pre-processing segmentation results. $\mu=1$, $\lambda_0=5$, $\lambda_1=1$.}
	\label{fig:preprocessschemeresult35278}
\end{figure}

%%%%%%%%%%%%%%%%%%%%%%%%%%%%%%%%%%%%%%%%%%%%%%%%%%%%%%%
% 37338
\begin{figure}[!h]
	\centering
	\subfigure[Image 19 without any pre-processing.]
	{
		\includegraphics[width=0.22\columnwidth]{mypreprocess/orig/37338gray}
		\label{fig:37338none}
	}
	\subfigure[Image 19 after CEDORI. $\sigma=0.5$, $\rho=5.0$, $n=50$, $\tau=0.0002$.]
	{
		\includegraphics[width=0.22\columnwidth]{mypreprocess/cedori/37338ced}
		\label{fig:37338ced}
	}
	\subfigure[Image 19 after TV denoising. $\lambda=1.0$, $\epsilon=1\times10^{-3}$, $\gamma=20$.]
	{
		\includegraphics[width=0.22\columnwidth]{mypreprocess/tv/37338graytv}
		\label{fig:37338tv}
	}
	\subfigure[Image 19 after Proposed Pre-processing scheme. $\kappa_1=0.17$, $\kappa_2=0.65$, $R_1=12$, $R_2=225$,$N=12$.]
	{
		\includegraphics[width=0.22\columnwidth]{mypreprocess/proposed/37338ced}
		\label{fig:37338prop}
	}
	
	\subfigure[Original segmentation.]
	{
		\includegraphics[width=0.22\columnwidth]{mypreprocess/origseg/def37338}
		\label{fig:373388noneseg}
	}
	\subfigure[CEDORI segmentation.]
	{
		\includegraphics[width=0.22\columnwidth]{mypreprocess/cedseg/def37338}
		\label{fig:37338cedseg}
	}
	\subfigure[Total Variation Denoising segmentation.]
	{
		\includegraphics[width=0.22\columnwidth]{mypreprocess/tvseg/def37338}
		\label{fig:37338tvseg}
	}
	\subfigure[Proposed Pre-process segmentation.]
	{
		\includegraphics[width=0.22\columnwidth]{mypreprocess/proposedseg/def37338}
		\label{fig:37338propseg}
	}
	\caption{Image 19 from test set \Cref{AppendixA} pre-processing segmentation results. $\mu=1$, $\lambda_0=6$, $\lambda_1=2$.}
	\label{fig:preprocessschemeresult37338}
\end{figure}

%%%%%%%%%%%%%%%%%%%%%%%%%%%%%%%%%%%%%%%%%%%%%%%%%%%%%%%
% 37339
\begin{figure}[!h]
	\centering
	\subfigure[Image 20 without any pre-processing.]
	{
		\includegraphics[width=0.22\columnwidth]{mypreprocess/orig/37339gray}
		\label{fig:37339none}
	}
	\subfigure[Image 20 after CEDORI. $\sigma=0.5$, $\rho=5.0$, $n=50$, $\tau=0.0002$.]
	{
		\includegraphics[width=0.22\columnwidth]{mypreprocess/cedori/37339ced}
		\label{fig:37339ced}
	}
	\subfigure[Image 20 after TV denoising. $\lambda=1.0$, $\epsilon=1\times10^{-3}$, $\gamma=20$.]
	{
		\includegraphics[width=0.22\columnwidth]{mypreprocess/tv/37339graytv}
		\label{fig:37339tv}
	}
	\subfigure[Image 20 after Proposed Pre-processing scheme. $\kappa_1=0.17$, $\kappa_2=0.65$, $R_1=12$, $R_2=225$,$N=12$.]
	{
		\includegraphics[width=0.22\columnwidth]{mypreprocess/proposed/37339ced}
		\label{fig:37339prop}
	}
	
	\subfigure[Original segmentation.]
	{
		\includegraphics[width=0.22\columnwidth]{mypreprocess/origseg/def37339}
		\label{fig:37339noneseg}
	}
	\subfigure[CEDORI segmentation.]
	{
		\includegraphics[width=0.22\columnwidth]{mypreprocess/cedseg/def37339}
		\label{fig:37339cedseg}
	}
	\subfigure[Total Variation Denoising segmentation.]
	{
		\includegraphics[width=0.22\columnwidth]{mypreprocess/tvseg/def37339}
		\label{fig:37339tvseg}
	}
	\subfigure[Proposed Pre-process segmentation.]
	{
		\includegraphics[width=0.22\columnwidth]{mypreprocess/proposedseg/def37339}
		\label{fig:37339propseg}
	}
	\caption{Image 20 from test set \Cref{AppendixA} pre-processing segmentation results. $\mu=1$, $\lambda_0=6$, $\lambda_1=2$.}
	\label{fig:preprocessschemeresult37339}
\end{figure}

%%%%%%%%%%%%%%%%%%%%%%%%%%%%%%%%%%%%%%%%%%%%%%%%%%%%%%%
% 38974
\begin{figure}[!h]
	\centering
	\subfigure[Image 21 without any pre-processing.]
	{
		\includegraphics[width=0.22\columnwidth]{mypreprocess/orig/38974gray}
		\label{fig:38974none}
	}
	\subfigure[Image 21 after CEDORI. $\sigma=0.5$, $\rho=1.0$, $n=15$, $\tau=0.0005$.]
	{
		\includegraphics[width=0.22\columnwidth]{mypreprocess/cedori/38974ced}
		\label{fig:38974ced}
	}
	\subfigure[Image 21 after TV denoising. $\lambda=1.0$, $\epsilon=1\times10^{-3}$, $\gamma=5$.]
	{
		\includegraphics[width=0.22\columnwidth]{mypreprocess/tv/38974graytv}
		\label{fig:38974tv}
	}
	\subfigure[Image 21 after Proposed Pre-processing scheme. $\kappa_1=0.17$, $\kappa_2=0.55$, $R_1=50$, $R_2=225$,$N=50$.]
	{
		\includegraphics[width=0.22\columnwidth]{mypreprocess/proposed/38974ced}
		\label{fig:38974prop}
	}
	
	\subfigure[Original segmentation.]
	{
		\includegraphics[width=0.22\columnwidth]{mypreprocess/origseg/def38974}
		\label{fig:38974noneseg}
	}
	\subfigure[CEDORI segmentation.]
	{
		\includegraphics[width=0.22\columnwidth]{mypreprocess/cedseg/def38974}
		\label{fig:38974cedseg}
	}
	\subfigure[Total Variation Denoising segmentation.]
	{
		\includegraphics[width=0.22\columnwidth]{mypreprocess/tvseg/def38974}
		\label{fig:38974tvseg}
	}
	\subfigure[Proposed Pre-process segmentation.]
	{
		\includegraphics[width=0.22\columnwidth]{mypreprocess/proposedseg/def38974}
		\label{fig:38974propseg}
	}
	\caption{Image 21 from test set \Cref{AppendixA} pre-processing segmentation results. $\mu=1$, $\lambda_0=300$, $\lambda_1=30$.}
	\label{fig:preprocessschemeresult38974}
\end{figure}

%%%%%%%%%%%%%%%%%%%%%%%%%%%%%%%%%%%%%%%%%%%%%%%%%%%%%%%
% 40217
\begin{figure}[!h]
	\centering
	\subfigure[Image 22 without any pre-processing.]
	{
		\includegraphics[width=0.22\columnwidth]{mypreprocess/orig/40217gray}
		\label{fig:40217none}
	}
	\subfigure[Image 22 after CEDORI. $\sigma=1.0$, $\rho=1.0$, $n=20$, $\tau=0.0005$.]
	{
		\includegraphics[width=0.22\columnwidth]{mypreprocess/cedori/40217ced}
		\label{fig:40217ced}
	}
	\subfigure[Image 22 after TV denoising. $\lambda=1.0$, $\epsilon=1\times10^{-3}$, $\gamma=20$.]
	{
		\includegraphics[width=0.22\columnwidth]{mypreprocess/tv/40217graytv}
		\label{fig:40217tv}
	}
	\subfigure[Image 22 after Proposed Pre-processing scheme. $\kappa_1=0.17$, $\kappa_2=0.55$, $R_1=50$, $R_2=225$,$N=50$.]
	{
		\includegraphics[width=0.22\columnwidth]{mypreprocess/proposed/40217ced}
		\label{fig:40217prop}
	}
	
	\subfigure[Original segmentation.]
	{
		\includegraphics[width=0.22\columnwidth]{mypreprocess/origseg/def40217}
		\label{fig:40217noneseg}
	}
	\subfigure[CEDORI segmentation.]
	{
		\includegraphics[width=0.22\columnwidth]{mypreprocess/cedseg/def40217}
		\label{fig:40217cedseg}
	}
	\subfigure[Total Variation Denoising segmentation.]
	{
		\includegraphics[width=0.22\columnwidth]{mypreprocess/tvseg/def40217}
		\label{fig:40217tvseg}
	}
	\subfigure[Proposed Pre-process segmentation.]
	{
		\includegraphics[width=0.22\columnwidth]{mypreprocess/proposedseg/def40217}
		\label{fig:40217propseg}
	}
	\caption{Image 22 from test set \Cref{AppendixA} pre-processing segmentation results. $\mu=1$, $\lambda_0=5$, $\lambda_1=1$.}
	\label{fig:preprocessschemeresult40217}
\end{figure}

%%%%%%%%%%%%%%%%%%%%%%%%%%%%%%%%%%%%%%%%%%%%%%%%%%%%%%%
% 40968
\begin{figure}[!h]
	\centering
	\subfigure[Image 23 without any pre-processing.]
	{
		\includegraphics[width=0.22\columnwidth]{mypreprocess/orig/40968gray}
		\label{fig:40968none}
	}
	\subfigure[Image 23 after CEDORI. $\sigma=1.0$, $\rho=2.0$, $n=15$, $\tau=0.0005$.]
	{
		\includegraphics[width=0.22\columnwidth]{mypreprocess/cedori/40968ced}
		\label{fig:40968ced}
	}
	\subfigure[Image 23 after TV denoising. $\lambda=1.0$, $\epsilon=1\times10^{-3}$, $\gamma=20$.]
	{
		\includegraphics[width=0.22\columnwidth]{mypreprocess/tv/40968graytv}
		\label{fig:40968tv}
	}
	\subfigure[Image 23 after Proposed Pre-processing scheme. $\kappa_1=0.17$, $\kappa_2=0.55$, $R_1=50$, $R_2=225$,$N=50$.]
	{
		\includegraphics[width=0.22\columnwidth]{mypreprocess/proposed/40968ced}
		\label{fig:40968prop}
	}
	
	\subfigure[Original segmentation.]
	{
		\includegraphics[width=0.22\columnwidth]{mypreprocess/origseg/def40968}
		\label{fig:40968noneseg}
	}
	\subfigure[CEDORI segmentation.]
	{
		\includegraphics[width=0.22\columnwidth]{mypreprocess/cedseg/def40968}
		\label{fig:40968cedseg}
	}
	\subfigure[Total Variation Denoising segmentation.]
	{
		\includegraphics[width=0.22\columnwidth]{mypreprocess/tvseg/def40968}
		\label{fig:40968tvseg}
	}
	\subfigure[Proposed Pre-process segmentation.]
	{
		\includegraphics[width=0.22\columnwidth]{mypreprocess/proposedseg/def40968}
		\label{fig:40968propseg}
	}
	\caption{Image 23 from test set \Cref{AppendixA} pre-processing segmentation results. $\mu=1$, $\lambda_0=5$, $\lambda_1=1$.}
	\label{fig:preprocessschemeresult40968}
\end{figure}

%%%%%%%%%%%%%%%%%%%%%%%%%%%%%%%%%%%%%%%%%%%%%%%%%%%%%%%
% 41066
\begin{figure}[!h]
	\centering
	\subfigure[Image 24 without any pre-processing.]
	{
		\includegraphics[width=0.22\columnwidth]{mypreprocess/orig/41066gray}
		\label{fig:41066none}
	}
	\subfigure[Image 24 after CEDORI. $\sigma=1.0$, $\rho=2.0$, $n=15$, $\tau=0.0001$.]
	{
		\includegraphics[width=0.22\columnwidth]{mypreprocess/cedori/41066ced}
		\label{fig:41066ced}
	}
	\subfigure[Image 24 after TV denoising. $\lambda=1.0$, $\epsilon=1\times10^{-3}$, $\gamma=20$.]
	{
		\includegraphics[width=0.22\columnwidth]{mypreprocess/tv/41066graytv}
		\label{fig:41066tv}
	}
	\subfigure[Image 24 after Proposed Pre-processing scheme. $\kappa_1=0.17$, $\kappa_2=0.55$, $R_1=50$, $R_2=225$,$N=50$.]
	{
		\includegraphics[width=0.22\columnwidth]{mypreprocess/proposed/41066ced}
		\label{fig:41066prop}
	}
	
	\subfigure[Original segmentation.]
	{
		\includegraphics[width=0.22\columnwidth]{mypreprocess/origseg/def41066}
		\label{fig:41066noneseg}
	}
	\subfigure[CEDORI segmentation.]
	{
		\includegraphics[width=0.22\columnwidth]{mypreprocess/cedseg/def41066}
		\label{fig:41066cedseg}
	}
	\subfigure[Total Variation Denoising segmentation.]
	{
		\includegraphics[width=0.22\columnwidth]{mypreprocess/tvseg/def41066}
		\label{fig:41066tvseg}
	}
	\subfigure[Proposed Pre-process segmentation.]
	{
		\includegraphics[width=0.22\columnwidth]{mypreprocess/proposedseg/def41066}
		\label{fig:41066propseg}
	}
	\caption{Image 24 from test set \Cref{AppendixA} pre-processing segmentation results. $\mu=1$, $\lambda_0=5$, $\lambda_1=1$.}
	\label{fig:preprocessschemeresult41066}
\end{figure}

%%%%%%%%%%%%%%%%%%%%%%%%%%%%%%%%%%%%%%%%%%%%%%%%%%%%%%%
% 42451
\begin{figure}[!h]
	\centering
	\subfigure[Image 25 without any pre-processing.]
	{
		\includegraphics[width=0.22\columnwidth]{mypreprocess/orig/42451gray}
		\label{fig:42451none}
	}
	\subfigure[Image 25 after CEDORI. $\sigma=0.5$, $\rho=1.0$, $n=15$, $\tau=0.0005$.]
	{
		\includegraphics[width=0.22\columnwidth]{mypreprocess/cedori/42451ced}
		\label{fig:42451ced}
	}
	\subfigure[Image 25 after TV denoising. $\lambda=1.0$, $\epsilon=1\times10^{-3}$, $\gamma=20$.]
	{
		\includegraphics[width=0.22\columnwidth]{mypreprocess/tv/42451graytv}
		\label{fig:42451tv}
	}
	\subfigure[Image 25 after Proposed Pre-processing scheme. $\kappa_1=0.17$, $\kappa_2=0.55$, $R_1=50$, $R_2=225$,$N=50$.]
	{
		\includegraphics[width=0.22\columnwidth]{mypreprocess/proposed/42451ced}
		\label{fig:42451prop}
	}
	
	\subfigure[Original segmentation.]
	{
		\includegraphics[width=0.22\columnwidth]{mypreprocess/origseg/def42451}
		\label{fig:42451noneseg}
	}
	\subfigure[CEDORI segmentation.]
	{
		\includegraphics[width=0.22\columnwidth]{mypreprocess/cedseg/def42451}
		\label{fig:42451cedseg}
	}
	\subfigure[Total Variation Denoising segmentation.]
	{
		\includegraphics[width=0.22\columnwidth]{mypreprocess/tvseg/def42451}
		\label{fig:42451tvseg}
	}
	\subfigure[Proposed Pre-process segmentation.]
	{
		\includegraphics[width=0.22\columnwidth]{mypreprocess/proposedseg/def42451}
		\label{fig:42451propseg}
	}
	\caption{Image 25 from test set \Cref{AppendixA} pre-processing segmentation results. $\mu=1$, $\lambda_0=5$, $\lambda_1=1$.}
	\label{fig:preprocessschemeresult42451}
\end{figure}

\clearpage
\begin{longtable}[!h]{|c|c|c|c|c|c|c|c|c|}
	\caption{Pre-processing Scheme Segmentation Results.} \label{tab:preprocessresults}\\
	\hline	Image	&	TP	&	TN	&	FP	&	FN	&	Precision	&	Recall	&	Accuracy	&	MCC	\\
	\hline	1-o	&	38030	&	25503	&	13	&	1990	&	0.999658	&	0.950275	&	0.969437	&	0.938453	\\
	\hline	1-c	&	37673	&	25510	&	6	&	2347	&	0.999841	&	0.941354	&	0.964096	&	0.928266	\\
	\hline	\rowcolor{bad} 1-t	&	37520	&	25511	&	5	&	2500	&	0.999867	&	0.937531	&	0.961777	&	0.923879	\\
	\hline	\rowcolor{closest} 1-p	&	39215	&	25225	&	291	&	805	&	0.992634	&	0.979885	&	0.983276	&	0.965088	\\
	
	\hline \rowcolor{closest} 2-o	&	21312	&	39624	&	3966	&	634	&	0.843105	&	0.971111	&	0.929810	&	0.853334	\\
	\hline	2-c	&	21275	&	39136	&	4454	&	671	&	0.826888	&	0.969425	&	0.921799	&	0.838151	\\
	\hline	\rowcolor{bad} 2-t	&	21250	&	36946	&	6644	&	696	&	0.761813	&	0.968286	&	0.888000	&	0.778750	\\
	\hline	2-p	&	20292	&	40378	&	3212	&	1654	&	0.863342	&	0.924633	&	0.925751	&	0.837361	\\
	
	\hline	\rowcolor{bad} 3-o	&	11674	&	49067	&	4795	&	0	&	0.708847	&	1.000000	&	0.926834	&	0.803581	\\
	\hline	3-c	&	10105	&	53253	&	2178	&	0	&	0.822682	&	1.000000	&	0.966766	&	0.889020	\\
	\hline	3-t	&	9491	&	55234	&	197	&	614	&	0.979666	&	0.939238	&	0.987625	&	0.952035	\\
	\hline	\rowcolor{closest} 3-p	&	11674	&	53280	&	582	&	0	&	0.952513	&	1.000000	&	0.991119	&	0.970681	\\
	
	\hline	\rowcolor{bad} 4-o	&	15593	&	45595	&	4348	&	0	&	0.781957	&	1.000000	&	0.933655	&	0.844914	\\
	\hline	4-c	&	15593	&	46990	&	2953	&	0	&	0.840774	&	1.000000	&	0.954941	&	0.889416	\\
	\hline	\rowcolor{closest} 4-t	&	15588	&	47537	&	2406	&	5	&	0.866289	&	0.999679	&	0.963211	&	0.907842	\\
	\hline	4-p	&	15593	&	47397	&	2546	&	0	&	0.859639	&	1.000000	&	0.961151	&	0.903226	\\
	
	\hline	\rowcolor{bad} 5-o	&	24669	&	33229	&	7638	&	0	&	0.763581	&	1.000000	&	0.883453	&	0.787952	\\
	\hline	5-c	&	24669	&	35234	&	5633	&	0	&	0.814105	&	1.000000	&	0.914047	&	0.837789	\\
	\hline	5-t	&	24669	&	35134	&	5733	&	0	&	0.811427	&	1.000000	&	0.912521	&	0.835222	\\
	\hline	\rowcolor{closest} 5-p	&	24669	&	35801	&	5066	&	0	&	0.829628	&	1.000000	&	0.922699	&	0.852517	\\
	
	\hline	\rowcolor{bad} 6-o	&	37231	&	25337	&	418	&	2550	&	0.988897	&	0.935899	&	0.954712	&	0.908495	\\
	\hline	6-c	&	37685	&	25459	&	296	&	2096	&	0.992207	&	0.947312	&	0.963501	&	0.925927	\\
	\hline	6-t	&	38189	&	25218	&	537	&	1592	&	0.986133	&	0.959981	&	0.967514	&	0.932921	\\
	\hline	\rowcolor{closest} 6-p	&	38072	&	25560	&	195	&	1709	&	0.994904	&	0.957040	&	0.970947	&	0.940811	\\
	
	\hline	\rowcolor{bad} 7-o	&	14955	&	46580	&	771	&	3230	&	0.950973	&	0.822381	&	0.938950	&	0.845166	\\
	\hline	\rowcolor{closest} 7-c	&	16706	&	46374	&	977	&	1479	&	0.944749	&	0.918669	&	0.962524	&	0.905905	\\
	\hline	7-t	&	17704	&	44904	&	2447	&	481	&	0.878567	&	0.973550	&	0.955322	&	0.894514	\\
	\hline	7-p	&	16833	&	45609	&	1742	&	1352	&	0.906218	&	0.925653	&	0.952789	&	0.883128	\\
	
	\hline	\rowcolor{bad} 8-o	&	38556	&	22106	&	52	&	4822	&	0.998653	&	0.888838	&	0.925629	&	0.852381	\\
	\hline	8-c	&	39639	&	22114	&	44	&	3739	&	0.998891	&	0.913804	&	0.942276	&	0.882573	\\
	\hline	8-t	&	38925	&	22153	&	5	&	4453	&	0.999872	&	0.897344	&	0.931976	&	0.864207	\\
	\hline	\rowcolor{closest} 8-p	&	40658	&	22123	&	35	&	2720	&	0.999140	&	0.937295	&	0.957962	&	0.912393	\\
	
	\hline \rowcolor{bad} 9-o	&	12324	&	50443	&	2767	&	2	&	0.816646	&	0.999838	&	0.957748	&	0.879778	\\
	\hline	9-c	&	12312	&	51825	&	1385	&	14	&	0.898883	&	0.998864	&	0.978653	&	0.934988	\\
	\hline	9-t	&	12301	&	51571	&	1639	&	25	&	0.882425	&	0.997972	&	0.974609	&	0.923572	\\
	\hline	\rowcolor{closest} 9-p	&	12258	&	52314	&	896	&	68	&	0.931884	&	0.994483	&	0.985291	&	0.953825	\\
	
	\hline	\rowcolor{closest} 10-o	&	34387	&	31026	&	122	&	1	&	0.996465	&	0.999971	&	0.998123	&	0.996243	\\
	\hline	10-c	&	34376	&	30955	&	193	&	12	&	0.994417	&	0.999651	&	0.996872	&	0.993742	\\
	\hline	\rowcolor{bad} 10-t	&	34372	&	30809	&	339	&	16	&	0.990234	&	0.999535	&	0.994583	&	0.989183	\\
	\hline	10-p	&	34374	&	30829	&	319	&	14	&	0.990805	&	0.999593	&	0.994919	&	0.989851	\\
	
	\hline	11-o	&	24010	&	34854	&	5998	&	674	&	0.800120	&	0.972695	&	0.898193	&	0.803199	\\
	\hline	11-c	&	24060	&	34824	&	6028	&	624	&	0.799654	&	0.974720	&	0.898499	&	0.804291	\\
	\hline	\rowcolor{bad} 11-t	&	24173	&	33511	&	7341	&	511	&	0.767056	&	0.979298	&	0.880188	&	0.775454	\\
	\hline	\rowcolor{closest} 11-p	&	22670	&	37071	&	3781	&	2014	&	0.857056	&	0.918409	&	0.911575	&	0.815628	\\

	\hline	12-o	&	19444	&	39384	&	6577	&	131	&	0.747243	&	0.993308	&	0.897644	&	0.795294	\\
	\hline	12-c	&	19382	&	40090	&	5871	&	193	&	0.767513	&	0.990140	&	0.907471	&	0.811032	\\
	\hline	\rowcolor{bad} 12-t	&	19171	&	39115	&	6846	&	404	&	0.736864	&	0.979361	&	0.889374	&	0.776794	\\
	\hline	\rowcolor{closest} 12-p	&	15920	&	44976	&	985	&	3655	&	0.941733	&	0.813282	&	0.929199	&	0.828371	\\

	\hline	\rowcolor{closest} 13-o	&	10441	&	42330	&	12619	&	146	&	0.452775	&	0.986210	&	0.805222	&	0.583053	\\
	\hline	13-c	&	10442	&	42277	&	12672	&	145	&	0.451761	&	0.986304	&	0.804428	&	0.582072	\\
	\hline	\rowcolor{bad} 13-t	&	9583	&	39968	&	14981	&	1004	&	0.390124	&	0.905167	&	0.756088	&	0.480901	\\
	\hline	13-p	&	3818	&	46078	&	8871	&	6769	&	0.300891	&	0.360631	&	0.761353	&	0.185529	\\

	\hline	\rowcolor{closest} 14-o	&	12749	&	44255	&	7640	&	892	&	0.625288	&	0.934609	&	0.869812	&	0.690498	\\
	\hline	14-c	&	12758	&	44119	&	7776	&	883	&	0.621311	&	0.935269	&	0.867874	&	0.687447	\\
	\hline	14-t	&	10713	&	43713	&	8182	&	2928	&	0.566975	&	0.785353	&	0.830475	&	0.562565	\\
	\hline	\rowcolor{bad} 14-p	&	13338	&	39865	&	12030	&	303	&	0.525781	&	0.977788	&	0.811813	&	0.621765	\\
	
	\hline	15-o	&	3370	&	59038	&	3128	&	0	&	0.518621	&	1.000000	&	0.952271	&	0.701802	\\
	\hline	15-c	&	3370	&	58988	&	3178	&	0	&	0.514661	&	1.000000	&	0.951508	&	0.698821	\\
	\hline	\rowcolor{bad} 15-t	&	3289	&	55293	&	6873	&	81	&	0.323657	&	0.975964	&	0.893890	&	0.528042	\\
	\hline	\rowcolor{closest} 15-p	&	2316	&	62153	&	13	&	1054	&	0.994418	&	0.687240	&	0.983719	&	0.819597	\\

	\hline	\rowcolor{bad} 16-o	&	7443	&	53278	&	4815	&	0	&	0.607195	&	1.000000	&	0.926529	&	0.746236	\\
	\hline	16-c	&	7388	&	57196	&	897	&	55	&	0.891732	&	0.992611	&	0.985474	&	0.932971	\\
	\hline	\rowcolor{closest} 16-t	&	7365	&	57307	&	786	&	78	&	0.903570	&	0.989520	&	0.986816	&	0.938376	\\
	\hline	16-p	&	7228	&	57294	&	799	&	215	&	0.900461	&	0.971114	&	0.984528	&	0.926545	\\

	\hline	17-o	&	9824	&	54780	&	358	&	574	&	0.964840	&	0.944797	&	0.985779	&	0.946353	\\
	\hline	17-c	&	10398	&	49941	&	5197	&	0	&	0.666752	&	1.000000	&	0.920700	&	0.777115	\\
	\hline	\rowcolor{bad} 17-t	&	10398	&	47423	&	7715	&	0	&	0.574063	&	1.000000	&	0.882278	&	0.702666	\\
	\hline	\rowcolor{closest} 17-p	&	10064	&	55134	&	4	&	334	&	0.999603	&	0.967878	&	0.994843	&	0.980609	\\

	\hline	\rowcolor{bad} 18-o	&	31266	&	28291	&	84	&	5895	&	0.997321	&	0.841366	&	0.908768	&	0.831616	\\
	\hline	18-c	&	31543	&	28021	&	354	&	5618	&	0.988902	&	0.848820	&	0.908875	&	0.829087	\\
	\hline	18-t	&	32157	&	28158	&	217	&	5004	&	0.993297	&	0.865343	&	0.920334	&	0.850014	\\
	\hline	\rowcolor{closest} 18-p	&	28470	&	35580	&	377	&	1109	&	0.986931	&	0.962507	&	0.977325	&	0.954360	\\

	\hline	19-o	&	15285	&	46767	&	2533	&	951	&	0.857840	&	0.941426	&	0.946838	&	0.863586	\\
	\hline	\rowcolor{bad} 19-c	&	14435	&	47043	&	2257	&	1801	&	0.864786	&	0.889074	&	0.938080	&	0.835567	\\
	\hline	19-t	&	15359	&	47526	&	2459	&	192	&	0.861993	&	0.987654	&	0.959549	&	0.897312	\\
	\hline	\rowcolor{closest} 19-p	&	16195	&	49250	&	50	&	41	&	0.996922	&	0.997475	&	0.998611	&	0.996275	\\

	\hline	20-o	&	12628	&	50195	&	2224	&	489	&	0.850256	&	0.962720	&	0.958603	&	0.879548	\\
	\hline	20-c	&	12624	&	50209	&	2210	&	493	&	0.851018	&	0.962415	&	0.958755	&	0.879889	\\
	\hline	\rowcolor{bad} 20-t	&	13015	&	49425	&	2994	&	102	&	0.812980	&	0.992224	&	0.952759	&	0.870803	\\
	\hline	\rowcolor{closest} 20-p	&	12889	&	50599	&	1820	&	228	&	0.876266	&	0.982618	&	0.968750	&	0.909043	\\

	\hline	\rowcolor{closest} 21-o	&	11423	&	53657	&	372	&	84	&	0.968461	&	0.992700	&	0.993042	&	0.976311	\\
	\hline	21-c	&	10725	&	53839	&	190	&	782	&	0.982593	&	0.932041	&	0.985168	&	0.948192	\\
	\hline	21-t	&	10113	&	53340	&	689	&	1394	&	0.936216	&	0.878856	&	0.968216	&	0.888145	\\
	\hline	\rowcolor{bad} 21-p	&	9883	&	52360	&	1669	&	1624	&	0.855523	&	0.858869	&	0.949753	&	0.826708	\\

	\hline	\rowcolor{closest} 22-o	&	26975	&	36034	&	56	&	2471	&	0.997928	&	0.916084	&	0.961441	&	0.924093	\\
	\hline	22-c	&	26883	&	35756	&	334	&	2563	&	0.987728	&	0.912959	&	0.955795	&	0.912233	\\
	\hline	22-t	&	26684	&	35248	&	842	&	2762	&	0.969411	&	0.906201	&	0.945007	&	0.889781	\\
	\hline	\rowcolor{bad} 22-p	&	27007	&	34403	&	1687	&	2439	&	0.941207	&	0.917170	&	0.937042	&	0.872713	\\

	\hline	23-o	&	15978	&	48669	&	870	&	19	&	0.948362	&	0.998812	&	0.986435	&	0.964467	\\
	\hline	\rowcolor{closest} 23-c	&	15848	&	49274	&	265	&	149	&	0.983554	&	0.990686	&	0.993683	&	0.982935	\\
	\hline	23-t	&	15904	&	49184	&	355	&	93	&	0.978166	&	0.994186	&	0.993164	&	0.981635	\\
	\hline	\rowcolor{bad} 23-p	&	15852	&	49003	&	536	&	145	&	0.967293	&	0.990936	&	0.989609	&	0.972197	\\

	\hline	24-o	&	7859	&	43669	&	14006	&	2	&	0.359433	&	0.999746	&	0.786255	&	0.521556	\\
	\hline	\rowcolor{closest} 24-c	&	7808	&	52834	&	4841	&	53	&	0.617282	&	0.993258	&	0.925323	&	0.748597	\\
	\hline	24-t	&	7850	&	48994	&	8681	&	11	&	0.474865	&	0.998601	&	0.867371	&	0.634457	\\
	\hline	\rowcolor{bad} 24-p	&	502	&	46486	&	11189	&	7359	&	0.042939	&	0.063860	&	0.716980	&	-0.110446	\\

	\hline	25-o	&	28607	&	31722	&	307	&	4900	&	0.989382	&	0.853762	&	0.920547	&	0.849861	\\
	\hline	25-c	&	29110	&	31437	&	592	&	4397	&	0.980069	&	0.868774	&	0.923874	&	0.853820	\\
	\hline	\rowcolor{bad}25-t	&	29615	&	30347	&	1682	&	3892	&	0.946257	&	0.883845	&	0.914948	&	0.831958	\\
	\hline	\rowcolor{closest} 25-p	&	29825	&	31755	&	274	&	274	&	0.990897	&	0.890113	&	0.939636	&	0.884272\\
	\hline
\end{longtable}

\begin{longtable}[!h]{|c|c|c|c|c|c|c|}
	\caption{Overall Pre-processing Segmentation Efficiency.} \label{tab:overallpreprocessingsegmentationefficiency}\\
	\hline 
	\multirow{2}{*}{Scheme} & \multicolumn{2}{c|}{Precision} & \multicolumn{2}{c|}{Recall} & \multicolumn{2}{c|}{Accuracy} \\ 
	\hhline{~------}
	& Mean & Std. Dev. & Mean & Std. Dev & Mean & Std. Dev.  \\ 
	\hline	None	&	0.821354	&	0.186952	&	0.956262	&	0.054109	&	0.928629	&	0.052098	\\
	\hline	CED	&	0.844110	&	0.159666	&	0.958646	&	0.045520	&	0.939639	&	0.042925	\\
	\hline	\rowcolor{bad} TV	&	0.817916	&	0.203910	&	0.951100	&	0.057106	&	0.927104	&	0.056975	\\
	\hline	\rowcolor{closest} Proposed	&	0.859945	&	0.232488	&	0.883139	&	0.218280	&	0.940026	&	0.072479	\\
	
	\hline
\end{longtable}

\begin{figure}[!h]
	\centering
	\includegraphics[scale=0.98]{/mypreprocess/precisionvsrecall}
	\caption{Pre-processing scheme and methods segmentation precision against recall over test set.}
	\label{fig:proposedschemeprecisionvsrecall}
\end{figure}

\begin{figure}[!h]
	\centering
	\includegraphics[scale=0.98]{/mypreprocess/accuracy}
	\caption{Pre-processing scheme and methods segmentation accuracy over test set.}
	\label{fig:proposedschemeaccuracy}
\end{figure}

%----------------------------------------------------------------------------------------
%	SECTION 4
%----------------------------------------------------------------------------------------

\section{Discussion}
\label{sec:preprocessschemediscussion}

We have presented a scheme that adjusts the properties of an image such that more accurate segmentation can be obtained. Experiments performed on 2D fluorescence microscopy image data show that the proposed scheme outperforms the existing techniques. One limitation of the scheme is the tuning of the various parameters. The running time can also be reduced by using parallel computing or by optimizing algorithms used. From the precision versus recall plot in \Cref{fig:proposedschemeprecisionvsrecall}, we can see that all methods, even the original image after segmentation, produce very high precision and recall. This is due to the robustness of the Chan-Vese segmentation algorithm. However, over a large set we can see that the proposed method does allow for more accurate segmentation as seen in \Cref{tab:overallpreprocessingsegmentationefficiency}. From the accuracy plot in \Cref{fig:proposedschemeaccuracy}, we can see that all methods suffer from producing inconsistent results; however, due to the fact that they're so close to the ground truth, this is negligible, for the most part.

%----------------------------------------------------------------------------------------
%	SECTION 5
%----------------------------------------------------------------------------------------

\section{Conclusion}
\label{sec:preprocessschemeconclusion}

Fluorescence imaging is a low light and low contrast imaging technique. Therefore, segmentation algorithms often have to work on images that have low SNR. To reduce the inaccuracies in the segmentation step, the image is normally pre-processed to allow for enhanced visualisation and segmentation accuracy.

For the segmentation technique used, which is already very robust against low contrast, the pre-processing scheme is there to improve the quality of the boundary and remove segmentation artefacts. We have presented a novel pre-processing scheme, which is a hybrid of algorithms that are designed to combat the problems typically found in fluorescence images. We also design a novel contrast enhancing function based on the non-linear remapping from a quadratic Bezier curve with intuitive parameters.
The proposed scheme boosted the average segmentation accuracy to 94.0026\%, which is a 1.1397\% increase compared to an image with no pre-processing. The ground truth was developed on the original images. However, after the pre-processing, much of the hidden data surfaced which was captured by the segmentation algorithm. If the ground truth was developed on the enhanced image, then much better performance from the proposed method would have been seen.
 % Pre-Processing and Post-Processing Techniques
%%% Chapter Template

\chapter{Semi-Analytical Determination of $\lambda$ and $\sigma$ for Gaussian Smoothing Energies and Non-Uniform Graph Construction for Automatic Graph Cuts} % Main chapter title

\label{chap:Chapter5} % Change X to a consecutive number; for referencing this chapter elsewhere, use \ref{ChapterX}

%----------------------------------------------------------------------------------------
%	SECTION 1
%----------------------------------------------------------------------------------------

\section{Determining FG and BG seeds/probability distributions}


%----------------------------------------------------------------------------------------
%	SECTION 2
%----------------------------------------------------------------------------------------

\section{Determining Optimal Paramters Settings}


%----------------------------------------------------------------------------------------
%	SECTION 3
%----------------------------------------------------------------------------------------

\section{Single-Channel Data}


%----------------------------------------------------------------------------------------
%	SECTION 4
%----------------------------------------------------------------------------------------

\section{Multi-Channel Data}
 % Automatic Segmentation
% Chapter Template

\chapter{Graph Cut Solution to Chan-Vese Segmentation} % Main chapter title

\label{Chapter6} % Change X to a consecutive number; for referencing this chapter elsewhere, use \ref{ChapterX} % Graph Cut Solution to Chan-Vese Segmentation
% Chapter Template

\chapter{Interactive Segmentation on Single Channel Data} % Main chapter title

\label{chap:Chapter3} % Change X to a consecutive number; for referencing this chapter elsewhere, use \ref{ChapterX}

%----------------------------------------------------------------------------------------
%	SECTION 1
%----------------------------------------------------------------------------------------

\section{Seeding}

Lorem ipsum dolor sit amet, consectetur adipiscing elit. Aliquam ultricies lacinia euismod. Nam tempus risus in dolor rhoncus in interdum enim tincidunt. Donec vel nunc neque. In condimentum ullamcorper quam non consequat. Fusce sagittis tempor feugiat. Fusce magna erat, molestie eu convallis ut, tempus sed arcu. Quisque molestie, ante a tincidunt ullamcorper, sapien enim dignissim lacus, in semper nibh erat lobortis purus. Integer dapibus ligula ac risus convallis pellentesque.

%----------------------------------------------------------------------------------------
%	SECTION 2
%----------------------------------------------------------------------------------------

\section{Estimating Probability Distributions}
% distinguish between supervised and unsupervised learning

Sed ullamcorper quam eu nisl interdum at interdum enim egestas. Aliquam placerat justo sed lectus lobortis ut porta nisl porttitor. Vestibulum mi dolor, lacinia molestie gravida at, tempus vitae ligula. Donec eget quam sapien, in viverra eros. Donec pellentesque justo a massa fringilla non vestibulum metus vestibulum. Vestibulum in orci quis felis tempor lacinia. Vivamus ornare ultrices facilisis. Ut hendrerit volutpat vulputate. Morbi condimentum venenatis augue, id porta ipsum vulputate in. Curabitur luctus tempus justo. Vestibulum risus lectus, adipiscing nec condimentum quis, condimentum nec nisl. Aliquam dictum sagittis velit sed iaculis. Morbi tristique augue sit amet nulla pulvinar id facilisis ligula mollis. Nam elit libero, tincidunt ut aliquam at, molestie in quam. Aenean rhoncus vehicula hendrerit.

%-----------------------------------
%	SUBSECTION 1
%-----------------------------------
\subsection{Expectation Maximisation (Guassian Mixture Modelling)}

Nunc posuere quam at lectus tristique eu ultrices augue venenatis. Vestibulum ante ipsum primis in faucibus orci luctus et ultrices posuere cubilia Curae; Aliquam erat volutpat. Vivamus sodales tortor eget quam adipiscing in vulputate ante ullamcorper. Sed eros ante, lacinia et sollicitudin et, aliquam sit amet augue. In hac habitasse platea dictumst.

%-----------------------------------
%	SUBSUBSECTION 1
%-----------------------------------
\subsubsection{Fixed-Distribution Modelling}

Nunc posuere quam at lectus tristique eu ultrices augue venenatis. Vestibulum ante ipsum primis in faucibus orci luctus et ultrices posuere cubilia Curae; Aliquam erat volutpat. Vivamus sodales tortor eget quam adipiscing in vulputate ante ullamcorper. Sed eros ante, lacinia et sollicitudin et, aliquam sit amet augue. In hac habitasse platea dictumst.

%-----------------------------------
%	SUBSUBSECTION 2
%-----------------------------------
\subsubsection{Automatically Determining the Optimal Number of Mixtures}

Nunc posuere quam at lectus tristique eu ultrices augue venenatis. Vestibulum ante ipsum primis in faucibus orci luctus et ultrices posuere cubilia Curae; Aliquam erat volutpat. Vivamus sodales tortor eget quam adipiscing in vulputate ante ullamcorper. Sed eros ante, lacinia et sollicitudin et, aliquam sit amet augue. In hac habitasse platea dictumst.

%-----------------------------------
%	SUBSECTION 2
%-----------------------------------

\subsection{Naive Bayesian Classification}
Morbi rutrum odio eget arcu adipiscing sodales. Aenean et purus a est pulvinar pellentesque. Cras in elit neque, quis varius elit. Phasellus fringilla, nibh eu tempus venenatis, dolor elit posuere quam, quis adipiscing urna leo nec orci. Sed nec nulla auctor odio aliquet consequat. Ut nec nulla in ante ullamcorper aliquam at sed dolor. Phasellus fermentum magna in augue gravida cursus. Cras sed pretium lorem. Pellentesque eget ornare odio. Proin accumsan, massa viverra cursus pharetra, ipsum nisi lobortis velit, a malesuada dolor lorem eu neque.

%-----------------------------------
%	SUBSECTION 3
%-----------------------------------

\subsection{Supervised Learning for Multi-Layered Perceptron}
Morbi rutrum odio eget arcu adipiscing sodales. Aenean et purus a est pulvinar pellentesque. Cras in elit neque, quis varius elit. Phasellus fringilla, nibh eu tempus venenatis, dolor elit posuere quam, quis adipiscing urna leo nec orci. Sed nec nulla auctor odio aliquet consequat. Ut nec nulla in ante ullamcorper aliquam at sed dolor. Phasellus fermentum magna in augue gravida cursus. Cras sed pretium lorem. Pellentesque eget ornare odio. Proin accumsan, massa viverra cursus pharetra, ipsum nisi lobortis velit, a malesuada dolor lorem eu neque.

%----------------------------------------------------------------------------------------
%	SECTION 3
%----------------------------------------------------------------------------------------

\section{Single-Channel Data}

Lorem ipsum dolor sit amet, consectetur adipiscing elit. Aliquam ultricies lacinia euismod. Nam tempus risus in dolor rhoncus in interdum enim tincidunt. Donec vel nunc neque. In condimentum ullamcorper quam non consequat. Fusce sagittis tempor feugiat. Fusce magna erat, molestie eu convallis ut, tempus sed arcu. Quisque molestie, ante a tincidunt ullamcorper, sapien enim dignissim lacus, in semper nibh erat lobortis purus. Integer dapibus ligula ac risus convallis pellentesque.

%----------------------------------------------------------------------------------------
%	SECTION 4
%----------------------------------------------------------------------------------------

\section{Multi-Channel Data}

Lorem ipsum dolor sit amet, consectetur adipiscing elit. Aliquam ultricies lacinia euismod. Nam tempus risus in dolor rhoncus in interdum enim tincidunt. Donec vel nunc neque. In condimentum ullamcorper quam non consequat. Fusce sagittis tempor feugiat. Fusce magna erat, molestie eu convallis ut, tempus sed arcu. Quisque molestie, ante a tincidunt ullamcorper, sapien enim dignissim lacus, in semper nibh erat lobortis purus. Integer dapibus ligula ac risus convallis pellentesque. % Interactive Segmentation
%%% Chapter Template

\chapter{Texture Segmentation} % Main chapter title

\label{chap:Chapter9} % Change X to a consecutive number; for referencing this chapter elsewhere, use \ref{ChapterX}
 % Texture Segmentation (and Intensity)
% Chapter Template

\chapter{Conclusion} % Main chapter title

\label{chap:Chapter7} % Change X to a consecutive number; for referencing this chapter elsewhere, use \ref{ChapterX}

We have shown, thoroughly and comprehensively, that the proposed methods are superior to existing methods. In some cases, there is a greater than 12\% increase in the accuracy of the results over a greater variation of fluorescence images. We have succeeded in moving a few steps closer to fully automated analysis. However, that is not to say that there are no limitations with the proposed methods.

%----------------------------------------------------------------------------------------
%	SECTION 1
%----------------------------------------------------------------------------------------
\section{Limitations}

\begin{enumerate}
	\item The remapping function presented in \Cref{sec:contrastcorrection} has 5 tuning parameters. Although there is greater freedom in tuning the curve, the added complexity in parameter finding is not welcomed.
	
	\item The pre-processing scheme presented in \Cref{chap:Chapter4} has a total of 14 tuning parameters. This makes it difficult to devise parameter estimation methods.
	
	\item In the proposed parameter settings for the proposed parameter estimation method for ACWE graph cut segmentation in \Cref{sec:cvgc_weightingandparameterestimation}, we maintained a proportional relation between $\lambda_0$ and $\lambda_1$. Similarly, the $\alpha$ parameter remains constant and is not subject to appropriate variation with the image. 
\end{enumerate}

%----------------------------------------------------------------------------------------
%	SECTION 2
%----------------------------------------------------------------------------------------
\section{Future Work and Extensions}

\begin{enumerate}
	\item The parameters for the remapping function can be estimated from the image. This will allow the pre-processing scheme to be more adaptable over a greater variety of images and reduce manual parameter tuning.
	
	\item For the proposed parameter estimation scheme, we used simple mathematical relations. More sophisticated, comprehensive and accurate mathematical relations might allow for better segmentation results and applicability to a large class of images.
	
	\item Devise a method for automatic seeding and use those seeds as input for the proposed method in \Cref{sec:interactiveproposedweighting}. This will allow an interactive segmentation technique to become automatic.
	
	\item The parameter estimation technique in \Cref{sec:cvgc_weightingandparameterestimation} can be extended to colour images.
	
	\item The energy function proposed \Cref{sec:interactiveproposedweighting} can be extended to colour images.
\end{enumerate}

 % Conclusion and Future Work

%----------------------------------------------------------------------------------------
%	THESIS CONTENT - APPENDICES
%----------------------------------------------------------------------------------------

\appendix % Cue to tell LaTeX that the following "chapters" are Appendices

% Include the appendices of the thesis as separate files from the Appendices folder
% Uncomment the lines as you write the Appendices

% Appendix Template

\chapter{Introduction to Graph Theory} % Main appendix title

\label{AppendixB} % Change X to a consecutive letter; for referencing this appendix elsewhere, use \ref{AppendixX}

\begin{definition}[Graph]\label{def_graph}
	A graph $G$ is a pair $(V,E)$, where $V$ is the set of nodes/vertices and $E$ is the set of edges consisting of pairs $(u,v)$ where $u,v \in V$. The graph is assumed to be finite i.e. $|V| = n$ and $|E| = m$.
	
	In an \textbf{undirected graph}, the edge $(u,v)$ and $(v,u)$ are not distinct. That is, they refer to the same edge. However, in a \textbf{directed graph}, the two edge are now distinct. In a directed graph with edge $(u,v)$, $u$ is known as the \textbf{tail} and $v$ is know as the \textbf{head}. In directed graphs, edges, also known as arcs, are depicted by placing arrowheads at the head of the edge. Given an edge $e = (u,v)$, $u$ and $v$ are said to be \textbf{incident} on $e$. A graph is said to be \textbf{simple} if it does not contain any self-loops. A \textbf{self-loop} is an edge with of its end points being the same vertex.
\end{definition}

\tikzstyle{vertex}=[circle,thick,draw]
\tikzstyle{edge} = [draw=black!24, very thick,-]
\tikzstyle{weight} = [font=\small]
\begin{figure}[!h]
	\centering
	\resizebox {\columnwidth} {!} {
		\begin{tikzpicture}[scale=1.5, auto, swap, background rectangle/.style={fill=blue!10}, show background rectangle]
		\draw[black] node at (-0.2,2.8) [font=\large,right,rounded corners,inner sep=1ex] {$\textbf{G}$};
		% draw the vertices
		\foreach \pos/\name in {{(0,2)/a}, {(2,2)/b}, {(4,1)/c},{(0,0)/d}, {(3,0)/e}, 
			{(4,-1)/g}, {(1,-1)/f}}
		\node[vertex] (\name) at \pos {$\name$};
		% connect vertices with edges and draw weights
		\foreach \source/ \dest /\weight in {b/a/10,c/b/3,d/a/5,d/b/7, e/b/9, e/c/7,e/d/5,
			f/d/3,f/e/1, g/e/3,g/f/5}
		\path[edge] (\source) -- node[weight] {$\weight$} (\dest);
		% info box
		\draw[yshift=0cm,xshift=4.5cm]
		node [right,text width=5cm,rounded corners,fill=red!20,inner sep=1ex]
		{
			$V_{G} = \{a,b,c,d,e,f,g\}$, \\
			$|V_{G}| = 7$\\
			$E_{G} = \{ab,ad,bc,bd,be,ce,de,df,ef,$\\$eg,fg\}$, \\ 
			$|E_{G}| = 11$
		};	
		% degrees
		\draw[orange] node at (-0.2,2.4) [font=\footnotesize,right,rounded corners,inner sep=1ex] {$d(a)=2$};
		\draw[orange] node at (1.8,2.4) [font=\footnotesize,right,rounded corners,inner sep=1ex] {$d(b)=3$};
		\draw[orange] node at (3.8,1.4) [font=\footnotesize,right,rounded corners,inner sep=1ex] {$d(c)=2$};
		\draw[orange] node at (-0.3,-0.5) [font=\footnotesize,right,rounded corners,inner sep=1ex] {$d(d)=4$};
		\draw[orange] node at (3.2,0.0) [font=\footnotesize,right,rounded corners,inner sep=1ex] {$d(e)=5$};
		\draw[orange] node at (0.8,-1.5) [font=\footnotesize,right,rounded corners,inner sep=1ex] {$d(f)=3$};
		\draw[orange] node at (3.8,-1.5) [font=\footnotesize,right,rounded corners,inner sep=1ex] {$d(g)=2$};
		\end{tikzpicture}
	}
	\caption{Undirected weighted graph \textbf{G}. The degree of each node is shown next to the corresponding node. The graph is simple. The red box shows the vertex set, $V_{G}$, and edge set, $E_{G}$, and their corresponding norm.}
\end{figure}

\begin{definition}[Degree]
	The degree of a vertex $v$ is the number of edges incident on it. $deg(v) = |\{(u,v), (v,u) \in E\}|$. A self-loop counts for 2.
	
	If a graph is directed, also known as a \textbf{digraph}, then a node $v$ has an \textbf{in-degree} $d_{in}(v)$ and an \textbf{out-degree} $d_{out}(v)$. A digraph is said to be \textbf{balanced} if $d_{in}(v) = d_{out}(v), \forall v \in V$.
\end{definition}

\tikzstyle{vertex}=[circle,thick,draw]
\tikzstyle{edge} = [draw=black!24, very thick,<-]
\tikzstyle{weight} = [font=\small]
\begin{figure}[!h]
	\centering
	\resizebox {\columnwidth} {!} {
		\begin{tikzpicture}[scale=1.5, auto, swap, background rectangle/.style={fill=blue!10}, show background rectangle, >={Stealth[black!24]}]
		\draw[black] node at (-0.2,3.1) [font=\large,right,rounded corners,inner sep=1ex] {$\textbf{D}$};
		% draw the vertices
		\foreach \pos/\name in {{(0,2)/a}, {(2,2)/b}, {(4,1)/c},{(0,0)/d}, {(3,0)/e}, 
			{(4,-1)/g}, {(1,-1)/f}}
		\node[vertex] (\name) at \pos {$\name$};
		% connect vertices with edges and draw weights
		\foreach \source/ \dest /\weight in {b/a/10,c/b/3,d/a/5,e/b/9,e/d/5,
			f/d/3,f/e/1, g/e/3,g/f/5}
		\path[edge] (\source) edge node[weight] {$\weight$} (\dest);
		% bend edges
		\path [edge] (b) edge[bend right=20] node {$7$} (d);
		\path [edge] (d) edge[bend left=-20] node {$4$} (b);
		\path [edge] (e) edge[bend right=20] node {$7$} (c); 
		\path [edge] (c) edge[bend right=20] node {$2$} (e);
		% info box
		\draw[yshift=0cm,xshift=4.5cm]
		node [right,text width=5cm,rounded corners,fill=red!20,inner sep=1ex]
		{
			$V_{D} = \{a,b,c,d,e,f,g\}$, \\
			$|V_{D}| = 7$\\
			$E_{D} = \{ab,ad,bc,bd,be,ce,db,de,df,$\\$ec,ef,eg,fg\}$, \\
			$|E_{D}| = 13$
		};
		% degrees
		% a	
		\draw[orange] node at (-0.2,2.7) [font=\footnotesize,right,rounded corners,inner sep=1ex] {$d_{in}=0$};
		\draw[orange] node at (-0.2,2.4) [font=\footnotesize,right,rounded corners,inner sep=1ex] {$d_{out}=2$};	
		% b
		\draw[orange] node at (1.8,2.7) [font=\footnotesize,right,rounded corners,inner sep=1ex] {$d_{in}=2$};
		\draw[orange] node at (1.8,2.4) [font=\footnotesize,right,rounded corners,inner sep=1ex] {$d_{out}=5$};
		% c
		\draw[orange] node at (3.8,1.7) [font=\footnotesize,right,rounded corners,inner sep=1ex] {$d_{in}=2$};
		\draw[orange] node at (3.8,1.4) [font=\footnotesize,right,rounded corners,inner sep=1ex] {$d_{out}=1$};
		% d
		\draw[orange] node at (-0.3,-0.5) [font=\footnotesize,right,rounded corners,inner sep=1ex] {$d_{in}=2$};
		\draw[orange] node at (-0.3,-0.8) [font=\footnotesize,right,rounded corners,inner sep=1ex] {$d_{out}=2$};
		% e
		\draw[orange] node at (3.2,0.08) [font=\footnotesize,right,rounded corners,inner sep=1ex] {$d_{in}=3$};
		\draw[orange] node at (3.2,-0.12) [font=\footnotesize,right,rounded corners,inner sep=1ex] {$d_{out}=3$};
		% f
		\draw[orange] node at (0.8,-1.5) [font=\footnotesize,right,rounded corners,inner sep=1ex] {$d_{in}=2$};
		\draw[orange] node at (0.8,-1.8) [font=\footnotesize,right,rounded corners,inner sep=1ex] {$d_{out}=1$};
		%g
		\draw[orange] node at (3.8,-1.5) [font=\footnotesize,right,rounded corners,inner sep=1ex] {$d_{in}=2$};
		\draw[orange] node at (3.8,-1.8) [font=\footnotesize,right,rounded corners,inner sep=1ex] {$d_{out}=0$};
		\end{tikzpicture}
	}
	\caption{Directed weighted graph (Digraph) \textbf{D}. The in-degree and out-degree is shown next to each node. The graph is simple and not balanced. The red box shows the vertex set, $V_{D}$, and edge set, $E_{D}$, and their corresponding norm.}
\end{figure}

\begin{definition}[Subgraph]
	A graph $G' = (V', E')$ is said to be a sub-graph of $G = (V, E)$, denoted as $G' \subseteq G$, if $V' \subseteq V$ and $E' \subseteq E$.
\end{definition}

\tikzstyle{vertex}=[circle,thick,draw]
\tikzstyle{edge} = [draw=black!24, very thick,-]
\tikzstyle{weight} = [font=\small]
\begin{figure}[!h]
	\centering
	\resizebox {\columnwidth} {!} {
		\begin{tikzpicture}[scale=1.5, auto, swap, background rectangle/.style={fill=blue!10}, show background rectangle]
		\draw[black] node at (-0.2,2.8) [font=\large,right,rounded corners,inner sep=1ex] {$\textbf{H}$};
		% draw the vertices
		\foreach \pos/\name in {{(0,2)/a}, {(2,2)/b}, {(0,0)/d}, {(3,0)/e}}
		\node[vertex] (\name) at \pos {$\name$};
		% connect vertices with edges and draw weights
		\foreach \source/ \dest /\weight in {b/a/10,d/a/5,d/b/7,e/b/9,e/d/5}
		\path[edge] (\source) -- node[weight] {$\weight$} (\dest);
		% info box
		\draw[yshift=-1.6cm,xshift=0cm]
		node [right,text width=5cm,rounded corners,fill=red!20,inner sep=1ex]
		{
			$V_{H} = \{a,b,d,e\} \subseteq V_{G}$, \\
			$|V_{H}| = 4$\\
			$E_{H} = \{ab,ad,bd,be,de\} \subseteq E_{G}$, \\ 
			$|E_{H}| = 5$
		};	
		% degrees
		\draw[orange] node at (-0.2,2.4) [font=\footnotesize,right,rounded corners,inner sep=1ex] {$d(a)=2$};
		\draw[orange] node at (1.8,2.4) [font=\footnotesize,right,rounded corners,inner sep=1ex] {$d(b)=3$};
		\draw[orange] node at (-0.2,-0.4) [font=\footnotesize,right,rounded corners,inner sep=1ex] {$d(d)=4$};
		\draw[orange] node at (2.5,-0.4) [font=\footnotesize,right,rounded corners,inner sep=1ex] {$d(e)=5$};
		
		
		\draw[black] node at (4.2,2.8) [font=\large,right,rounded corners,inner sep=1ex] {$\textbf{I}$};
		% draw the vertices
		\foreach \pos/\name in {{(4.4,2)/a}, {(6.4,2)/b}, {(4.4,0)/d}, {(7.4,0)/e}}
		\node[vertex] (\name) at \pos {$\name$};
		% connect vertices with edges and draw weights
		\foreach \source/ \dest /\weight in {b/a/10,d/a/5,e/b/9,e/d/5}
		\path[draw=black!24, very thick,<-] (\source) -- node[weight] {$\weight$} (\dest);
		% bends
		\path [draw=black!24, very thick,<-] (d) edge[bend left=-20] node {$4$} (b);
		% info box
		\draw[yshift=-1.6cm,xshift=4.4cm]
		node [right,text width=5cm,rounded corners,fill=red!20,inner sep=1ex]
		{
			$V_{I} = \{a,b,d,e\} \subseteq V_{D}$, \\
			$|V_{I}| = 4$\\
			$E_{I} = \{ab,ad,bd,be,de\} \subseteq E_{D}$, \\ 
			$|E_{I}| = 5$
		};	
		% degrees
		% a	
		\draw[orange] node at (4.2,2.5) [font=\footnotesize,right,rounded corners,inner sep=1ex] {$d_{in}=0$};
		\draw[orange] node at (4.2,2.3) [font=\footnotesize,right,rounded corners,inner sep=1ex] {$d_{out}=2$};	
		% b
		\draw[orange] node at (6.2,2.5) [font=\footnotesize,right,rounded corners,inner sep=1ex] {$d_{in}=1$};
		\draw[orange] node at (6.2,2.3) [font=\footnotesize,right,rounded corners,inner sep=1ex] {$d_{out}=2$};
		% d
		\draw[orange] node at (4.1,-0.4) [font=\footnotesize,right,rounded corners,inner sep=1ex] {$d_{in}=2$};
		\draw[orange] node at (4.1,-0.6) [font=\footnotesize,right,rounded corners,inner sep=1ex] {$d_{out}=1$};
		% e
		\draw[orange] node at (7.1,-0.4) [font=\footnotesize,right,rounded corners,inner sep=1ex] {$d_{in}=2$};
		\draw[orange] node at (7.1,-0.6) [font=\footnotesize,right,rounded corners,inner sep=1ex] {$d_{out}=0$};
		\end{tikzpicture}
	}
	\caption{Undirected weighted graph \textbf{H} is a subgraph of \textbf{G} in Figure XX, $\textbf{H} \subseteq \textbf{G}$. Directed weighted graph \textbf{I} is a subgraph of \textbf{D} in Figure XX, $\textbf{I} \subseteq \textbf{D}$. The degree of each node is shown next to the corresponding node. The red box shows the vertex set, the edge set and their corresponding norms.}
\end{figure}


\begin{definition}[Clique]
	A clique is a maximal subgraph.
\end{definition}

\tikzstyle{vertex}=[circle,thick,draw]
\tikzstyle{edge} = [draw=black!24, very thick,-]
\tikzstyle{weight} = [font=\small]
\begin{figure}[!h]
	\centering
	\resizebox {\columnwidth} {!} {
		\begin{tikzpicture}[scale=1.5, auto, swap, background rectangle/.style={fill=blue!10}, show background rectangle]
		\draw[black] node at (-0.2,2.8) [font=\large,right,rounded corners,inner sep=1ex] {$\textbf{G}$};	
		% draw the vertices
		\foreach \pos/\name in {{(0,2)/a}, {(2,2)/b}, {(4,1)/c},{(0,0)/d}, {(3,0)/e}, 
			{(4,-1)/g}, {(1,-1)/f}}
		\node[vertex] (\name) at \pos {$\name$};
		% connect vertices with edges and draw weights
		\foreach \source/ \dest /\weight in {b/a/{},c/b/{},d/a/{},d/b/{},e/b/{},e/c/{},e/d/{},
			f/d/{},f/e/{}, g/e/{},g/f/{}}
		\path[edge] (\source) -- node[weight] {$\weight$} (\dest);
		
		% hyperedges around cliques
		% abd
		\draw[red, thick, rounded corners=1pt] (a.north) -- (a.north west) -- (a.west) -- (d.west) -- (d.south west) -- (d.south) -- (d.south east) -- (b.south east) -- (b.east) -- (b.north east) -- (b.north) -- (a.north);
		% bde
		\draw[green, thick, rounded corners=1pt] (b.north) -- (b.north west) -- (d.north west) -- (d.west) -- (d.south west) -- (d.south) -- (e.south) -- (e.south east) -- (e.east) -- (e.north east) -- (b.north east) -- (b.north);
		% def
		\draw[blue, thick, rounded corners=1pt] (d.north) -- (d.north west) -- (d.west) -- (d.south west) -- (f.south west) -- (f.south) -- (f.south east) -- (e.south east) -- (e.east) -- (e.north east) -- (e.north) -- (d.north);
		% efg
		\draw[red, thick, rounded corners=1pt] (e.north) -- (e.north west) -- (f.north west) -- (f.west) -- (f.south west) -- (f.south) -- (g.south) -- (g.south east) -- (g.east) -- (g.north east) -- (e.north east) -- (e.north);
		% bce
		\draw[orange, thick, rounded corners=1pt] (b.north) -- (b.north west) -- (b.west) -- (b.south west) -- (e.south west) -- (e.south) -- (e.south east) -- (c.south east) -- (c.east) -- (c.north east) -- (b.north east) -- (b.north);
		
		% info box
		\draw[yshift=2.5cm,xshift=3.0cm]
		node [right,text width=7cm,rounded corners,fill=red!20,inner sep=1ex]
		{
			$V_{G} = \{a,b,c,d,e,f,g\}$, \\
			$E_{G} = \{ab,ad,bc,bd,be,ce,de,df,ef,$\\$eg,fg\}$, \\ 
			$C_{G} = V_G \cup E_G \cup \{abd, bce, efg, edf, bde\}$
		};	
		\end{tikzpicture}
	}
	\caption{Cliques of the undirected weighted graph \textbf{G}. The maximal cliques are shown by the hyperedges that encompass the nodes of that clique.}
\end{figure}
% Appendix A

\chapter{Cell Images Dataset} % Main appendix title

\label{AppendixA} % For referencing this appendix elsewhere, use \ref{AppendixA}

The dataset is composed of two subsets. One as the sample set and one as the test set. The sample set is used for tuning parameters and testing theories or predictions. This dataset is composed of images that are relatively simple but still try to maintain some of the variation of images obtained in fluorescence microscopy. The other dataset is the test set. This dataset contains more complex images and is used to test the robustness of the segmentation schemes or techniques. We aim for a larger coverage of the types of images that are frequently obtained in fluorescence microscopy.

\section{Sample Set}
\begin{figure}[!h]
	\centering
	\subfigure[\citep{cil:9233}]
	{
		\includegraphics[width=0.3\columnwidth]{cell_database/1gray.jpg}
		\label{fig:1gray}
	}
	\subfigure[\citep{cil:11996}]
	{
		\includegraphics[width=0.3\columnwidth]{cell_database/2gray.jpg}
		\label{fig:2gray}
	}
	\subfigure[\citep{cil:13902}]
	{
		\includegraphics[width=0.3\columnwidth]{cell_database/3gray.jpg}
		\label{fig:3gray}
	}
	\subfigure[\citep{cil:13903}]
	{
		\includegraphics[width=0.3\columnwidth]{cell_database/4gray.jpg}
		\label{fig:4gray}
	}
	\subfigure[\citep{cil:13904}]
	{
		\includegraphics[width=0.3\columnwidth]{cell_database/5gray.jpg}
		\label{fig:5gray}
	}
	\subfigure[]
	{
		\includegraphics[width=0.3\columnwidth]{cell_database/6gray.jpg}
		\label{fig:6gray}
	}
	\caption{Sample set.}
	\label{fig:sampleset}
\end{figure}

\section{Test Set}

\begin{figure}[!h]
	\centering
	\subfigure[\citep{cil:12627}]
	{
		\includegraphics[width=0.3\columnwidth]{cell_database/12627gray.jpg}
		\label{fig:12627}
	}
	\caption{Uneven Illumination}
	\label{fig:unevenillumination}
\end{figure}

\begin{figure}[!h]
	\centering
	\subfigure[\citep{cil:13899}]
	{
		\includegraphics[width=0.3\columnwidth]{cell_database/13899gray.jpg}
		\label{fig:13899_2}
	}
	\subfigure[\citep{cil:13901}]
	{
		\includegraphics[width=0.3\columnwidth]{cell_database/13901gray.jpg}
		\label{fig:13901_2}
	}
	\subfigure[\citep{cil:40217}]
	{
		\includegraphics[width=0.3\columnwidth]{cell_database/40217gray.jpg}
		\label{fig:40217}
	}
	\caption{High cell density}
	\label{fig:highdensity}
\end{figure}

\begin{figure}[!h]
	\centering
	\subfigure[\citep{cil:195}]
	{
		\includegraphics[width=0.3\columnwidth]{cell_database/195gray.jpg}
		\label{fig:195}
	}
	\subfigure[\citep{cil:10102}]
	{
		\includegraphics[width=0.3\columnwidth]{cell_database/10102gray.jpg}
		\label{fig:10102}
	}
	\subfigure[\citep{cil:10104}]
	{
		\includegraphics[width=0.3\columnwidth]{cell_database/10104gray.jpg}
		\label{fig:10104}
	}
	\subfigure[\citep{cil:12294}]
	{
		\includegraphics[width=0.3\columnwidth]{cell_database/12294gray.jpg}
		\label{fig:12294}
	}
	\subfigure[\citep{cil:13899}]
	{
		\includegraphics[width=0.3\columnwidth]{cell_database/13899gray.jpg}
		\label{fig:13899}
	}
	\subfigure[\citep{cil:13901}]
	{
		\includegraphics[width=0.3\columnwidth]{cell_database/13901gray.jpg}
		\label{fig:13901}
	}
	\subfigure[\citep{cil:21749}]
	{
		\includegraphics[width=0.3\columnwidth]{cell_database/21749gray.jpg}
		\label{fig:21749}
	}
	\subfigure[\citep{cil:21759}]
	{
		\includegraphics[width=0.3\columnwidth]{cell_database/21759gray.jpg}
		\label{fig:21759}
	}
	\subfigure[\citep{cil:42451}]
	{
		\includegraphics[width=0.3\columnwidth]{cell_database/42451gray.jpg}
		\label{fig:42451}
	}

	\caption{Multi-modal (non-bi-modal)}
	\label{fig:multimodal}
\end{figure}

\begin{figure}[!h]
	\centering
	\subfigure[\citep{cil:188}]
	{
		\includegraphics[width=0.3\columnwidth]{cell_database/188gray.jpg}
		\label{fig:188}
	}
	\subfigure[\citep{cil:10093}]
	{
		\includegraphics[width=0.3\columnwidth]{cell_database/10093gray.jpg}
		\label{fig:10093}
	}
	\subfigure[\citep{cil:32140}]
	{
		\includegraphics[width=0.3\columnwidth]{cell_database/32140gray.jpg}
		\label{fig:32140}
	}
	\subfigure[\citep{cil:40968}]
	{
		\includegraphics[width=0.3\columnwidth]{cell_database/40968gray.jpg}
		\label{fig:40968}
	}
	\subfigure[\citep{cil:1057}]
	{
		\includegraphics[width=0.3\columnwidth]{cell_database/1057gray.jpg}
		\label{fig:1057}
	}
	\subfigure[\citep{cil:1265}]
	{
		\includegraphics[width=0.3\columnwidth]{cell_database/1265gray.jpg}
		\label{fig:1265}
	}
	\caption{Hazy/Glowing Edges}
	\label{fig:hazyedge}
\end{figure}

\begin{figure}[!h]
	\centering
	\subfigure[\citep{cil:12294}]
	{
		\includegraphics[width=0.3\columnwidth]{cell_database/12294gray.jpg}
		\label{fig:12294_2}
	}
	\subfigure[\citep{cil:195}]
	{
		\includegraphics[width=0.3\columnwidth]{cell_database/195gray.jpg}
		\label{fig:195_2}
	}
	\subfigure[\citep{cil:35278}]
	{
		\includegraphics[width=0.3\columnwidth]{cell_database/35278gray.jpg}
		\label{fig:35278}
	}
	\subfigure[\citep{cil:38974}]
	{
		\includegraphics[width=0.3\columnwidth]{cell_database/38974gray.jpg}
		\label{fig:38974}
	}
	\caption{Thin Tentacles}
	\label{fig:tentacles}
\end{figure}

\begin{figure}[!h]
	\centering
	\subfigure[\citep{cil:228}]
	{
		\includegraphics[width=0.3\columnwidth]{cell_database/228gray.jpg}
		\label{fig:228}
	}
	\subfigure[\citep{cil:41066}]
	{
		\includegraphics[width=0.3\columnwidth]{cell_database/41066gray.jpg}
		\label{fig:41066}
	}
	\subfigure[\citep{cil:37338}]
	{
		\includegraphics[width=0.3\columnwidth]{cell_database/37338gray.jpg}
		\label{fig:37338}
	}
	\subfigure[\citep{cil:37339}]
	{
		\includegraphics[width=0.3\columnwidth]{cell_database/37339gray.jpg}
		\label{fig:37339}
	}
	\subfigure[\citep{cil:13432}]
	{
		\includegraphics[width=0.3\columnwidth]{cell_database/13432gray.jpg}
		\label{fig:13432}
	}
	\subfigure[\citep{cil:13438}]
	{
		\includegraphics[width=0.3\columnwidth]{cell_database/13438gray.jpg}
		\label{fig:13438}
	}
	\caption{Bright Spots and Speckles}
	\label{fig:peckles}
\end{figure}
%%\include{Appendices/AppendixC}

%----------------------------------------------------------------------------------------
%	BIBLIOGRAPHY
%----------------------------------------------------------------------------------------

\printbibliography[heading=bibintoc]

%----------------------------------------------------------------------------------------

\end{document}  
